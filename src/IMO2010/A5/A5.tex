\documentclass{article}

\usepackage{amsmath, amsfonts, amssymb, amsthm}
\usepackage{hyperref}

\setlength{\parindent}{0pt}
\setlength{\parskip}{5pt}

\newcommand{\Z}{\mathbb{Z}}
\newcommand{\Q}{\mathbb{Q}}

\title{IMO 2010 A5}
\author{}
\date{}

\begin{document}

\maketitle



\subsection*{Problem}

Let $G$ be an abelian group.
Suppose that $G$ embeds into the group of functions $\Z^S$ for some set $S$.
Find all functions $f : G \to G$ such that, for any $x, y \in G$,
\[ f(2 f(x) + y) = 3x + f(x + y). \tag{*}\label{2010a5-eq0} \]



\subsection*{Answer}

$x \mapsto -x$.



\subsection*{Solution}

Official solution: \url{https://www.imo-official.org/problems/IMO2010SL.pdf}

We present the official solution in the setting of additive groups as opposed to multiplicative groups.
We only need the fact that $G$ is torsion-free up to obtaining the functional equation $2 f(f(x)) - f(x) - 3x = 0$.
We need the fact that $G$ embeds into a $\Z^S$ for some set $S$ to solve this final functional equation. 

Clearly, $x \mapsto -x$ satisfies~\eqref{2010a5-eq0}.
So it remains to show that it is the only such function.
First, plugging $y = 0$ into~\eqref{2010a5-eq0} yields that for all $x \in G$,
\[ f(2 f(x)) = 3x + f(x). \tag{1}\label{2010a5-eq1} \]
Since $G$ is torsion-free, this implies that $f$ is injective.
Now, replacing $y$ with $2 f(y)$ in~\eqref{2010a5-eq0} and using~\eqref{2010a5-eq1} gives us
\[ f(2 (f(x) + f(y))) = 3x + f(x + 2 f(y)) = 3(x + y) + f(x + y) = f(2 f(x + y)). \]
By injectivity and again, the fact that $G$ is torsion-free, we get $f(x + y) = f(x) + f(y)$ for all $x, y \in G$.
Thus, $f$ is an endomorphism of $G$.
Then~\eqref{2010a5-eq1} becomes $2 f(f(x)) - f(x) - 3x = 0$.
That is, as an element of the endomorphism ring of $G$, $(f + 1)(2f - 3) = 2 f^2 - f - 3 = 0$.
Letting $h = f + 1$, we have $h(2h - 5) = 0$.
We now show that $h = 0$, i.e. $h(x) = 0$ for each $x \in G$.

Let $\phi : G \to \Z^S$ be an embedding.
We use subscript notation for applications of $\phi$.
That is, we denote $\phi(x)$ by $\phi_x$ for any $x \in G$.
Now, fixing $x \in G$, we have $2 h(h(x)) = 5 h(x)$, and thus $2 \phi_{h(h(x))}(s) = 5 \phi_{h(x)}(s)$ for each $s \in S$.
We have $h(x) = 0 \iff \phi_{h(x)} = 0$.
Thus, it remains to show that $\phi_{h(x)}(s) = 0$ for each $s \in S$.
Fixing $s$, we prove by induction that $2^k \mid \phi_{h(x)}(s)$ for each $k \geq 0$ and $x \in G$.
This gives the desired conclusion.

The base case $k = 0$ is obvious.
Now suppose that, for some $k \geq 0$, we have $2^k \mid \phi_{h(x)}(s)$ for all $x \in G$.
Then $5 \phi_{h(x)}(s) = 2 \phi_{h(h(x))}(s)$ is divisible by $2^{k + 1}$.
Since $5$ is coprime with $2$, this shows that $2^{k + 1} \mid  \phi_{h(x)}(s)$ for any $x \in G$.
Induction step is complete; we are done.



\subsection*{Implementation details}

In a separate file, we give the answer for the case where $G$ is the multiplicative group $\Q^+$.
The fact that $\Q^+$ embeds into $\Z^P$ is implemented in the file
\[ \texttt{extra/number\_theory/pos\_rat\_primes.lean}. \]



\end{document}
