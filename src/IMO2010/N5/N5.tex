\documentclass{article}

\usepackage{amsmath, amsfonts, amssymb, amsthm}
\usepackage{hyperref}

\setlength{\parindent}{0pt}
\setlength{\parskip}{5pt}

\newcommand{\N}{\mathbb{N}}

\newtheorem*{claim}{Claim}

\title{IMO 2010 N5 (P3)}
\author{}
\date{}

\begin{document}

\maketitle



\section*{Problem}

Fix some $c \in \N$.
Find all functions $f : \N \to \N$ such that $(f(n) + m + c)(f(m) + n + c)$ is a square for all $m, n \in \N$.



\section*{Answer}

$n \mapsto n + k$ for some $k \in \N$.



\section*{Solution}

Official solution: \url{https://www.imo-official.org/problems/IMO2010SL.pdf}

This solution is a copy of the official solution.
Obviously, functions given in the answer section works.
Thus it remains to prove that no other functions satisfy the problem condition.
The main claim is as follows.

\begin{claim}
Let $p$ be a prime and $x, y \in \N$.
If $f(x) \equiv f(y) \pmod{p}$, then $x \equiv y \pmod{p}$.
\end{claim}
\begin{proof}
It suffices to find a positive integer $n$ such that $\nu_p(n + f(x) + c)$ and $\nu_p(n + f(y) + c)$ are both odd.
Indeed, then $p$ has to divide $f(n) + x + c$ and $f(n) + y + c$, and thus $x \equiv y \pmod{p}$.

First suppose that $f(x) \equiv f(y) \pmod{p^2}$.
Choose a positive integer $N > f(x) + c$ such that $\nu_p(N) = 1$.
Write $n = N - (f(x) + c) > 0$.
Then $\nu_p(n + f(x) + c) = \nu_p(n + f(y) + c) = 1$.
The latter holds since $f(x) \equiv f(y) \pmod{p^2}$.

Now suppose that $f(x) \not\equiv f(y) \pmod{p^2}$.
This time, we choose $N > f(x) + c$ such that $\nu_p(N) = 3$.
Now, writing $n = N - (f(x) + c)$, we have $\nu_p(n + f(x) + c) = 3$ and $\nu_p(n + f(y) + c) = 1$.
The latter holds since $f(x) \not\equiv f(y) \pmod{p^2}$ but $f(x) \equiv f(y) \pmod{p}$.
\end{proof}

Due to the above claim, $f$ is injective.
Indeed, if $f(x) = f(y)$, then $f(x) \equiv f(y) \pmod{p}$ for every $p$.
The claim then implies $x \equiv y \pmod{p}$.
Taking $p$ large enough yields $x = y$.

The claim also implies that $f(x + 1) = f(x) \pm 1$ for all $x \in \N$.
Indeed, it implies $f(x + 1) \not\equiv f(x) \pmod{p}$ for any $x \in \N$ and $p$ prime.
This is a contradiction for some choice of $p$ if $f(x + 1) \neq f(x) \pm 1$.

Furthermore, by injectivity of $f$, one can show that either $f(x + 1) = f(x) + 1$ for all $x \in \N$ or $f(x + 1) = f(x) - 1$ for all $x \in \N$.
The latter is a clear contradiction, so the former holds.
This means that $f(n) = n + f(0)$ for all $n \in \N$, as desired.



\section*{Extra notes}

The above version of the problem with $c = 2$ is actually equivalent to the original problem.
Indeed, then $f$ satisfies the above condition if and only if the map $n \mapsto f(n - 1) + 1$ satisfies the original condition.



\end{document}
