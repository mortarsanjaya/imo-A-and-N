\documentclass{article}

\usepackage{amsmath, amsfonts, amssymb, amsthm}
\usepackage{hyperref}

\setlength{\parindent}{0pt}
\setlength{\parskip}{5pt}

\newcommand{\N}{\mathbb{N}}

\newtheorem{lemma}{Lemma}

\title{IMO 2010 C4 (P5)}
\author{}
\date{}

\begin{document}

\maketitle



\subsection*{Problem}

Given a positive integer $N \geq 3$, let $\N^N$ denote the set of ordered $N$-tuples of non-negative integers.
In this problem, an $N$-tuple $(n_1, n_2, \ldots, n_N)$ of non-negative integers resemble an ordered stack of coins, with $n_i$ coins in the $i^{\text{th}}$ stack for each $i \leq N$.

Given an $N$-tuple $\mathbf{n} = (n_1, n_2, \ldots, n_N) \in \N^N$, we are allowed to do one of the following operations:
\begin{itemize}

    \item
    If $n_k > 0$ for some $k \leq N - 1$, then subtract $1$ from $n_k$ and add $2$ to $n_{k + 1}$.
    That is, from a non-empty stack (except the last one), remove a coin from the stack and add two coins to the stack right next to it.

    \item
    If $n_k > 0$ for some $k \leq N - 2$, then subtract $1$ from $n_k$ and swap $n_{k + 1}$ with $n_{k + 2}$.
    That is, from a non-empty stack (except the last two), remove a coin from the stack and swap the contents of the next two stacks.


\end{itemize}

The game consists of $N = 6$ stacks, starting with $\mathbf{1} = (1, 1, 1, 1, 1, 1)$.
The goal is to reach the state $\mathbf{p} = (0, 0, 0, 0, 0, A)$ using the above operations.
Is there a sequence of operations that change $\mathbf{1}$ to $\mathbf{p}$?



\subsection*{Answer}

Yes.



\subsection*{Solution}

Official solution: \url{https://www.imo-official.org/problems/IMO2010SL.pdf}

We present the official solution below.

Given $\mathbf{n}, \mathbf{p} \in \N^N$, we denote $\mathbf{n} \to \mathbf{p}$ if there is a sequence of (possibly empty) operations that change $\mathbf{n}$ to $\mathbf{p}$.
The main observations are as follows.

\begin{itemize}

    \item
    $(b + k, c) \to (b, c + 2k)$ for any $b, c, k \in \N$.

    This follows by simple induction on $k$.

    \item
    $(a, 0, 0) \to (0, 2^a, 0)$ for any $a \geq 1$.

    For any $b \geq 0$, we have $(b + 1, 0, 0) \to (b, 2, 0)$.
    By the previous observation, we have $(b + 1, k, 0) \to (b + 1, 0, 2k) \to (b, 2k, 0)$ for any $b, k \in \N$.
    Now we are done by induction on $a$.

    \item
    Consider the recursive function $P : \N \to \N$ defined by $P(0) = 1$ and $P(n + 1) = 2^{P(n)}$ for any $n \in \N$.
    Then $(a, 0, 0, 0) \to (0, P(a), 0, 0)$ for any $a \geq 1$.

    Again, start with $(b + 1, 0, 0, 0) \to (b, 2, 0, 0)$; note that $P(1) = 2$.
    By the above second observation, for any $b \geq 0$ and $k \geq 1$, $(b + 1, k, 0, 0) \to (b + 1, 0, 2^k, 0) \to (b, 2^k, 0, 0)$.
    The rest follows by induction on $a$.

\end{itemize}

For any $N \geq 2$, let $\mathbf{1}_N = (1, 1, \ldots, 1) \in \N^N$.
Then $\mathbf{1}_N \to (0, 3, 0, 0, \ldots, 0)$.
Indeed, $\mathbf{1}_N \to (1, 1, \ldots, 1, 0, 3)$.
Then repeat the second type of operation if necessary.
In particular, for $N = 6$, we get $(1, 1, 1, 1, 1, 1) \to (0, 3, 0, 0, 0, 0)$.
It remains to show that $(0, 3, 0, 0, 0, 0) \to (0, 0, 0, 0, 0, A)$.
As in the official solution, we only need the fact that $A \leq P(16)$.

Apply the second observation twice.
Since $P_3 = 16$, we get $(0, 3, 0, 0, 0, 0) \to (0, 0, 0, P(16), 0, 0)$.
By induction using the second type of operation, one has $(k, 0, 0) \to (m, 0, 0)$ for any $m \leq k$.
Since $A \leq P(16)$, we have $(0, 0, 0, A/4, 0, 0)$; note that $4 \mid A$.
Using the first type of operation twice finishes the solution:
\[ (0, 0, 0, A/4, 0, 0) \to (0, 0, 0, 0, 0, A). \]



\subsection*{Implementation details}

We implement the $6$-tuple as a function from $\N$ to itself instead of a function from a totally ordered type of cardinality $6$ to $\N$.
This makes it easier to implement the legal operations.



\end{document}
