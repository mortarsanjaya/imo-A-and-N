\documentclass{article}

\usepackage{amsmath, amsfonts, amssymb, amsthm}
\usepackage{hyperref}

\setlength{\parindent}{0pt}
\setlength{\parskip}{5pt}

\newcommand{\Z}{\mathbb{Z}}

\title{IMO 2010 A1 (P1)}
\author{}
\date{}

\begin{document}

\maketitle



\subsection*{Problem}

A \emph{floor function} $\lfloor \cdot \rfloor : R \to \Z$ on a totally ordered ring $R$ is a function such that, for any $r \in R$ and $n \in \Z$, $n \leq \lfloor r \rfloor$ if and only if $n \leq r$ in $R$.

Let $F$ and $R$ be a totally ordered field and a totally ordered ring, respectively, both equipped with a floor function.
Find all functions $f : F \to R$ such that, for all $x, y \in F$,
\[ f(\lfloor x \rfloor y) = f(x) \lfloor f(y) \rfloor. \tag{*}\label{eq0} \]



\subsection*{Solution}

Official solution: \url{https://www.imo-official.org/problems/IMO2010SL.pdf}

Although we start with the same substitution as Solution 1, we make a shortcut in the case $f(0) = 0$.

First, plugging $x = y = 0$ into~\eqref{eq0} yields $f(0) = f(0) \lfloor f(0) \rfloor$.
Thus either $\lfloor f(0) \rfloor = 1$ or $f(0) = 0$.

In the former case, plugging $y = 0$ into~\eqref{eq0} yields $f(x) = f(0)$ for all $x \in F$.
That is, $f \equiv C$ for some constant $C$.
Furthermore, $\lfloor f(0) \rfloor = 1$ forces $C \in [1, 2)$.

In the latter case, we claim that $f \equiv 0$.
Indeed, plugging $x = 1$ into~\eqref{eq0} yields $f(y) = f(1) \lfloor f(y) \rfloor$ for all $y \in F$.
In the view of this equality, it suffices to show that $f(1) = 0$.
Plugging $x = y = 1/2$ into~\eqref{eq0} yields $f(1/2) \lfloor f(1/2) \rfloor = f(0) = 0$, so $\lfloor f(1/2) \rfloor = 0$.
Then plugging $x = 2$ and $y = 1/2$ into~\eqref{eq0} yields $f(1) = 0$, as desired.




\end{document}
