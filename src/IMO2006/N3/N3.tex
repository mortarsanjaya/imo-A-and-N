\documentclass{article}

\usepackage{amsmath, amsfonts, amssymb, amsthm}
\usepackage{hyperref}

\setlength{\parindent}{0pt}
\setlength{\parskip}{5pt}

\newcommand{\N}{\mathbb{N}}

\title{IMO 2006 N3}
\author{}
\date{}

\begin{document}

\maketitle



\subsection*{Problem}

For each $n \in \N^+$, denote
\[ f(n) = \frac{1}{n} \sum_{k = 1}^n \left\lfloor \frac{n}{k} \right\rfloor. \]
\begin{enumerate}
    \item Prove that there exists infinitely many $n \in \N^+$ such that $f(n + 1) > f(n)$.
    \item Prove that there exists infinitely many $n \in \N^+$ such that $f(n + 1) < f(n)$.
\end{enumerate}



\subsection*{Solution}

Official solution: \url{https://www.imo-official.org/problems/IMO2006SL.pdf}

The following solution is simply a rewrite of the official solution.

For each positive integer $n$, let $d(n)$ denote the number of divisors of $n$.
Also, let $g(n) = n f(n)$.
First of all, it is well-known that
\[ g(n) = \sum_{k = 1}^n \left\lfloor \frac{n}{k} \right\rfloor = \sum_{k = 1}^n d(k). \]
In particular, we have $g(n + 1) = g(n) + d(n + 1)$ for any $n \geq 1$.

\begin{enumerate}

\item
Notice that $f(n + 1) > f(n) \iff d(n + 1) > f(n) \iff n d(n + 1) > g(n)$.
This immediately follows if $d(n + 1) > d(k)$ for all $k \leq n$.
Thus, to solve part 1, it suffices to show that there exists infinitely many $n \geq 1$ with such property.
Equivalently, given $N \geq 1$ arbitrary, there exists $n > N$ with the property.

This follows by the fact that $d$ is unbounded.
Indeed, given $N \geq 1$, since $d$ is unbounded, there exists $x \geq 1$ such that $d(x + 1) > N + 1$.
Now pick the minimal such $x$, so that $d(k + 1) \leq N + 1$ for all $k < x$.
Note that $N \geq 1 = d(1)$, so $d(x + 1) > N + 1 \geq d(k)$ for all $k \leq x$.
It remains to show that $x > N$, which is immediate from $x + 1 \geq d(x + 1) > N + 1$.

\item
This time, $f(n + 1) < f(n) \iff d(n + 1) < f(n) \iff n d(n + 1) < g(n)$.
The inequality follows if $n + 1$ is prime and $g(n) > 2n$.
Clearly, there exists infinitely many primes.
So, it suffices to prove that $g(n) > 2n$ for $n$ big enough.
We will prove this inequality for $n \geq 6$ by induction.

As $d(n) \geq 2$ for all $n \geq 3$, the induction step is immediate.
It remains to verify the base case: $g(6) > 12$.
Indeed, one can verify by hand that $g(6) = 14$.

\end{enumerate}



\subsection*{Implementation details}

We work with the \texttt{nat} datatype.
Our implementation of $f$ yields $f(0) = 0$.
Also, instead of working directly with $f$, we work most of the time with $g$.
The advantage is that $g$ is properly a map from $\N$ to itself, with $g(0) = 0$.

The lemmas \texttt{lem1} to \texttt{lem5} implement some basic properties of $g$ and the divisor counting function $d$.
For the solution to both problem, we first reduce to showing that a certain subset is infinite.
We then prove formally that these subsets are indeed infinite.



\end{document}
