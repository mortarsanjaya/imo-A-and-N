\documentclass{article}

\usepackage{amsmath, amsfonts, amssymb, amsthm}
\usepackage{hyperref}

\setlength{\parindent}{0pt}
\setlength{\parskip}{5pt}

\newcommand{\Z}{\mathbb{Z}}
\newcommand{\N}{\mathbb{N}}

\newtheorem{lemma}{Lemma}

\title{IMO 2006 A1}
\author{}
\date{}

\begin{document}

\maketitle



\subsection*{Problem}

Let $R$ be an archimedean (totally ordered) ring.
It is naturally equipped with a floor function $\lfloor \cdot \rfloor : R \to \Z$ such that, for any $r \in R$ and $n \in \Z$, $n \leq \lfloor r \rfloor$ if and only if $n \leq r$ in $R$.
We denote $\{r\} = r - \lfloor r \rfloor$ for any $r \in R$.

Define $f : R \to R$ by $f(r) = \lfloor r \rfloor \{r\}$.
Prove that, for any $x \in R$, where exists $N \geq 0$ such that $f^{n + 2}(x) = f^n(x)$ for all $n \geq N$.



\subsection*{Solution}

Official solution: \url{https://www.imo-official.org/problems/IMO2006SL.pdf}

The solution we present is a rewriting of the official solution in a setting that works on rings, not just fields.
We will also use several lemmas to represent the solution.
Some of them are straightforward, and we will omit their proof.

The first three lemmas allow us to solve the case $x \geq 0$.
In fact, we can say something stronger: $f^{\lfloor x \rfloor + 1}(x) = 0$.
This suffices, since clearly $f(0) = 0$.

\begin{lemma}\label{2006a1-1}
For any $r \in R$ with $r \geq 0$, $f(r) \geq 0$.
\end{lemma}

\begin{lemma}\label{2006a1-2}
For any $r \in R$ with $0 \leq r < 1$, $f(r) = 0$.
\end{lemma}

\begin{lemma}\label{2006a1-3}
For any $r \in R$ with $r \geq 1$, $\lfloor f(r) \rfloor < \lfloor r \rfloor$.
\end{lemma}

Now, suppose that $x \geq 0$, and let $N = \lfloor x \rfloor$.
Due to Lemma~\ref{2006a1-2}, it suffices to show that $\lfloor f^N(x) \rfloor = 0$.
More generally, we have by induction on $n$ that $0 \leq \lfloor f^n(x) \rfloor \leq N - n$ for any $n \in \N$ with $n \leq N$.
The base case $n = 0$ is clear.
For the induction step, suppose that this inequality holds for some $0 \leq n < N$.
If $\lfloor f^n(x) \rfloor = 0$, then Lemma~\ref{2006a1-3} implies $f^{n + 1}(x) = 0$, which implies $0 \leq \lfloor f^{n + 1}(x) \rfloor \leq N - (n + 1)$.
Otherwise, $\lfloor f^n(x) \rfloor \geq 1$, so Lemma~\ref{2006a1-1} and Lemma~\ref{2006a1-2} implies
\[ 0 \leq \lfloor f^{n + 1}(x) \rfloor < \leq \lfloor f^{n}(x) \rfloor \leq N - n. \]
And thus, $\lfloor f^{n + 1}(x) \rfloor \leq N - (n + 1)$.
Induction step is complete.

We now focus our attention to the case $x < 0$.
Similar to the previous case, we can analyze the sequence $(\lfloor f^n(r) \rfloor)_{n \geq 0}$.
However, we will not get a concrete bound on $N$.
The main result for this case is Lemma~\ref{2006a1-8}.

\begin{lemma}\label{2006a1-4}
For any $r \in R$ with $r \leq 0$, $f(r) \leq 0$.
\end{lemma}
    
\begin{lemma}\label{2006a1-5}
For any $r \in R$ with $r \leq 0$, $\lfloor r \rfloor \leq \lfloor f(r) \rfloor$.
\end{lemma}
    
\begin{lemma}\label{2006a1-6}
For any $r \in R$ with $-1 < r < 0$, $f(r) = -1 - r$ and $f^2(r) = r$.
\end{lemma}

\begin{lemma}\label{2006a1-7}
Fix some $k \in \N$, and suppose that there exists $r \in R$ such that $(k + 1)r = -k^2$.\footnote{This is equivalent to $n + 1$ being invertible in $R$.}
Then $f(r) = r$.
\end{lemma}
\begin{proof}
If $k = 0$ then $r = 0$, and clearly $f(0) = 0$.
So now, suppose that $k > 0$.
Since $(k + 1)k > k^2 > (k + 1)(k - 1)$, we have $-k - 1 < r < -k$.
Thus, $\lfloor r \rfloor = -k$, and so we get
\[ f(r) = -k (r + k) = r -(k + 1)r - k^2 = r. \]
\end{proof}

\begin{lemma}\label{2006a1-8}
Fix some $r \in R$ and $k \in \N$ with $k > 1$.
Suppose that $\lfloor f^n(r) \rfloor = -k$ for all $n \geq 0$.
Then $(k + 1)r = -k^2$.
\end{lemma}
\begin{proof}
Let $c = (k + 1)r + k^2$.
For any $t \in R$ with $\lfloor t \rfloor = -k$, we have $f(t) = -k(t + k)$.
Some manipulation gives us $(k + 1)f(t) + k^2 = -k((k + 1)t + k^2)$.
Thus, by small induction, we have $(k + 1) f^n(r) + k^2 = (-k)^n ((k + 1) r + k^2) = (-k)^n c$ for all $n \geq 0$.
In particular, we have $(k + 1) (f^n(r) - r) = ((-k)^n - 1) c$, and thus $(k + 1) |f^n(r) - r| \geq (k^n - 1) |c|$.
But we always have $|f^n(r) - r| < 1$ since $\lfloor f^n(r) \rfloor = \lfloor r \rfloor = -k$.
Thus, for any $n \geq 0$, we have $k + 1 > (k^n - 1) |c|$.
Since $R$ is an archimedean ring and $k > 1$, this necessarily implies $c = 0$.
This proves the lemma.
\end{proof}

Now, suppose that $x < 0$.
We prove that either $-1 < f^N(x) < 0$ for some $N \geq 0$ or $(k + 1) f^N(x) = -k^2$ for some $k, N \geq 0$.
The latter implies that $f^{N + n}(x)$ equals $-1 - f^N(x)$ if $n$ is odd and $f^N(x)$ if $n$ is even.
For the latter, Lemma~\ref{2006a1-7} yields $f^n(x) = f^N(x)$ for all $n \geq N$.

Indeed, by Lemma~\ref{2006a1-4}, Lemma~\ref{2006a1-5}, and induction, we get $\lfloor f^n(r) \rfloor \leq \lfloor f^{n + 1}(r) \rfloor \leq 0$ for all $n \geq 0$.
Thus the sequence $(\lfloor f^n(r) \rfloor)_{n \geq 0}$, consisting of integers, is non-decreasing but bounded above by $0$.
As a result, it has to eventually stabilize.
That is, there exists $N, k \geq 0$ such that $\lfloor f^n(x) \rfloor = -k$ for all $n \geq N$.
If $k > 1$, then Lemma~\ref{2006a1-8} yields $(k + 1) f^N(x) = -k^2$.
If $k = 1$, then $-1 \leq f^N(x) < 0$.
But $f^N(x) = -1$ yields $f^{N + 1}(x) = 0$, so in fact we have $-1 < f^N(x) < 0$.
If $k = 0$, then $f^{N + 1}(x) = 0$ by Lemma~\ref{2006a1-2}.
We are done.



\subsection*{Implementation details}

As one can see, we gain more information than just $(f^n(x))_{n \geq 0}$ being eventually $2$-periodic.
Thus, we will implement each of the lemmas separately.



\end{document}
