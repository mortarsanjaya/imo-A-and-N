\documentclass{article}

\usepackage{amsmath, amsfonts, amssymb, amsthm}
\usepackage{hyperref}

\setlength{\parindent}{0pt}
\setlength{\parskip}{5pt}

\newcommand{\Z}{\mathbb{Z}}
\newcommand{\N}{\mathbb{N}}

\newtheorem{lemma}{Lemma}

\title{IMO 2006 A1}
\author{}
\date{}

\begin{document}

\maketitle



\subsection*{Problem}

Let $R$ be an archimedean (totally ordered) ring.
It is naturally equipped with a floor function $\lfloor \cdot \rfloor : R \to \Z$ such that, for any $r \in R$ and $n \in \Z$, $n \leq \lfloor r \rfloor$ if and only if $n \leq r$ in $R$.
We denote $\{r\} = r - \lfloor r \rfloor$ for any $r \in R$.

Define $f : R \to R$ as $r \mapsto \lfloor r \rfloor \{r\}$.
Prove that, for any $x \in R$, there exists $N \geq 0$ such that $f^{N + 2}(x) = f^N(x)$.

Extra: find all $x \in R$ such that $f(f(x)) = x$.



\subsection*{Answer}

Given $x \in R$, we have $f^2(x) = x$ if and only if one of the following holds:
\begin{itemize}
    
    \item
    $(n + 1) x = -n^2$ for some $n \in \N$.
    In this case, $f(x) = x$.

    \item
    $-1 < x < 0$.
    Then $f(x) = -1 - x$ equals $x$ iff $x = 1/2$.

\end{itemize}



\subsection*{Solution}

Official solution: \url{https://www.imo-official.org/problems/IMO2006SL.pdf}

The solution below is the official solution, rearranged in a way to work over any archimedean ring.

Clearly, for any $x \geq 0$, we have $0 \leq f(x) \leq \lfloor x \rfloor$.
Actually, the second inequality is strict if $x \geq 1$.
This means that if $x$ is non-negative, then $\lfloor f^{\lfloor x \rfloor}(x) \rfloor = 0$.
When $\lfloor x \rfloor = 0$, we get $f(x) = 0$.
So, this means that $f^{\lfloor x \rfloor + 1}(x) = 0$.
We are done, since $f(0) = 0$.

Now, consider the case $x < 0$.
Similar as before, we have $\lfloor x \rfloor \leq \lfloor f(x) \rfloor \leq 0$.
This time, we notice that the integer sequence $\left(\lfloor f^n(x) \rfloor\right)_{n \geq 0}$ has to be eventually constant.
Thus, for some $N \geq 0$, we have $\lfloor f^{N + k}(x) \rfloor = \lfloor f^N(x) \rfloor$ for all $k \geq 0$.

For convenience, let $z = f^N(x)$.
Note that $z \leq 0$, and recall that $\lfloor f^k(z) \rfloor = \lfloor z \rfloor$ for all $k \geq 0$.
It now suffices to prove that either $f(z) = z$, or $0 < z < 1$ and $f(z) = -1 - z$.

To prove this, notice that for any $a, b \in R$ with $\lfloor a \rfloor = \lfloor b \rfloor = n$, we have $f(a) - f(b) = n(a - b)$.
In particular, we get $f^{k + 1}(z) - f^k(z) = m^k(f(z) - z)$ for any $k \geq 0$, where $m = \lfloor z \rfloor$.
If $f(z) \neq z$ and $|m| > 1$, then $|f^{k + 1}(z) - f^k(z)| \geq 1$ for $k$ large enough.
On the other hand, $\lfloor f^{k + 1}(z) \rfloor = \lfloor f^k(z) \rfloor = \lfloor z \rfloor$ forces $|f^{k + 1}(z) - f^k(z)| < 1$.
A contradiction, so $|m| \leq 1$.
Since $z \leq 0$, this forces $-1 \leq z \leq 0$.
Note that $z = -1$ does not work, and $z = 0$ means $f(z) = z = 0$.
Thus, we get the remaining possible case: $-1 < z < 0$.

\subsubsection*{Extra}

For the extra part, we do not need $R$ to be archimedean.
We have seen that the case $x > 0$ is impossible, so now assume that $x \leq 0$.
We have also seen that $\lfloor x \rfloor \leq \lfloor f(x) \rfloor \leq \lfloor f(f(x)) \rfloor$.
Thus, $f(f(x)) = x$ implies $\lfloor f(x) \rfloor = \lfloor x \rfloor$.
Furthermore, this implies $|f(f(x)) - f(x)| = |\lfloor x \rfloor| |f(x) - x|$.
Thus, we have either $f(x) = x$ or $\lfloor x \rfloor = -1$.
The former yields $(n + 1)x = -n^2$, where $n = -\lfloor x \rfloor \geq 0$.
The latter does not work if $x = -1$, and it works otherwise.



\end{document}
