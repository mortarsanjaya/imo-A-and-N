\documentclass{article}

\usepackage{amsmath, amsfonts, amssymb, amsthm}
\usepackage{hyperref}

\setlength{\parindent}{0pt}
\setlength{\parskip}{5pt}

\newtheorem{lemma}{Lemma}

\title{IMO 2006 N5}
\author{}
\date{}

\begin{document}

\maketitle



\subsection*{Problem}

Determine all triples $(x, y, p)$ of integers, with $p$ prime, such that
\[ \sum_{k = 0}^{p - 1} x^k = y^{p - 2} - 1. \tag{*}\label{2006n5-eq0} \]



\subsection*{Answer}

$(-1, t, 2)$ and $(t, t^2 + t + 2, 3)$ for some integer $t$.



\subsection*{Solution}

Official solution: \url{https://www.imo-official.org/problems/IMO2006SL.pdf}

We follow the official solution, with the whole idea being the following lemma.

\begin{lemma}\label{2006n5-1}
Let $p$, $q$ be primes and $x$ be an integer such that $q \mid \sum_{k = 0}^{p - 1} x^k$.
Then either $q \equiv 1 \pmod{p}$ or $q = p$.
\end{lemma}
\begin{proof}
First note that $q \nmid x$.
If $x \equiv 1 \pmod{q}$ then we get $q \mid p$, which implies $q = p$.
Otherwise, $q \mid x^p - 1$ but $q \nmid x - 1$, so the order of $x$ modulo $q$ is not $1$ but divides $p$.
Since $p$ is prime, that order must be exactly $p$.
Thus $p \mid q - 1$.
\end{proof}

In particular, the same can be said if we replace $q$ with any positive divisor $d$ of $\sum_{k = 0}^{p - 1} x^k$.
That is, $d \equiv 0, 1 \pmod{p}$.
In addition, note that $y > 0$, since the LHS is always positive.
This is all we need.

The case $p = 2$ and $p = 3$ are trivial; no need to hard-bash.
The case $p \geq 5$ uses the above lemma: $y - 1 \equiv 0, 1 \pmod{p}$ since $y - 1$ is a factor of the RHS.
If $y \equiv 1 \pmod{p}$, then another factor, $\sum_{j = 0}^{p - 3} y^j$, is congruent to $p - 3 \pmod{p}$.
But $p - 3 \not\equiv 0, 1 \pmod{p}$ since $p \geq 5$.
Else, if $y \equiv 2 \pmod{p}$, then $y^{p - 2} - 1 \equiv -1/2 \not\equiv 0, 1 \pmod{p}$; a contradiction.



\end{document}
