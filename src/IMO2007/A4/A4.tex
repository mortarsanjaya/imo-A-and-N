\documentclass{article}

\usepackage{fullpage}
\usepackage{amsmath, amsfonts, amssymb, amsthm}
\usepackage{hyperref}

\setlength{\parindent}{0pt}
\setlength{\parskip}{5pt}

\title{IMO 2007 A4}
\author{}
\date{}

\begin{document}

\maketitle



\subsection*{Problem}

Let $G$ be a totally ordered abelian group.
Let $G^+ = \{x \in G : x > 0\}$.
Find all functions $f : G^+ \to G^+$ such that, for any $x, y \in G^+$,
\[ f(x + f(y)) = f(x + y) + f(y). \tag{*}\label{2007a4-eq0} \]



\subsection*{Answer}

$x \mapsto 2x$.



\subsection*{Solution}

Official solution: \url{https://www.imo-official.org/problems/IMO2007SL.pdf}

We follow Solution 1 of the official solution.

We first show that $f(y) > y$ for any $y \in G^+$.
Indeed, by~\eqref{2007a4-eq0}, $f(y) = y$ yields $f(y) = 0$; impossible.
Meanwhile, if $f(y) < y$, then there exists $x \in G^+$ such that $x + f(y) = y$.
Then~\eqref{2007a4-eq0} yields $f(x + y) = 0$; again, impossible.

As a result, there exists a function $g : G^+ \to G^+$ such that $g(y) = f(y) - y$ for any $y \in G^+$.
Plugging into~\eqref{2007a4-eq0}, we get $g(x + y + g(y)) = y + g(x + y)$ for all $x, y \in G^+$.
That is, we have
\[ g(t + g(y)) = g(t) + y \quad \forall t, y \in G^+, y < t. \tag{1}\label{2007a4-eq1} \]

We now focus solely on~\eqref{2007a4-eq1}; our goal reduces to showing that $g$ is the identity.
We start by proving that $g$ is injective.
Indeed, for any $a, b \in G^+$ such that $g(a) = g(b)$, we have
\[ g(a + b) + a = g(a + b + g(a)) = g(a + b + g(b)) = g(a + b) + b. \]

Next, we prove that $g$ is additive, i.e., $g(x) + g(y) = g(x + y)$ for any $x, y \in G^+$.
Indeed, take any $t > x + y$, say $t = 2(x + y)$.
We get
\[ g(t + g(x) + g(y)) = g(t + g(x)) + y = g(t) + x + y = g(t + g(x + y)). \]
By injectivity, we get $g(x) + g(y) = g(x + y)$ for any $x, y \in G^+$.
Thus,~\eqref{2007a4-eq1} simplifies to $g(g(y)) = y$ for any $y \in G^+$.

Finally, we show that $g(y) = y$ for any $y \in G^+$.
Additivity yields that $g$ is strictly increasing.
So now, suppose for the sake of contradiction that $g(y) \neq y$ for some $y \in G^+$.
Then we have either $y > g(y) > g(g(y)) = y$, or $y < g(y) < g(g(y)) = y$.
In either cases, we get a contradiction.



\end{document}
