\documentclass{article}

\usepackage{amsmath, amsfonts, amssymb, amsthm}
\usepackage{hyperref}

\setlength{\parindent}{0pt}
\setlength{\parskip}{5pt}

\title{IMO 2007 N2}
\author{}
\date{}

\begin{document}

\maketitle



\subsection*{Problem}

\begin{enumerate}

\item
Fix some integers $b > 0$ and $n \geq 0$.
Suppose that for each positive integer $k$, there exists an integer $a$ such that $k \mid b - a^n$.
Prove that $b = A^n$ for some integer $A$.

\item
Prove that, for each $p$ prime, there exists an integer $a$ such that $a^8 \equiv 16 \pmod{p}$.

\end{enumerate}



\subsection*{Solution}

Official solution: \url{https://www.imo-official.org/problems/IMO2007SL.pdf}

\begin{enumerate}

\item
As in the official solution, we pick $k = b^2$.
However, there is a very short solution as follows.

Since $b^2 \mid b - c^n$ for some integer $c$, we can write $c^n = b - ab^2 = (1 - ab) b$ for some integer $a$.
But $1 - ab$ and $b$ are coprime, so indeed $b$ must be an $n^{\text{th}}$ power.

\item
If either $2$ or $-2$ is a quadratic residue modulo $p$, then we are done.
Indeed, we can pick an integer $a$ such that $a^2 \equiv \pm 2 \pmod{p}$.
But this implies $a^8 \equiv 16 \pmod{p}$.

Otherwise, $-1$ must be a quadratic residue modulo $p$.
We denote by $i$ an integer such that $i^2 \equiv -1 \pmod{p}$.
Then $(1 + i)^2 \equiv 2i \pmod{p}$, and $(2i)^4 \equiv 16 \pmod{p}$.

\end{enumerate}



\subsection*{Extra notes}

Part 1 is the original IMO 2007 N2 problem.
Meanwhile, part 2 is mentioned in the comment section of the official solution.
It emphasizes that $k$ cannot be restricted to primes only; $16$ is not an $8^{\text{th}}$ power.

The only implementation details worth mentioning is that we use \texttt{zmod p} for part 2.



\end{document}
