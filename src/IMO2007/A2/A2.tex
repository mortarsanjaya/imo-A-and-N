\documentclass{article}

\usepackage{amsmath, amsfonts, amssymb, amsthm}
\usepackage{hyperref}

\setlength{\parindent}{0pt}
\setlength{\parskip}{5pt}

\newcommand{\N}{\mathbb{N}}

\newtheorem{lemma}{Lemma}

\title{IMO 2007 A2}
\author{}
\date{}

\begin{document}

\maketitle



\subsection*{Problem}

A function $f : \N \to \N$ is called \textit{good} if, for any $m, n \in \N$,
\[ f(m + n + 1) \geq f(m) + f(f(n)). \tag{*}\label{eq0} \]
Fix some $N \in \N$.
Determine all $k \in \N$ for which there exists a good function $f$ such that $f(N) = k$.



\subsection*{Answer}

If $N = 0$, then only $k = 0$ works.
If $N > 0$, then any $k \leq N + 1$ works.



\subsection*{Solution}

Official solution: \url{https://www.imo-official.org/problems/IMO2007SL.pdf}

While the original problem uses $\N^+$ instead of $\N$, the same solution works here except for the case $N = 0$.
The usage of $\N$ will be explained later in the "Implementation details" subsection.
We translate the official solution into a solution for our version of the problem.
One exception is the final step of obtaining $f(n) \leq n + 1$, where we will not follow the official solution.

We start with the following lemma on good functions.

\begin{lemma}\label{2007a2-1}
Any good function is non-decreasing.
\end{lemma}
\begin{proof}
Fix a good function $f : \N \to \N$ and some $m, n \in \N$ with $m < n$.
We can then write $n = m + k + 1$ for some $k \in \N$.
Then~\eqref{eq0} yields $f(n) \geq f(m) + f(f(k)) \geq f(m)$, as desired.
\end{proof}

Now, we divide into two cases.

\begin{itemize}

    \item
    \textit{\underline{Case 1.}}
    $N = 0$.

    Clearly, the zero map $n \mapsto 0$ is good, so it remains to show that $f(0) = 0$ for any $f : \N \to \N$ good.
    Fix $f$, and suppose for the sake of contradiction that $f(0) > 0$.
    This also means $f(0) \geq 1$, so~\eqref{eq0} yields $f(1) \geq f(0) + f(f(0)) > f(f(0))$.
    But Lemma~\ref{2007a2-1} yields $f(f(0)) \geq f(1)$; a contradiction.

    \item
    \textit{\underline{Case 2.}}
    $N > 0$.

    We first show that $f(N) \leq N + 1$ for any good function $f : \N \to \N$.
    Then, we construct a good function for $k \leq N$ and $k = N + 1$ separately.

    Set $K = N + 1$, and suppose for the sake of contradiction that $f(N) > K$.
    In particular, Lemma~\ref{2007a2-1} yields $f(f(N)) \geq f(K)$.
    Then~\eqref{eq0} yields $f(m + K) = f(m + N + 1) \geq f(m) + f(K)$ for all $m \in \N$.
    By small induction, we get $f(mK) \geq m f(K)$ for all $m \in \N$.

    By Lemma~\ref{2007a2-1} again, we have $f(K) \geq f(N) > K$, so $f(K^2) \geq K f(K) \geq K (K + 1)$.
    Thus we can write $f(K^2) = K^2 + d + 1$ with $d \geq K - 1 = N$.
    Plugging $m = d$ and $n = K^2$ into~\eqref{eq0} yields
    \[ f(K^2 + d + 1) \geq f(d) + f(f(K^2)) = f(d) + f(K^2 + d + 1) \implies f(d) \leq 0. \]
    A contradiction, since Lemma~\ref{2007a2-1} gives us $f(d) \geq f(N) > K \geq 0$.
    Thus, $f(N) \leq K = N + 1$, as desired.

    Now, for $k \leq N$, consider the function $f(n) = \max\{0, n - c\}$, where $c = N - k \geq 0$.
    Clearly, $f(N) = k$, so it remains to show that $f$ is good.
    First note that $f$ is non-decreasing and $f(n) \leq n$ for all $n \in \N$.
    Thus,~\eqref{eq0} is satisfied for $n \leq c$ as then $f(f(n)) = 0$.
    For $n > c$,
    \[ f(m + n + 1) = m + (n - c) + 1 \geq f(m) + f(n) + 1 > f(m) + f(f(n)). \]

    Finally, consider the case $k = N + 1$.
    We set $f(n) = n$ if $n \not\equiv -1 \bmod{N + 1}$ and $f(n) = n + 1$ if $n \equiv -1 \bmod{N + 1}$.
    Clearly $f(N) = N + 1$, so it remains to show that $f$ is good.
    
    We have $f(f(n)) = f(n) \leq n + 1$ for all $n \in \N$.
    So, $f(m + n + 1) \geq m + n + 1 \geq f(m) + f(n)$ holds as long as $N + 1$ does not divide $m + 1$ and $n + 1$.
    But if $N + 1$ divides both $m + 1$ and $n + 1$, then $N + 1$ divides $(m + n + 1) + 1$.
    So one can verify that the inequality still holds.

\end{itemize}



\subsection*{Implementation details}

We use $\N$/\texttt{nat} as opposed to $\N^+$/\texttt{pnat} for the main formulation.
The main reason is a better Lean API for \texttt{nat} to implement the above solution.
However, we will still implement the \texttt{pnat} version of the problem.
This is done through the bijection $f \mapsto \lambda n, f(n + 1) - 1$ between $\N \to \N$ and $\N^+ \to \N^+$.



\end{document}
