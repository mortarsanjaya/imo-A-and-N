\documentclass{article}

\usepackage{fullpage}
\usepackage{amsmath, amsfonts, amssymb, amsthm}
\usepackage{hyperref}

\setlength{\parindent}{0pt}
\setlength{\parskip}{5pt}

\newcommand{\Z}{\mathbb{Z}}
\newcommand{\N}{\mathbb{N}}
\newcommand{\Q}{\mathbb{Q}}

\title{IMO 2018 A1}
\author{}
\date{}

\begin{document}

\maketitle



\subsection*{Problem}

Let $G$ be a group and $n$ be an integer with $|n| > 1$.
Suppose that $G$ embeds into the group of functions $\Z^S$ for some set $S$.
Find all functions $f : G \to G$ such that
\[ f(n(x + f(y))) = n f(x) + f(y) \quad \text{for all } x, y \in G. \tag{*}\label{2018a1-eq0} \]



\subsection*{Answer}

The zero function.



\subsection*{Solution}

Official solution: \url{https://www.imo-official.org/problems/IMO2018SL.pdf}

We present the official solution in the setting of additive groups.
However, we obtain $f(0) = 0$ in the beginning, and then deduce $n f(f(x)) = f(x)$ for all $x \in G$.
We then proceed to prove $f = 0$ afterwards.
Before that, note that $G$ must be abelian and torsion-free due to the embedding.

First, plugging $(x, y) = (-f(0), 0)$ into~\eqref{2018a1-eq0} gives us
\[ f(0) = n f(-f(0)) + f(0) \implies f(-f(0)) = 0. \]
Next, pluging $y = -f(0)$ into~\eqref{2018a1-eq0} gives us $f(nx) = n f(x)$ for all $x \in G$.
Plugging $x = -f(0)$ into~\eqref{2018a1-eq0} yields
\[ n f(-f(0) + f(y)) = f(n (-f(0) + f(y))) = f(y) \quad \text{for all } y \in G. \tag{1}\label{2018a1-eq1} \]

We now prove that $f = 0$.
Let $\phi : G \to \Z^S$ be an embedding.
We use subscript notation for applications of $\phi$.
That is, we denote $\phi(x)$ by $\phi_x$ for any $x \in G$.
To show $f = 0$, it suffices to show that
\[ \phi_{f(y)}(s) = 0 \quad \text{for all } s \in S. \]

Fix some $s \in S$.
By~\eqref{2018a1-eq1} and small induction on $k$, we get
\[ n^k \phi_{f^{k + 1}(y)}(s) = \phi_{f(y)}(s) \quad \text{for all } k \in \N. \]
Since $|n| > 1$, this is possible only if $\phi_{f(y)}(s) = 0$, as desired.



\subsection*{Implementation details}

In a separate file, we give the answer for the case where $G$ is the multiplicative group $\Q^+$.
The fact that $\Q^+$ embeds into $\Z^P$ is implemented in the file
\[ \texttt{extra/number\_theory/pos\_rat\_primes.lean}. \]



\end{document}
