\documentclass{article}

\usepackage{amsmath, amsfonts, amssymb, amsthm}
\usepackage{hyperref}

\setlength{\parindent}{0pt}
\setlength{\parskip}{5pt}

\newcommand{\Z}{\mathbb{Z}}
\newcommand{\Q}{\mathbb{Q}}

\title{IMO 2018 A1}
\author{}
\date{}

\begin{document}

\maketitle



\subsection*{Problem}

Let $G$ be a group and $n$ be an integer with $|n| > 1$.
Suppose that $G$ embeds into the group of functions $\Z^S$ for some set $S$.
Find all functions $f : G \to G$ such that, for any $x, y \in G$,
\[ f(n(x + f(y))) = n f(x) + f(y). \tag{*}\label{2018a1-eq0} \]



\subsection*{Answer}

The zero function.



\subsection*{Solution}

Official solution: \url{https://www.imo-official.org/problems/IMO2018SL.pdf}

We present the official solution in the setting of additive groups.
However, we obtain $f(0) = 0$ in the beginning, and then deduce $n f(f(x)) = f(x)$ for all $x \in G$.
We then proceed to prove $f = 0$ afterwards.
Before that, note that $G$ must be abelian and torsion-free due to the embedding.

First, plugging $(x, y) = (-f(0), 0)$ into~\eqref{2018a1-eq0} gives us $f(0) = n f(-f(0)) + f(0)$.
Since $n \neq 0$, we get $f(-f(0)) = 0$.
Next, pluging $y = -f(0)$ into~\eqref{2018a1-eq0} gives us $f(nx) = n f(x)$ for all $x \in G$.
This, in particular, yields $f(0) = 0$ since $n \neq 1$.
Now plugging $x = 0$ into~\eqref{2018a1-eq0} yields $n f(f(y)) = f(n f(y)) = f(y)$ for all $y \in G$.
This suffices to prove that $f = 0$.

Let $\phi : G \to \Z^S$ be an embedding.
We use subscript notation for applications of $\phi$.
That is, we denote $\phi(x)$ by $\phi_x$ for any $x \in G$.
To show that $f = 0$, it suffices to show that $n^k \mid \phi_{f(x)}(s)$ for any $s \in S$, $x \in G$, and $k \geq 0$.
This would imply $f(x) = 0$ since $|n| > 1$.
We fix $s$ and then proceed by induction on $k$.

The base case $k = 0$ is trivial.
Now suppose that, for some $k \geq 0$, we have $n^k \mid \phi_{f(x)}(s)$ for all $x \in G$.
Due to the equation $n f(f(x)) = f(x)$, we have $\phi_{f(x)}(s) = n \phi_{f(f(x))}(s)$.
But $n^k \mid \phi_{f(f(x))}(s)$ by the inductive hypothesis.
This implies $n^{k + 1} \mid \phi_{f(x)}(s)$.
Induction step is complete; we are done.



\subsection*{Implementation details}

We implement a version with multiplicative group as well.
However, we just provide a correspondence without the full solution for the multiplicative case.
Then we state the answer for the case where the multiplicative group is $\Q^+$, which embeds into $\Z^P$, where $P$ is the set of primes.

The fact that $\Q^+$ embeds into $\Z^P$ is implemented in \texttt{extra/number\_theory/pos\_rat\_primes.lean}.



\end{document}
