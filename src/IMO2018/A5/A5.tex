\documentclass{article}

\usepackage{amsmath, amsfonts, amssymb, amsthm}
\usepackage{hyperref}

\setlength{\parindent}{0pt}
\setlength{\parskip}{5pt}

\title{IMO 2018 A5}
\author{}
\date{}

\begin{document}

\maketitle



\subsection*{Problem}

Let $F$ be a totally ordered field.
Denote by $F^+$ the set of positive elements of $F$.
Let $V$ be an $F$-vector space.
Find all functions $f : F^+ \to V$ such that, for all $x, y \in F^+$,
\[ (x + x^{-1}) f(y) = f(xy) + f(x^{-1} y). \tag{*}\label{2018a5-eq0} \]



\subsection*{Answer}

$x \mapsto x v_1 + x^{-1} v_2$ for some $v_1, v_2 \in V$.



\subsection*{Solution}

Reference: \url{https://artofproblemsolving.com/community/c6h1876752p12753740}

We refer to the solution in the AoPS thread of IMO 2018 A5 by the user \texttt{MarkBcc168} (post \#6).
We will also present the same solution below.
Note that the solution generalizes well to our setting.

Clearly, the set of functions satisfying~\eqref{2018a5-eq0} form an $F$-vector space.
Furthermore, it is easy to check that $x \mapsto xv$ and $x \mapsto x^{-1} v$ satisfies~\eqref{2018a5-eq0} for any $v \in V$.
This verifies the answer, so it remains to prove that no other solutions exist.
Consider the function $g(x) = \frac{2x - 2x^{-1}}{3} f(2) + \frac{4x^{-1} - x}{3} f(1)$.
One checks that $g$ satisfies~\eqref{2018a5-eq0}, $g(1) = f(1)$, and $g(2) = f(2)$.
Subtracting with $g$, it suffices to show that if $f$ satisfies~\eqref{2018a5-eq0} with $f(1) = f(2) = 0$, then $f \equiv 0$.

Now suppose that $f$ satisfies~\eqref{2018a5-eq0} and $f(1) = f(2) = 0$.
Fix an arbitrary $t \in F^+$.
Plugging $(x, y) = (t, 1)$ into~\eqref{2018a5-eq0} gives us $g(t) + g(t^{-1}) = 0$ for all $t \in F^+$.
Plugging $(x, y) = (2t, 2)$ into~\eqref{2018a5-eq0} gives us $g(4t) + g(t^{-1}) = 0$ for all $t \in F^+$.
Thus, we get $g(4t) = g(t)$ for all $t \in F^+$.
Now plugging $(x, y) = (4, t)$ into~\eqref{2018a5-eq0} gives us $\frac{17}{4} g(y) = 2 g(t)$, so $g(t) = 0$, as desired.




\subsection*{Implementation details}

To implement the solution, it is most convenient to construct the set of solutions as an $F$-vector space.
We implement this construction, and we also implement solution checks separately.
The harder direction also uses the construction and solution checks.



\end{document}
