\documentclass{article}

\usepackage{amsmath, amsfonts, amssymb, amsthm}
\usepackage{hyperref}

\setlength{\parindent}{0pt}
\setlength{\parskip}{5pt}

\title{IMO 2011 A6 (P3)}
\author{}
\date{}

\begin{document}

\maketitle



\subsection*{Problem}

Let $R$ be a totally ordered commutative ring.
Let $f : R \to R$ be a function such that, for all $x, y \in R$,
\[ f(x + y) \leq y f(x) + f(f(x)). \tag{*}\label{2011a6-eq0} \]
Prove that $f(x) = 0$ for all $x \in R$ with $x \leq 0$.



\subsection*{Solution}

Official solution: \url{https://www.imo-official.org/problems/IMO2011SL.pdf}

Our solution below will use Solution 1 of the official solution, except proving $f(t) \leq 0$ for all $t \in R$.
We present a simpler proof for the fact.

First, we prove that $x f(x) \leq 0$ for all $x \in R$.
Indeed, first replace $y$ with $f(t) - x$ in~\eqref{2011a6-eq0} to get that for any $t, x \in R$,
\[ f(f(t)) \leq (f(t) - x) f(x) + f(f(x)) \implies f(f(t)) + x f(x) \leq f(f(x)) + f(t) f(x). \]
Switching $t$ and $x$ and then adding with the above inequality, we get
\[ x f(x) + t f(t) \leq 2 f(t) f(x). \]
In particular, setting $t = 2 f(x)$ gives us $xf(x) \leq 0$ for all $x \in R$.
More importantly, we get $f(x) \geq 0$ for $x < 0$.

Next, we prove that $f(x) \leq 0$ for all $x \in R$.
Indeed, plugging $y = 0$ into~\eqref{2011a6-eq0} gives $f(x) \leq f(f(x))$ for all $x \in R$.
If $f(x) > 0$, then $f(f(x)) > 0$, but the previous paragraph gives us $f(f(x)) \leq 0$; a contradiction.
Thus, $f(x) \leq 0$ for any $x \in R$.
In particular, we get $f(x) = 0$ for $x < 0$.

It remains to show that $f(0) = 0$.
The previous paragraph gives us $f(0) \leq 0$, so we just have to show that $f(0) \geq 0$.
Indeed, plugging $(x, y) = (-1, 0)$ into~\eqref{2011a6-eq0} gives us $0 = f(-1) \leq f(f(-1)) = f(0)$.



\subsection*{Implementation details}

We shorten the very first step of showing $x f(x) \leq 0$ for all $x \in R$ by using \texttt{linarith}.
We also show $f(x) \leq f(f(x))$ for all $x \in R$ and $f(x) = 0$ for all $x < 0$ as extra hypotheses.



\end{document}
