\documentclass{article}

\usepackage{amsmath, amsfonts, amssymb, amsthm}
\usepackage{hyperref}

\setlength{\parindent}{0pt}
\setlength{\parskip}{5pt}

\newcommand{\Z}{\mathbb{Z}}
\newcommand{\Q}{\mathbb{Q}}

\newtheorem*{lemma*}{Lemma}

\title{IMO 2011 N7}
\author{}
\date{}

\begin{document}

\maketitle



\subsection*{Problem}

Let $p$ be an odd prime.
For each $a \in \Q_p$, define
\[ S_a = \sum_{i = 1}^{p - 1} \frac{a^i}{i} \in \Q_p. \]
Prove that $\|S_3 + S_4 - 3 S_2\|_p < 1$.



\subsection*{Solution}

Official solution: \url{https://www.imo-official.org/problems/IMO2011SL.pdf}

We present Solution 1 of the official solution in the language of $p$-adics.
That is, we start with the following lemma expressing $S_a$ for each $a \in \Z_p$ up to modulo $p$ precision.
The result we need to prove is almost immediate afterwards.
Note that $S_a \in \Z_p$ for $a \in \Z_p$, and $\|S_3 + S_4 - 3 S_2\|_p < 1$ is equivalent to $p \mid S_3 + S_4 - 3 S_2$ over $\Z_p$.

\begin{lemma*}
For any $a \in \Z_p$, $p S_a \equiv (a - 1)^p + 1 - a^p \pmod{p^2}$.
\end{lemma*}
\begin{proof}
The main observation is that, for any $0 < k < p$,
\[ \binom{p}{k} = \frac{p(p - 1) \ldots (p - (k - 1))}{k!} \equiv p \frac{(-1)^{k - 1}}{k} \pmod{p^2}. \]
Rewrite as $p/k \equiv (-1)^{k - 1} \binom{p}{k} \pmod{p^2}$.
Thus, by binomial expansion,
\[ p S_a \equiv \sum_{i = 1}^{p - 1} \binom{p}{i} (-1)^{i - 1} a^i \equiv (a - 1)^p + 1 - a^p \pmod{p^2}. \]
\end{proof}

Due to the lemma, we have
\begin{align*}
    p (S_3 + S_4 - 3 S_2)
    &\equiv (2^p + 1 - 3^p) + (3^p + 1 - 4^p) - 3 (2 - 2^p) \pmod{p^2} \\
    &\equiv 4 \cdot 2^p - 4 - 4^p \pmod{p^2} \\
    &\equiv -(2^p - 2)^2 \pmod{p^2}.
\end{align*}
But $p \mid 2^p - 2$ by Fermat's little theorem.
We get $p^2 \mid p (S_3 + S_4 - 3 S_2)$, and thus $p \mid S_3 + S_4 - 3 S_2$, as desired.



\subsection*{Implementation details}

It seems to be more convenient to define the sum over $\Q_p$.
This is the reason for the choice of problem statement.
We will prove all results in terms of $\Q_p$ as opposed to $\Z_p$.

At the time of writing, it seems that the \texttt{mathlib} API does not support working with general ultrametric spaces yet.



\end{document}
