\documentclass{article}

\usepackage{fullpage}
\usepackage{amsmath, amsfonts, amssymb, amsthm}
\usepackage{hyperref}

\setlength{\parindent}{0pt}
\setlength{\parskip}{5pt}

\newcommand{\N}{\mathbb{N}}

\newtheorem{lemma}{Lemma}

\title{IMO 2011 N5 (P5)}
\author{}
\date{}

\begin{document}

\maketitle



\subsection*{Problem}

Let $G$ be an additive (not necessarily abelian) group.
A function $f : G \to \N^+$ is said to be \emph{good} if
\[ f(x - y) \mid f(x) - f(y) \quad \text{for any } x, y \in G. \tag{*}\label{2011n5-eq0} \]
Prove that for any $f : G \to \N^+$ good and $x, y \in G$, if $f(x) \leq f(y)$, then $f(x) \mid f(y)$.

Extra: Find all good functions.



\subsection*{Answer for the extra part}

Let $G = G_0 > G_1 > \ldots > G_k$ be an arbitrary strict chain of subgroups, $k \geq 0$.
Let $n_0 < n_1 < \ldots < n_k$ be a sequence of positive integers such that $n_i \mid n_{i + 1}$ for each $0 \leq i < k$.
Then the function $f = x \mapsto n_{i(x)}$, where $i(x) = \max\{i : x \in G_i\}$, is good.
Futhermore, all good functions can be described as above.



\subsection*{Solution}

Official solution: \url{https://www.imo-official.org/problems/IMO2011SL.pdf}

The official solution almost works in the general (possibly non-abelian) group settings.
One extra result that we need to be able to modify the official solution is that $f$ is even.
Setting $y = 0$ in~\eqref{2011n5-eq0} yields $f(x) \mid f(0)$ for any $x \in G$.
Then setting $x = 0$ instead yields $f(-y) \mid f(0) - f(y)$ and thus $f(-y) \mid f(y)$ for any $y \in G$.
Since $f(-y)$ and $f(y)$ divides each other and are positive integers, we get $f(-y) = f(y)$ for all $y \in G$.

Now, we prove that $f(x) \leq f(y)$ implies $f(x) \mid f(y)$.
Clearly, we can assume that $f(x) < f(y)$.
By~\eqref{2011n5-eq0}, we have $f(y - x) \mid f(y) - f(x)$.
Thus, it suffices to prove that $f(y - x) = f(x)$.

First note that $f(x) < f(y)$ and $f(y - x) \mid f(y) - f(x)$ implies $f(y) - f(x) \geq f(y - x)$.
That is, $f(y) \geq f(y - x) + f(x) > |f(y - x) - f(x)|$.
It remains to show that $f(y) \mid f(y - x) - f(x)$.

Now~\eqref{2011n5-eq0} yields $f(y - x - (-x)) \mid f(y - x) - f(-x)$.
Note that, even for $G$ non-abelian, we have $y - x - (-x) = y$.
The fact that $f$ is even yields $f(y) \mid f(y - x) - f(x)$, as desired.


\subsubsection*{Extra}

While the extra problem is a generalization of the original problem, the original problem is still useful as a step.

We now find all good functions $f : G \to \N^+$.
Recall that $f(x) \mid f(0)$ for any $x \in G$ and that $f$ is even.

Now, for each integer $n$ dividing $f(0)$, let $H_n = \{x \in G : n \mid f(x)\}$.
By assumption, we get $0 \in H_n$.
Since $f$ is even, it is easy to check that $H_n$ is closed under negation.
Finally,~\eqref{2011n5-eq0} yields $f(x) \mid f(x + y) - f(y)$ for any $x, y \in G$.
That means if $x, y \in G$, then $n \mid f(x + y)$ as well, implying $x + y \in G$.
Thus $H_n$ is in fact a subgroup of $G$.

Given the above informations, we now describe $f$ as in the answer section.
Since any element of $f(G)$ divides $f(0)$, $f(G)$ is in fact finite.
Enumerate its elements as $n_0 < n_1 < n_2 < \ldots < n_k$ in increasing order, where $k \geq 0$.
The original problem yields $n_i \mid n_j$ for each $i \leq j$.
Now we can check by hand that $f$ matches the description in the answer section, with $G_i = H_{n_i}$ for each $i \leq k$.

We now prove the converse.
Fix $x, y \in G$, and for convenience, write $H = G_{i(x - y)}$.
Then $x - y \in H$, and since $H$ is a subgroup, we have $x \in H \iff y \in H$.
If $x, y \in H$, then $f(x - y)$ divides both $f(x)$ and $f(y)$, so we are done.
If $x, y \notin H$, then $i(x), i(y) < i(x - y)$, so $x - y \in G_{i(x)}$ and $x - y \in G_{i(y)}$.
Since $x - y, x \in G_{i(x)}$, we have $y \in G_{i(x)}$, so $i(x) \leq i(y)$.
Similarly, $i(y) \leq i(x)$, so $i(x) = i(y)$ and thus $f(x) = f(y)$.



\subsection*{Extra notes}

The extra part has only been partially implemented.



\end{document}
