\documentclass{article}

\usepackage{amsmath, amsfonts, amssymb, amsthm}
\usepackage{hyperref}

\setlength{\parindent}{0pt}
\setlength{\parskip}{5pt}

\newcommand{\N}{\mathbb{N}}

\newtheorem{lemma}{Lemma}

\title{IMO 2011 N5 (P5)}
\author{}
\date{}

\begin{document}

\maketitle



\subsection*{Problem}

Let $G$ be an abelian group.
We say that $f : G \to \N^+$ is \emph{good} if $f(x - y) \mid f(x) - f(y)$ for any $x, y \in G$.
Find all good functions.



\subsection*{Answer}

Let $G = G_0 > G_1 > \ldots > G_k$ be an arbitrary strict chain of subgroups, $k \geq 0$.
Let $n_0 < n_1 < \ldots < n_k$ be a sequence of positive integers such that $n_i \mid n_{i + 1}$ for each $0 \leq i < k$.
Then the function $f = x \mapsto n_{i(x)}$, where $i(x) = \max\{i : x \in G_i\}$, is good.
Futhermore, all good functions can be described as above.



\subsection*{Solution}

Official solution: \url{https://www.imo-official.org/problems/IMO2011SL.pdf}

While our problem is a generalization of the original problem, the original problem is still useful here.
We state it as a lemma, and use it to classify all good functions.
To solve this problem, we proceed with lemmas.

\begin{lemma}\label{2011n5-1}
Let $f : G \to \N^+$ be a good function and $x, y \in G$.
If $f(x) \leq f(y)$, then $f(x) \mid f(y)$.
\end{lemma}
\begin{proof}
The divisibility is clear if $f(x) = f(y)$, so now suppose that $f(x) < f(y)$.
Since $f$ is good, $f(x - y) \mid f(x) - f(y)$, so $f(x - y) \mid f(y) - f(x)$.
Since $f(y) > f(x)$, this implies $f(y) \geq f(x) + f(x - y) > f(x - y)$.
Now, $f(y) \mid f(x) - f(x - y)$, but both $f(x)$ and $f(x - y)$ are between $0$ and $f(y)$.
This necessarily implies $f(x) = f(x - y)$.
Finally, $f(x) = f(x - y) \mid f(x) - f(y)$ implies $f(x) \mid f(y)$.
\end{proof}

\begin{lemma}\label{2011n5-2}
Let $f : G \to \N^+$ be a good function.
Then $f(x) \mid f(0)$ for all $x \in G$.
\end{lemma}
\begin{proof}
Plugging $y = 0$ gives us $f(x) \mid f(x) - f(0)$ and thus $f(x) \mid f(0)$.
\end{proof}

\begin{lemma}\label{2011n5-3}
Let $f : G \to \N^+$ be a good function.
For any $n \in \Z$ such that $n \mid f(0)$, the set $H_n = \{x \in G | n \mid f(x)\}$ is a subgroup of $G$.
\end{lemma}
\begin{proof}
First, Lemma~\ref{2011n5-2} implies that $0 \in H_n$.
Second, for any $x \in G$, we have $f(x) \mid f(0) - f(-x)$, so $f(x) \mid f(-x)$ since $f(x) \mid f(0)$.
This implies that $H_n$ is closed under additive inverse.
Finally, for any $x, y \in H_n$, we have $n \mid f(x) \mid f(x + y) - f(y) \implies n \mid f(x + y)$.
So, $x + y \in H_n$, which proves that $H_n$ is closed under addition.
\end{proof}

Now, Lemma~\ref{2011n5-2} shows that $f(G)$ is finite.
We can enumerate the elements of $f(G)$ as $n_0 < n_1 < \ldots n_k$ for some $k \geq 0$.
Lemma~\ref{2011n5-1} shows a stronger property: we have $n_i \mid n_{i + 1}$ for all $i < k$.
Now let $G_i = H_{n_i}$ for each $i \leq k$.
It is easy to see that $G_0 = G$, and Lemma~\ref{2011n5-3} tells us that $G_{i + 1} \leq G_i$ for each $i < k$.
It remains to show that $f$ matches the description in the answer section.

Fix an arbitrary $x \in G$, and let $i(x) = \max\{i : x \in G_i\}$.
Then $x \in G_{i(x)} = H_{n_{i(x)}}$, so $n_{i(x)} \mid f(x)$.
But also, we have $n_i \nmid f(x)$ for all $i > i(x)$.
Since $f(G) = \{n_0 < n_1 < \ldots < n_k\}$, this forces $f(x) = n_{i(x)}$, as desired.

We now prove the converse.
Fix $x, y \in G$, and for convenience, write $H = G_{i(x - y)}$.
Then $x - y \in H$, and since $H$ is a subgroup, we have $x \in H \iff y \in H$.
If $x, y \in H$, then $f(x - y)$ divides both $f(x)$ and $f(y)$, so we are done.
If $x, y \notin H$, then $i(x), i(y) < i(x - y)$, so $x - y \in G_{i(x)}$ and $x - y \in G_{i(y)}$.
Since $x - y, x \in G_{i(x)}$, we have $y \in G_{i(x)}$, so $i(x) \leq i(y)$.
Similarly, $i(y) \leq i(x)$, so $i(x) = i(y)$ and thus $f(x) = f(y)$.



\end{document}
