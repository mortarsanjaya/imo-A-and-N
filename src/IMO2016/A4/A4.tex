\documentclass{article}

\usepackage{amsmath, amsfonts, amssymb, amsthm}
\usepackage{hyperref}

\setlength{\parindent}{0pt}
\setlength{\parskip}{5pt}

\title{IMO 2016 A4}
\author{}
\date{}

\begin{document}

\maketitle



\subsection*{Problem}

Let $R$ be a totally ordered commutative ring.
Denote by $R^+$ the set of positive elements of $R$.
Find all functions $f : R^+ \to R^+$ such that, for all $x, y \in R^+$,
\[ x f(x^2) f(f(y)) + f(y f(x)) = f(xy) (f(f(x^2)) + f(f(y^2))). \tag{*}\label{2016a4-eq0} \]



\subsection*{Answer}

A function $f : R^+ \to R^+$ satisfies~\eqref{2016a4-eq0} if and only if $x f(x) = 1$ for all $x \in R^+$.
In particular, such a function does not exist if $R$ is not a field, and $x \mapsto x^{-1}$ is the only such function if $R$ is a field.



\subsection*{Solution}

Official solution: \url{http://www.imo-official.org/problems/IMO2016SL.pdf}

We present Solution 1 of the official solution.
We prove injectivity in a different way from the solution.

First, we check that $f$ satisfies~\eqref{2016a4-eq0} if $x f(x) = 1$ for all $x \in R^+$.
Start by verifying that $f(f(x)) = x$ for all $x \in R^+$; this holds due to $x f(x) = f(x) f(f(x)) = 1$ for all $x \in R^+$.
The equation to be proved reduces to $xy f(x^2) + f(y f(x)) = f(xy) (x^2 + y^2)$.
Now, multiply both sides by $xy f(x) = y$ and get
\[ (xy)^2 f(x^2) f(x) + x = f(x) (x^2 + y^2) \iff y^2 f(x) + x = x + f(x) y^2. \]

Now, we go for the converse.
Start by plugging $y = 1$ in~\eqref{2016a4-eq0}.
We get that, for all $x \in R^+$,
\[ x f(x^2) f(f(1)) + f(f(x)) = f(x) (f(f(x^2)) + f(f(1))). \]
Plugging $x = 1$ into this equation gives us $f(1) = 1$.
So, the equation simplifies to
\[ x f(x^2) + f(f(x)) = f(x) (f(f(x^2)) + 1). \]
Plugging $x = 1$ into~\eqref{2016a4-eq0} gives us
\[ f(f(y)) + f(y) = f(y) (f(f(y^2)) + 1) \implies f(y) f(f(y^2)) = f(f(y)). \tag{1}\label{2016a4-eq1} \]
Then the previous equation becomes
\[ x f(x^2) + f(f(x)) = f(f(x)) + f(x) \implies x f(x^2) = f(x). \tag{2}\label{2016a4-eq2} \]

Next, we show that $f$ is injective.
Fix some $a, b \in R^+$, and suppose that $f(a) = f(b)$.
Write $c = f(a) = f(b)$.
For any $x, y \in R^+$ such that $f(x) = f(y) = c$,~\eqref{2016a4-eq0},~\eqref{2016a4-eq1}, and~\eqref{2016a4-eq2} gives us
\[ c(c f(c) + f(yc)) = c f(xy) (f(f(x^2)) + f(f(y^2))) = 2 f(c) f(xy). \]
This implies $f(y^2) = f(xy)$, as
\[ 2 f(c) f(xy) = 2 f(c) f(y^2) = c(c f(c) + f(yc)). \]
In particular, we get $f(a^2) = f(ab) = f(b^2)$.
Then~\eqref{2016a4-eq2} implies $a = b$, as desired.

Finally, we show that $x f(x) = 1$ for all $x \in R^+$.
Indeed,~\eqref{2016a4-eq1} and~\eqref{2016a4-eq2} gives us $f(x) f(f(x^2)) = f(f(x)) = f(x) f(f(x)^2)$, so $f(f(x^2)) = f(f(x)^2)$.
But $f$ is injective, so $f(x^2) = f(x)^2$.
Thus,~\eqref{2016a4-eq2} gives $x f(x)^2 = f(x)$ and thus $x f(x) = 1$.



\subsection*{Implementation details}

The above solution only needs $R^+$ to be an integral (multiplicative) monoid with a commutative (additive) semigroup structure satisfying some extra properties.
First, the semigroup is cancellative in the sense that $a + b = a + c$ implies $b = c$.
Second, the addition on $R^+$ distributes with respect to multiplication.
We define this more general class in
\[ \texttt{extra.integral\_domain\_without\_zero.lean}. \]



\end{document}
