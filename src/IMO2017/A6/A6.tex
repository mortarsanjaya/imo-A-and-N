\documentclass{article}

\usepackage{amsmath, amsfonts, amssymb, amsthm}
\usepackage{hyperref}

\setlength{\parindent}{0pt}
\setlength{\parskip}{5pt}

\DeclareMathOperator{\rchar}{char}

\newtheorem*{claim}{Claim}

\title{IMO 2017 A6 (P2)}
\author{}
\date{}

\begin{document}

\maketitle



\subsection*{Problem}

Fix a ring $R$.
In the following two cases, determine all functions $f : R \to R$ such that, for any $x, y \in R$,
\[ f(f(x) f(y)) + f(x + y) = f(xy). \tag{*}\label{2017a6-eq0} \]
\begin{enumerate}
    \item   $R$ is a division ring with $\rchar(R) \neq 2$,
    \item   $R$ is a field with $\rchar(R) = 2$.
\end{enumerate}


\subsection*{Answer}

$x \mapsto 0$, $x \mapsto x - 1$, and $x \mapsto 1 - x$.

In the case where $\rchar(R) = 2$, this reduces to just $x \mapsto 0$ and $x \mapsto x + 1$.



\subsection*{Solution}

Reference: \url{https://artofproblemsolving.com/community/c6h1480146p8693244}

The reference links to the solution in the AoPS thread for IMO 2017 P2 by the user \textbf{anantmudgal09} (see post \#75).
The solution works for fields of characteristic not equal to $2$.
With a careful modification, the solution works for any division rings $R$ with $\rchar(R) \neq 2$.
We write the same solution here with the necessary modification and a bit of rewording.
We provide our own solution for the case where $R$ is a field with $\rchar(R) = 2$.
First, note that it is easy to check that all the provided functions satisfy~\eqref{2017a6-eq0}.

We start with some small properties.
First, for any $t \in R$, plugging $(x, y) = (t + 1, t^{-1} + 1)$ into~\eqref{2017a6-eq0} yields
\[ f(f(t + 1) f(t^{-1} + 1)) = 0. \tag{1}\label{2017a6-eq1} \]
The above equation is useful in the following claim, but it is also useful for the solution in the case $\rchar(R) = 2$.

\begin{claim}
Let $f$ be a function satisfying~\eqref{2017a6-eq0}.
Then $-f$ also satisfies~\eqref{2017a6-eq0}.
If $f \neq 0$, we also have $f(0)^2 = 1$, $f(1) = 0$, and for any $x \in R$, $f(x) = 0 \iff x = 1$.
\end{claim}
\begin{proof}
The fact that $-f$ satisfies~\eqref{2017a6-eq0} is immediate.
Plugging $(x, y) = (0, 0)$ into~\eqref{2017a6-eq0} yields $f(f(0)^2) = 0$.
Thus, to prove the claim, it suffices to prove that, if $f \neq 0$, then $f(x) = 0$ implies $x = 1$ for any $x \in R$.

Write $x = t + 1$, and suppose that $f(t + 1) = 0$ but $t \neq 0$.
Then~\eqref{2017a6-eq1} tells us that $f(0) = 0$.
Finally, plugging $y = 0$ into~\eqref{2017a6-eq0} yields $f(x) = 0$ for all $x \in R$, i.e., $f \equiv 0$.
\end{proof}

In view of the claim, it suffices to prove that, if $f$ satisfies~\eqref{2017a6-eq0} and $f(0) = 1$, then $f(x) = 1 - x$ for all $x \in R$.
Thus, from now on, we assume that $f(0) = 1$.

First, we reduce even more: it suffices to prove that $f$ is injective.
Indeed, plugging $y = 0$ into~\eqref{2017a6-eq0} yields $f(f(x)) + f(x) = 1$ for all $x \in R$.
This means $f(f(f(x))) = f(x) = 1 - f(f(x))$, so injectivity yields $f(f(x)) = x$ and thus $f(x) = 1 - x$ for all $x \in R$.
It now remains to prove that $f$ is injective.

One important equation throughout the rest of the proof is as follows.
Plugging $y = 1$ into~\eqref{2017a6-eq0} yields that, for any $x \in R$,
\[ f(x + 1) + 1 = f(x). \tag{2}\label{2017a6-eq2} \]


\subsubsection*{Case 1: $R$ is a division ring and $\rchar(R) \neq 2$.}

First,~\eqref{2017a6-eq2} implies that $f(-1) = 2$.
Thus, plugging $x = -1$ into~\eqref{2017a6-eq0} yields that for all $y \in R$,
\[ f(2f(y)) + f(y) + 1 = f(-y). \]
In particular, for any $a, b \in R$, $f(a) = f(b)$ implies $f(-a) = f(-b)$.
Furthermore, this means that $f(y) = f(-y)$ implies
\[ f(2f(y)) = -1 \iff 2f(y) = 2 \iff f(y) = 1 \iff y = 0. \]

Thus, it suffices to show that, for any $a, b \in R$, $f(a) = f(b)$ implies $f(a - b) = f(b - a)$.
First, comparing~\eqref{2017a6-eq0} with  $(x, y) = (a, b)$ and $(x, y) = (b, a)$ gives us $f(ab) = f(ba)$.
In the virtue of the previous paragraph, we get $f(-a) = f(-b)$, and in addition, $f(-ab) = f(-ba)$.
Now, comparing~\eqref{2017a6-eq0} with $(x, y) = (a, -b)$ and $(x, y) = (b, -a)$ gives us the desired equality.


\subsubsection*{Case 2: $R$ is a field and $\rchar(R) = 2$.}

By~\eqref{2017a6-eq2}, it suffices to prove that $f(a + 1) = f(b + 1)$ implies $a = b$.
The claim yields $f(x + 1) = 0 \iff x = 0$, so either $a = b = 0$ or $a, b \neq 0$.
We now assume that the latter case holds.
The main step to get injectivity is proving $f(x + y) = f(xy) + 1$, where $x = a + b + 1$ and $y = a^{-1} + b^{-1} + 1$.

First, we see why the equality implies $a = b$.
Indeed, by~\eqref{2017a6-eq0}, the equality yields
\[ f(f(x) f(y)) = 1 \iff f(x) f(y) = 0 \iff x = 1 \lor y = 1. \]
Since $x = a + b + 1$, $y = a^{-1} + b^{-1} + 1$, and $\rchar(R) = 2$, both $x = 1$ and $y = 1$ are equivalent to $a = b$.
It now remains to show that indeed we have $f(x + y) = f(xy) + 1$.

First, for any $a, b \in R$ with $f(a + 1) = f(b + 1)$, plugging $(x, y) = (a + 1, b^{-1} + 1)$ into~\eqref{2017a6-eq0} and using the claim yields
\[ f(ab^{-1} + a + b^{-1} + 1) = f(f(a + 1) f(b^{-1} + 1)) + f(a + b^{-1}) = f(a + b^{-1}). \]
In particular, we have $f(ab^{-1} + a + b^{-1}) = f(1 + a + b^{-1})$.

Now, we have the similar equation with $a^{-1}$ and $b$ replacing $a$ and $b^{-1}$.
That is, we also have $f(ba^{-1} + b + a^{-1}) = f(1 + b + a^{-1})$.
Now let $t = 1 + a + b^{-1}$, $u = 1 + b + a^{-1}$, $v = ab^{-1} + a + b^{-1}$ and $w = ba^{-1} + b + a^{-1}$.
Recall that $f(t) = f(v)$ and $f(u) = f(w)$, but also, we have $tu = vw$.
Thus, by comparing~\eqref{2017a6-eq0} with $(x, y) = (t, u)$ and $(x, y) = (v, w)$ gives us $f(t + u) = f(v + w)$.
Finally, notice that $t + u = x + y$ and $v + w = xy + 1$, where $x = a + b + 1$ and $y = a^{-1} + b^{-1} + 1$.
By the claim, we get $f(x + y) = f(xy + 1) = f(xy) + 1$, as desired.



\subsection*{Implementation details}

Clearly, we break the claim into several small lemmas.
We also implement~\eqref{2017a6-eq1} and~\eqref{2017a6-eq2} as a separate lemma.
In the above step, we show that if $f(0) = 1$ and $f$ is injective, then $f(x) = 1 - x$ for all $x \in R$.
This holds in any ring $R$, so we implement this result as another lemma as well.
Another lemma we implement is checking that $x \mapsto 1 - x$ indeed satisfies~\eqref{2017a6-eq0}.
As we deal with two very distinct cases, we implement the two cases as separate results.
Lastly, we implement a wrapper for cases where $R$ is a field.



\end{document}
