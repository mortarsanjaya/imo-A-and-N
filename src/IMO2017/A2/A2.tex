\documentclass{article}

\usepackage{amsmath, amsfonts, amssymb, amsthm}
\usepackage{hyperref}

\setlength{\parindent}{0pt}
\setlength{\parskip}{5pt}

\newcommand{\Z}{\mathbb{Z}}

\DeclareMathOperator{\rchar}{char}

\newtheorem{lemma}{Lemma}


\title{IMO 2017 A2}
\author{}
\date{}

\begin{document}

\maketitle



\subsection*{Introduction}

Let $R$ be a ring.
Given $q \in R$, we say that a finite set $T \subseteq R$ is \emph{$q$-good} if for any $u, v \in T$, there exists $a, b, c, d \in T$ such that $quv = a^2 + b^2 - c^2 - d^2$.
Given $k \geq 0$, we say that $q \in R$ is \emph{$k$-excellent} if for any finite set $T \subseteq R$ of size $k$, the set $T - T := \{s - t : s, t \in T\}$ is $q$-good.
We are interested in finding all $k$-excellent elements of $R$ with some restrictions on $R$ and $k$.
The main results are as follows.

\begin{itemize}
    \item   If $k = 0$ or $k = 1$, any element of $R$ is $k$-excellent.
    \item   If $k = 2$, then the set of $k$-excellent elements is $\{0, \pm 1, \pm 2\}$.
    \item   If $k \geq 3$ and $\rchar(R) = 0$, then the set of $k$-excellent elements is either $\{0\}$ or $\{0, \pm 2\}$.
            If $R$ is commutative, then this set is exactly $\{0, \pm 2\}$.
            Conversely, if $R$ is a domain and the set is $\{0, \pm 2\}$, then $R$ has to be commutative.
\end{itemize}



\subsection*{Solution}

Official solution: \url{http://www.imo-official.org/problems/IMO2017SL.pdf}

The cases $k = 0$ and $k = 1$ are trivial.
The case $k = 2$ involves a simple bash compared to the case $k \geq 3$.
For the case $k \geq 3$ with $\rchar(R) = 0$, we follow Solution 1 of the official solution.
In particular, we will actually reduce the problem to the case where $R = \Z$.

It is clear that if $q \in R$ is $k$-excellent, then $-q$ is also $k$-excellent.
It is also clear that $0$ is always $k$-excellent; any set is $0$-good.
We can now solve for the case $k = 2$.

\begin{lemma}\label{2017a2-1}
Over any (nontrivial) ring $R$, $q \in R$ is $2$-excellent if and only if $q \in \{0, \pm 1, \pm 2\}$.
\end{lemma}
\begin{proof}
For the backwards direction, it suffices to prove that $1, 2 \in R$ are $2$-excellent.
For any $T \subseteq R$ of size $2$, there exists $x \in R$ such that $T - T = \{0, \pm x\}$.
So, it remains to prove that $S = \{0, \pm x\}$ is $1$-good and $2$-good for any $x \in R$.
Indeed, the set of elements of form $a^2 + b^2 - c^2 - d^2$ with $a, b, c, d \in S$ is $\{0, \pm x^2, \pm 2x^2\}$.
On the other hand, for any $u, v \in \{0, \pm x\}$, we have $uv \in \{0, \pm x^2\}$.
It is now immediate that $S$ is $1$-good and $2$-good.

Conversely, suppose that $q$ is $2$-excellent.
Consider $T = \{0, 1\}$, with $T - T = \{0, \pm 1\}$.
Since we assume that $R$ is nontrivial, $0 \neq 1$ in $R$, so $|\{0, 1\}| = 2$.
Since $q$ is $2$-excellent, this means that $\{0, \pm 1\}$ is $q$-good.
In particular, $q = a^2 + b^2 - c^2 - d^2$ for some $a, b, c, d \in \{0, \pm 1\}$.
The rest is standard bashing.
\end{proof}

Now we solve the case $k \geq 3$, under the assumption that $\rchar(R) = 0$.
If $R$ is commutative, then the following identity implies that $2 \in R$ is $k$-excellent:
\[ 2(w - x)(y - z) = (w - z)^2 + (x - y)^2 - (w - y)^2 - (x - z)^2. \]
We show later on that the converse necessarily holds if $R$ has no zero divisors.
But now, we first show that $k$-excellent elements are either $0$ or $\pm 2$.
This means that if $R$ is commutative, then the set of $k$-excellent elements is precisely $\{0, \pm 2\}$.

\begin{lemma}\label{2017a2-2}
Let $R$ be a ring of characteristic zero, and let $k \geq 3$.
Then a $k$-excellent element of $R$ must be an integer.

More formally, let $\iota : \Z \to R$ be the unique, injective ring homomorphism.
Then a $k$-excellent element of $R$ must be of form $\iota(n)$, where $n$ is a $k$-excellent integer.
\end{lemma}
\begin{proof}
Fix some $k$-excellent $q \in R$.
Take a finite set $S$ of size $k$ consisting only of integers, such that $1 \in S - S$.
For example, $S = \{0, 1, 2, \ldots, k - 1\}$ works.
All we need is that $S - S$ consists only of integers and it contains $1$.

Since $\iota$ is injective, $\iota(S)$ also has size $k$.
By assumption, $\iota(S) - \iota(S)$ is $q$-good.
Since $1 \in S - S$, we get $q = a^2 + b^2 - c^2 - d^2$ for some $a, b, c, d \in \iota(S) - \iota(S)$.
That is, $q = \iota(a^2 + b^2 - c^2 - d^2)$ for some $a, b, c, d \in S - S$.
\end{proof}

\begin{lemma}\label{2017a2-3}
Let $R$ be a ring of characteristic zero, and let $k \geq 3$.
Let $n$ be an integer, and suppose that the image of $n$ in $R$ is $k$-excellent.
Then $n$ is $k$-excellent over $\Z$.
\end{lemma}
\begin{proof}
Again, let $\iota : \Z \to R$ be the natural embedding.
Let $T \subseteq \Z$ be a finite set of size $k$.
Then $\iota(T) \subseteq R$ is finite of size $k$ as well.
So, by assumption, $\iota(T - T) = \iota(T) - \iota(T)$ is $\iota(n)$-good.
 
Now we prove that $T - T$ is $n$-good.
For any $u, v \in T - T$, by the previous paragraph, we can write $\iota(nuv) = a^2 + b^2 - c^2 - d^2$ for some $a, b, c, d \in \iota(T - T)$.
This means we can also write $\iota(nuv) = \iota(a^2 + b^2 - c^2 - d^2)$ for some $a, b, c, d \in T - T$.
Since $\iota$ is injective, this implies $nuv = a^2 + b^2 - c^2 - d^2$.
This proves that $T - T$ is $n$-good.
\end{proof}

We now work exclusively in $\Z$ to show that $k$-excellent elements over $R$ are either $0$ or $\pm 2$.
For a further step when $R$ is a non-commutative domain, we need to work over $R$ again.

\begin{lemma}\label{2017a2-4}
Over $\Z$, Any $k$-excellent integer has absolute value at most $2$.
\end{lemma}
\begin{proof}
A more general statement is as follows: for any $q \in \Z$, if $|q| > 2$, then the only $q$-good sets are $\emptyset$ and $\{0\}$.
This suffices to prove the lemma, as $S - S$ always contains a non-zero integer if $|S| \geq 2$.

Without loss of generality, assume that $q > 2$.
Let $T$ be a non-empty $q$-good set, and let $n$ be the element of $T$ with maximum absolute value.
Since $T$ is $q$-good, we can write $qn^2 = a^2 + b^2 - c^2 - d^2 \leq 2n^2$ for some $a, b, c, d \in T$.
Then $q > 2$ necessaily implies $n^2 \leq 0$ and thus $n = 0$.
By maximality, this means $T = \{0\}$, as desired.
\end{proof}

\begin{lemma}\label{2017a2-5}
For any $k \geq 3$, $1$ is not $k$-excellent over $\Z$.
\end{lemma}
\begin{proof}
Take $S = \{0, 1, 4, 8, 12, \ldots, 4(k - 2)\}$.
By this construction, $S - S$ contains $1$ and $4$, and all its elements are either odd or divisible by $4$.
In particular, for any $a \in S - S$, we have $a^2 \equiv 0, 1 \pmod{8}$.
Thus $4 = 1 \cdot 1 \cdot 4$ cannot be represented as $a^2 + b^2 - c^2 - d^2$ for any $a, b, c, d \in S - S$.
This shows that $1$ is not $k$-excellent over $\Z$.
\end{proof}

Thus, the $k$-excellent integers are $0$ and $\pm 2$.
The same holds in $R$ when $R$ is commutative with characteristic zero.

We now show a partial converse.
Let $R$ be a domain with $\rchar(R) = 0$.
Suppose that $2$ is $k$-excellent over $R$.
We prove that $R$ has to be commutative.

\begin{lemma}\label{2017a2-6}
For any $x, y \in R$, there exists an integer $c = c_{x, y}$ such that $2xy = c(xy + yx)$.
\end{lemma}
\begin{proof}
If $x$ and $y$ commute, then we can just take $c = 1$.
So now, suppose that $x$ and $y$ do not commute.
Fix a set $M$ of $k - 1 \geq 2$ integers, containing both $0$ and $1$.
For example, $M = \{0, 1, 2, \ldots, k - 2\}$ works.
Take $T = \{mx : m \in M\} \cup \{y\}$.
Note that if $mx = y$ for some $m \in M$, then certainly $x$ and $y$ has to commute; a contradiction.
Also, $x = 0$ implies $xy = yx$, so $x \neq 0$.
By assumption, $R$ is torsion free as a $\Z$-module, i.e., $mx = 0$ implies $m = 0$.
This suffices to justify that $T$ has size $k$.

By assumption, $2$ is $k$-excellent.
So that means $T - T$ is $2$-good.
One can check that $x, y \in T$.
Thus this means that $2xy = a^2 + b^2 - c^2 - d^2$ for some $a, b, c, d \in T - T$.
But $T - T$ can be written as $\{nx : n \in M - M\} \cup \{mx - y, y - mx : m \in M\}$.
In particular, for any $a \in T - T$, we can write either $a^2 = n^2 x^2$ for some $n \in M - M$ or $a^2 = m^2 x^2 + y^2 - m(xy + yx)$ for some $m \in M$.
Oversimplifying, this means we can write $2xy = q_1 x^2 + q_2 y^2 + q_3 (xy + yx)$ for some integers
\[ q_1 \in S_1 := \{n_1^2 + n_2^2 - n_3^2 - n_4^2 : n_1, n_2, n_3, n_4 \in M - M\}, \]
\[ q_2 \in S_2 := \{0, \pm 1, \pm 2\}, \quad q_3 \in S_3 := (M + M) - (M + M). \]
Note that the sets $S_1, S_2, S_3$ are finite and they depend only on $M$.
In summary, we have found finite subsets $S_1, S_2, S_3 \subseteq \Z$ with the following property:
    for any $x, y \in R$, there exists $q_1 \in S_1$, $q_2 \in S_2$, and $q_3 \in S_3$ such that $2xy = q_1 x^2 + q_2 y^2 + q_3 (xy + yx)$.

We now use the previous paragraph and the pigeonhole principle to prove the lemma.
Consider the above result but with $x$ replaced by $\ell x$, where $\ell$ ranges over the integers.
By pigeonhole principle, there exists a fixed triple $(q_1, q_2, q_3) \in S_1 \times S_2 \times S_3$ and an infinite set $L \subseteq \Z$ such that, for any $\ell \in L$,
\[ 2 \ell xy = q_1 \ell^2 x^2 + q_2 y^2 + q_3 \ell (xy + yx). \]
Actually, we will only need three distinct elements.
First, given $\ell_1, \ell_2 \in L$ distinct,
\[ 2 (\ell_1 - \ell_2) xy = q_1 (\ell_1^2 - \ell_2^2) x^2 + q_3 (\ell_1 - \ell_2) (xy + yx). \]
By assumption, $R$ is $\Z$-torsion free, so we get $2xy = q_1 (\ell_1 + \ell_2) x^2 + q_3 (xy + yx)$.
Fixing $\ell_1$, we can find at least two distinct values of $\ell_2$ satisfying the equality.
This forces $q_1 = 0$ and thus $2xy = q_3 (xy + yx)$.
The lemma follows by taking $c = q_3$.
\end{proof}

Now we use the lemma to prove that $R$ is commutative.
By Lemma 6, there exists integers $c$ and $d$ such that $2xy = c(xy + yx)$ and $2(x + 1)y = d(xy + yx + 2y)$.
By algebraic manipulation, we get $2c(x + 1)y = 2dxy + 2cdy$, i.e. $2c(d - 1) y = 2(c - d) xy$.
If $d = 1$, then the second equation yields $xy = yx$.
If $c = d$, then the last equation yields either $d = 1$ or $c = 0$.
But if $c = 0$, then $2xy = 0$ which implies $xy = yx = 0$ since $R$ is a domain.
Otherwise, the last equation necessarily implies that $y$ and $xy$ commute.
That is, $yxy = xyy$, which forces $yx = xy$ or $y = 0$, which also implies $yx = xy$.
This proves that $R$ is commutative.



\subsection*{Implementation details}

We have four cases with distinct treatments.
Thus we need to split the solution across several files.
The basic file \texttt{A2\_basic.lean} contains the definitions and the easier lemmas.
Each of the cases $k = 0$, $k = 1$, and $k = 2$ are implemented in one file.
Meanwhile, the case $k \geq 3$ (only with $\rchar(R) = 0$) are again split into several files in one folder called \texttt{A2\_k\_ge\_3}.




The case $k \geq 3$ contains a large set of results.
Thus it 

Thus we split the problem into $5$ files (for now).
The definitions and small lemmas are put in \texttt{A2\_basic.lean}.
The (short) solution for the case $k = 0$ are in \texttt{A2\_k\_eq\_0.lean}.
The other cases also have their corresponding files, which the "obvious" name.




\end{document}
