\documentclass{article}

\usepackage{amsmath, amsfonts, amssymb, amsthm}
\usepackage{hyperref}

\setlength{\parindent}{0pt}
\setlength{\parskip}{5pt}

\newcommand{\Z}{\mathbb{Z}}

\DeclareMathOperator{\rchar}{char}

\newtheorem{lemma}{Lemma}


\title{IMO 2017 A2}
\author{}
\date{}

\begin{document}

\maketitle



\subsection*{Problem}

Let $R$ be a commutative ring of characteristic zero.
Given $q \in R$, we say that a finite set $S \subseteq R$ is \emph{$q$-good} if for any $u, v \in S$, there exists $a, b, c, d \in S$ such that $quv = a^2 + b^2 - c^2 - d^2$.
Given $k \geq 0$, we say that $q \in R$ is \emph{$k$-excellent} if for any finite set $S \subseteq R$ of size $k$, the set $S - S := \{s - t : s, t \in S\}$ is $q$-good.
Find all $k$-excellent elements of $R$.


\subsection*{Answer}

\begin{itemize}
    \item If $k = 0$ or $k = 1$, any element of $R$ is $k$-excellent.
    \item If $k = 2$, then the set of $k$-excellent elements is $\{-2, -1, 0, 1, 2\}$.
    \item If $k \geq 3$, then the set of $k$-excellent elements is $\{-2, 0, 2\}$.
\end{itemize}



\subsection*{Solution}

Official solution: \url{http://www.imo-official.org/problems/IMO2017SL.pdf}

We follow Solution 1 of the official solution for the case $k \geq 3$.
For the case $k \geq 2$, the solution we present will involve reduction to the case where $R = \Z$.

First, the case $k = 0$ and $k = 1$ are obvious.
We have $\emptyset - \emptyset = \emptyset$, and $S - S = \{0\}$ if $|S| = 1$.
Obviously, $\emptyset$ and $\{0\}$ are $q$-good regardless of the choice of $q \in R$.

It is clear that if $q \in R$ is $k$-excellent, then $-q$ is also $k$-excellent.
It is also clear that $0$ is always $k$-excellent; any set is $0$-good.
In a less obvious way, $1$ is $2$-excellent and $2$ is $k$-excellent for any $k \geq 0$.
These are what we prove next.

\begin{lemma}\label{2017a2-1}
Over any ring $R$, $1$ and $2$ are $2$-excellent.
\end{lemma}
\begin{proof}
For any $S \subseteq R$ of size $2$, there exists $x \in R$ (non-zero) such that $S - S = \{-x, 0, x\}$.
Now, for any $x \in R$, $T = \{-x, 0, x\}$ is $1$-good and $2$-good.
Indeed, for any $u, v \in T$, we have $uv \in \{-x^2, 0, x^2\}$.
All of $-2x^2$, $-x^2$, $0$, $x^2$, and $2x^2$ can be expressed as $a^2 + b^2 - c^2 - d^2$ for some $a, b, c, d \in T$.
This shows that $1$ and $2$ are $2$-excellent.
\end{proof}

\begin{lemma}\label{2017a2-2}
For any $k \geq 0$, over a commutative ring $R$, $2$ is $k$-excellent.
\end{lemma}
\begin{proof}
It suffices to prove that $S - S$ is $2$-good for any finite set $S \subseteq R$.
Indeed, take any $u, v \in S - S$, and express them as $u = w - x$, $v = y - z$ for some $w, x, y, z \in S$.
The lemma follows from the equation
\[ 2(w - x)(y - z) = (w - z)^2 + (x - y)^2 - (w - y)^2 - (x - z)^2. \]
\end{proof}

We now focus on proving that there are no other $k$-excellent elements of $R$.
It turns out that they must be integers, and they are $k$-excellent over $\Z$.
We prove these two results next.

\begin{lemma}\label{2017a2-3}
Let $R$ be a ring and $k \geq 3$.
Then a $k$-excellent element of $R$ must be an integer.
\end{lemma}
\begin{proof}
Take a finite set $S$ of size $k$ consisting only of integers, such that $1 \in S - S$.
For example, $S = \{0, 1, 2, \ldots, k - 1\}$ works.
All we need is that $S - S$ consists only of integers and it contains $1$.
Indeed, by assumption, $S - S$ is $q$-good.
Since $1 \in S - S$, we get $q = a^2 + b^2 - c^2 - d^2$ for some $a, b, c, d \in S - S$.
In particular, $q$ has to be an integer.
\end{proof}

\begin{lemma}\label{2017a2-4}
Let $R$ be a ring with characteristic zero and $k \geq 3$.
Let $n \in \Z$ such that $n$ is $k$-excellent over $R$.
Then $n$ is $k$-excellent over $\Z$.
\end{lemma}
\begin{proof}
It suffices to show that for any finite subset $S \subseteq \Z$, if $S$ is $n$-good over $R$, then $S$ is also $n$-good over $\Z$.
The lemma follows since $S - S$ is $n$-good over $R$ whenever $|S| = k$.

Let $S \subseteq \Z$ be $n$-good over $R$.
For any $u, v \in S$, there exists $a, b, c, d \in S$ such that $nuv = a^2 + b^2 - c^2 - d^2$ over $R$.
But $R$ has characteristic zero, so this equation must also hold over $\Z$.
This proves that $S$ is $n$-good over $\Z$.
\end{proof}

It remains to work over $\Z$.
We show that for any $k \geq 2$, any $k$-excellent integers have absolute value at most $2$.
We also show that for any $k \geq 3$, $1$ is not $k$-excellent.

\begin{lemma}\label{2017a2-5}
Over $\Z$, Any $k$-excellent integer has absolute value at most $2$.
\end{lemma}
\begin{proof}
A more general statement is as follows: for any $q \in \Z$, if $|q| > 2$, then the only $q$-good sets are $\emptyset$ and $\{0\}$.
This suffices to prove the lemma, as $S - S$ always contains a non-zero integer if $|S| \geq 2$.
Without loss of generality, assume that $q > 2$.
Let $T$ be a non-empty $q$-good set, and let $n$ be the element of $T$ with maximum absolute value.
Since $T$ is $q$-good, we can write $qn^2 = a^2 + b^2 - c^2 - d^2 \leq 2n^2$ for some $a, b, c, d \in T$.
Then $q > 2$ necessaily implies $n^2 \leq 0$ and thus $n = 0$.
By maximality, this means $T = \{0\}$, as desired.
\end{proof}

\begin{lemma}\label{2017a2-6}
For any $k \geq 3$, $1$ is not $k$-excellent over $\Z$.
\end{lemma}
\begin{proof}
Take $S = \{0, 1, 4, 8, 12, \ldots, 4(k - 2)\}$.
By this constructing, $S - S$ contains $1$ and $4$, and all its elements are either odd or divisible by $4$.
In particular, for any $a \in S - S$, we have $a^2 \equiv 0, 1 \pmod{8}$.
Thus $4 = 1 \cdot 1 \cdot 4$ cannot be represented as $a^2 + b^2 - c^2 - d^2$ for any $a, b, c, d \in S - S$.
This shows that $1$ is not $k$-excellent over $\Z$.
\end{proof}



\subsection*{Extra notes}

It is of interest to see what happens in a general ring of characteristic zero.
The answer is still the same for $k \leq 2$, but for $k \geq 3$, $0$ might be the only $k$-excellent element.
An example for $k \geq 3$ is as follows.

Let $R$ be the free (non-commutative) ring $\Z[X, Y]$ over two variables.
Then $2 \in R$ is not $3$-excellent.
Indeed, take $S = \{0, X, Y\}$, so that $S - S = \{0, \pm X, \pm Y, \pm (X - Y)\}$.
It is impossible to represent $2XY$ as $a^2 + b^2 - c^2 - d^2$ for some $a, b, c, d \in S - S$.
The $XY$ and $YX$ coefficient of $a^2$ are the same for all $a \in S - S$, but this is false for $2XY$.

At the time of writing, I do not know if there exists a non-commutative ring $R$ such that $2 \in R$ is $k$-excellent for some $k \geq 3$.



\end{document}
