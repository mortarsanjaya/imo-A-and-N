\documentclass{article}

\usepackage{amsmath, amsfonts, amssymb, amsthm}
\usepackage{hyperref}

\setlength{\parindent}{0pt}
\setlength{\parskip}{5pt}

\newcommand{\N}{\mathbb{N}}
\newcommand{\Q}{\mathbb{Q}}
\newcommand{\Z}{\mathbb{Z}}

\newtheorem{lemma}{Lemma}

\title{IMO 2017 N6}
\author{}
\date{}

\begin{document}

\maketitle

\subsection*{Problem}

Find all non-negative integers $n$ such that there exists infinitely many multisets $S$ of positive rational numbers of size $n$ for which
\[ \sum_{q \in S} q \in \Z \text{ and } \sum_{q \in S} \frac{1}{q} \in \Z. \tag{*}\label{2017n6-eq0} \]



\subsection*{Answer}

$n \geq 3$.



\subsection*{Solution}

Official solution: \url{http://www.imo-official.org/problems/IMO2017SL.pdf}

We follow Solution 1 of the official solution.
However, the constructions for $n = 3$ will be explicit; it is described in the comment section.

We say that a multiset $S$ of positive rational numbers is \emph{nice} if it satisfies~\eqref{2017n6-eq0}.
We say that a positive integer $n$ is \emph{good} if there are infinitely nice multisets of size $n$.
First of all, due to the map $S \mapsto \{1\} + S$ on multisets, if $n \geq 0$ is good, then $n + 1$ is good.
Thus, it suffices to show that $2$ is not good but $3$ is good.

A nice multiset of size $2$ corresponds to a pair $(x, y) \in \Q^+$ such that $x + y$ and $1/x + 1/y$ are integers.
Write $x = a/b$ and $y = c/d$, where $a, b, c, d \in \N^+$ and $\gcd(a, b) = \gcd(c, d) = 1$.
Just from $x + y \in \Z$, we claim that $b = d$.
Indeed, we get $bd \mid ad + bc$, so $b \mid ad$ and thus $b \mid d$ since $\gcd(a, b) = 1$.
Similarly, $d \mid b$, so $b = d$.

From $1/x + 1/y \in \Z$, we also get $a = c$.
In particular, this means $x = y$ and $2x, 2/x \in \Z$.
Direct bashing yields $x = y \in \{1/2, 1, 2\}$.
So, this means that $2$ is not good; there are only $3$ nice multisets of size $2$.

To prove that $3$ is good, for any $k \geq 0$, take the multiset $\left\{\frac{1}{A_k}, \frac{F_{2k + 1}}{A_k}, \frac{F_{2k + 3}}{A_k}\right\}$, where $A_k = F_{2k + 3} + F_{2k + 1} + 1$.
Clearly their sum is $1$, so we only have to prove that for any $k \geq 0$,
\[ \frac{F_{2k + 3} + 2}{F_{2k + 1} + 1} + \frac{F_{2k + 1} + 2}{F_{2k + 3} + 1} = 3. \]
This can be rewritten as
\[ F_{2k + 3}^2 + F_{2k + 1}^2 + 1 = 3 F_{2k + 3} F_{2k + 1}, \]
    or just $F_{2k + 2}^2 + 1 = F_{2k + 3} F_{2k + 1}$.
This is a well-known identity for Fibonacci numbers.



\end{document}
