\documentclass{article}

\usepackage{amsmath, amsfonts, amssymb, amsthm}
\usepackage{hyperref}

\setlength{\parindent}{0pt}
\setlength{\parskip}{5pt}

\newcommand{\N}{\mathbb{N}}

\newtheorem{lemma}{Lemma}

\title{IMO 2017 C7}
\author{}
\date{}

\begin{document}

\maketitle



\section*{Problem}

Let $S$ denote the set of finite subsets of $\N$.
For any $X \in S$ and $k \in \N$, denote by $f_X(k)$ the $k^{\text{th}}$ smallest natural number not in $X$, indexed from $0$.
That is, $f_X(0)$ is the smallest natural number not in $X$.
For any finite sets $X, Y \in S$, denote $XY = X \cup f_X(Y)$.
Also, for $k \geq 0$, define $X^k$ recursively by $X^0 = \emptyset$ and $X^{k + 1} = XX^k$.

Let $A, B \in S$ such that $AB = BA$.
Prove that $A^{|B|} = B^{|A|}$.



\section*{Solution}

Official solution: \url{http://www.imo-official.org/problems/IMO2017SL.pdf}

We follow Solution 1; we view finite subsets of $\N$ as a function via the map $X \mapsto f_X$.
This allows us to borrow an idea from abstract algebra.
Lemma 2 has an alternative proof using the property of the functions $f_X$.
We will use this proof instead of one given in Solution 1.

It is easy to verify that the map $X \mapsto f_X$ is injective on $S$.
The range of $f_X$ is just $X^c$, and a strictly increasing function from $\N$ to itself is determined by its range.
Thus, the following lemma implies that multiplication on $S$ is associative.
In fact, it makes $S$ a monoid, as one can easily check that $\emptyset X = X \emptyset = X$ for any $X \in S$.

\begin{lemma}\label{2017c7-1}
For any $X, Y \in S$, we have $f_{XY} = f_X \circ f_Y$.
\end{lemma}
\begin{proof}
it suffices to show that $f_{XY}(\N) = f_X(f_Y(\N))$, or equivalently, $(XY)^c = f_X(Y^c)$.
Indeed,
\[ f_X(Y^c) = f_X(\N) \setminus f_X(Y) = \N \setminus (X \cup f_X(Y)) = (XY)^c. \]
\end{proof}

In particular, the definition of power of an element of $S$ given in the problem statement is just the monoid power on $S$.
Furthermore, $X \mapsto f_X$ is a monoid homomorphism; the multiplication on $\N^\N$ is defined by composition.

Given $A, B \in S$ with $AB = BA$, it is clear that $A^{|B|} B^{|A|} = B^{|A|} A^{|B|}$.
The definition of multiplication yields $|XY| = |X| + |Y|$ for any $X, Y \in S$.
Then one can check that $|A^{|B|}| = |A||B| = |B^{|A|}|$.
Thus, it suffices to show that for any $X, Y \in S$, if $XY = YX$ and $|X| = |Y|$, then $X = Y$.
It follows from the following lemma by working over strictly increasing functions on $\N$.

\begin{lemma}\label{2017c7-2}
Let $f, g : \N \to \N$ strictly increasing.
Suppose that $fg = gf$ and $f(n) = g(n)$ for every $n$ large enough.
Then $f = g$.
\end{lemma}
\begin{proof}
It suffices to show that for any $N \geq 0$, if $f(n) = g(n)$ for all $n > N$, then $f(N) = g(N)$.
The lemma would then follow by downwards induction.

Now fix some such $N$.
Since $f$ and $g$ are strictly increasing, we have $f(N), g(N) \geq N$.
If $f(N) > N$, induction hypothesis yields $f(g(N)) = g(f(N)) = f(f(N))$.
By injectivity, we get $f(N) = g(N)$.
Similarly, we get $f(N) = g(N)$ if $g(N) > N$.
Otherwise, we have $f(N) = g(N) = N$.
\end{proof}

\end{document}
