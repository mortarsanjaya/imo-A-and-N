\documentclass{article}

\usepackage{amsmath, amsfonts, amssymb, amsthm}
\usepackage{hyperref}

\setlength{\parindent}{0pt}
\setlength{\parskip}{5pt}

\newtheorem*{claim}{Claim}

\title{IMO 2017 A3}
\author{}
\date{}

\begin{document}

\maketitle



\subsection*{Problem}

Let $S$ be a finite set, and fix some $f : S \to S$.
Suppose that, for any $g : S \to S$, $f \circ g \circ f = g \circ f \circ g$ implies $g = f$.
Prove that $f^2(S) = f(S)$.



\subsection*{Solution}

Official solution: \url{http://www.imo-official.org/problems/IMO2017SL.pdf}

We will follow the official solution.
We also generalize the result to any monoids, treating $S^S$ as a monoid under function composition.

First, notice that it suffices to show that $f^m = f$ for some $m > 1$.
Indeed, this means $f(S) \subseteq f^m(S) \subseteq f^2(S) \subseteq f(S)$.
The reduced statement can now be solved under a more generalized setting as follows.

\begin{claim}
Let $M$ be a finite monoid and fix some $x \in M$.
Suppose that, for any $y \in M$ such that $xyx = yxy$, we have $x = y$.
Then $x^m = x$ for some $m > 1$.
\end{claim}
\begin{proof}
Since $M$ is finite, there exists $k > 0$ and $n \geq 0$ such that $x^{n + k} = x^n$.
Take $y = x^{dk + 1}$, where $d > 0$ is large enough so that $dk + 3 \geq n$.
Then $xyx = x^{dk + 3}$ and $yxy = x^{2dk + 3}$.
Since $dk + 3 \geq n$ and $x^{n + k} = x^n$, we have $x^{2dk + 3} = x^{dk + 3}$.
So, $xyx = yxy$, and thus $x^{dk + 1} = y = x$.
We can then take $m = dk + 1$, which works since $d > 0$.
\end{proof}



\end{document}
