\documentclass{article}

\usepackage{amsmath, amsfonts, amssymb, amsthm}
\usepackage{hyperref}

\setlength{\parindent}{0pt}
\setlength{\parskip}{5pt}

\newcommand{\N}{\mathbb{N}}

\title{IMO 2015 N2}
\author{}
\date{}

\begin{document}

\maketitle



\subsection*{Problem}

Let $a, b \in \N$ such that $a! + b! \mid a! b!$.
Prove that $3a \geq 2b + 2$, and find all the equality cases.



\subsection*{Answer}

Equality holds iff $(a, b) \in \{(2, 2), (4, 5)\}$.



\subsection*{Solution}

Official solution: \url{http://www.imo-official.org/problems/IMO2015SL.pdf}

We present Solution 1 of the official solution below.
The original problem only asks to prove the inequality, but the equality case comes naturally afterwards.

Clearly, $b! > 1$, so $b \geq 2$.
If $a \geq b$, then the inequality $3a \geq 2b + 2$ is obvious.
So now, assume that $a < b$, and write $b = a + c$ for some $c \in \N$ positive.
It now suffices to show that $a \geq 2c + 2$.

Clearly, $a! + b! \mid a! b!$ implies $a! + b! \mid (a!)^2$.
Then $a! + b! = a! + (a + c)! = a!$ divide $(a!)^2$.
This simplifies to
\[ 1 + \binom{a + c}{c} c! = 1 + (a + 1)(a + 2) \ldots (a + c) \mid a!. \]
Clearly, $a > c$, and the above also means that
\[ 1 + (a + 1)(a + 2) \ldots (a + c) \mid (c + 1)(c + 2) \ldots a. \tag{1}\label{2015n2-eq1} \]
    since the divisor is coprime with $1, 2, \ldots, c$.
Now this implies the inequality
\[ (a + 1)(a + 2) \ldots (a + c) < (c + 1)(c + 2) \ldots a. \]

Clearly, this fails if $a \leq 2c$.
Next, consider the case $a = 2c + 1$.
Then $a + 1 = 2(c + 1)$, so the factor $c + 1$ on the right hand side of~\eqref{2015n2-eq1} can be cancelled out.
Now the divisibility implies
\[ (a + 1)(a + 2) \ldots (a + c) < (a - c + 1) \ldots a, \]
    which is again a contradiction.

Finally, consider the case $a = 2c + 2$.
Plugging into~\eqref{2015n2-eq1} yields
\[ 1 + (2c + 3)(2c + 4) \ldots (3c + 2) \mid (c + 1)(c + 2) \ldots (2c + 2). \]
It is easy to check that $c = 0$ works (with the product on the LHS taken to be $1$).
It is also easy to check that $c = 1$ works, so it remains to show that $c \leq 1$.

Suppose for the sake of contradiction that $c \geq 2$.
For $c = 2$, we would have $57 \mid 360$; a contradiction.
For $c \geq 3$, it suffices to show that $c + 2$ and $c + 3$ divides $M = (2c + 3)(2c + 4) \ldots (3c + 2)$.
Indeed, then they will be coprime with $1 + M$, and so
\[ 1 + M \mid (c + 1)(c + 4)(c + 5) \ldots (2c + 2). \]
This is a contradiction since for $c \geq 3$,
\[ M = (2c + 3)(2c + 4) \ldots (3c + 2) \geq (c + 3)(c + 4) \ldots (2c + 2). \]

Now we show that $c + 2$ and $c + 3$ divides $M$ for $c \geq 3$.
For $c \geq 3$, the term $2c + 4 = 2(c + 2)$ exists in the product expansion of $M$, so $c + 2 \mid M$.
For $c \geq 4$, the term $2c + 6 = 2(c + 3)$ exists in the product expansion of $M$, so $c + 3 \mid M$.
Finally, $c + 3 \mid M$ for $c = 3$ follows by direct computation.



\end{document}
