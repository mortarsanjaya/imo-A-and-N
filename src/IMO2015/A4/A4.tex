\documentclass{article}

\usepackage{fullpage}
\usepackage{amsmath, amsfonts, amssymb, amsthm}
\usepackage{hyperref}

\setlength{\parindent}{0pt}
\setlength{\parskip}{5pt}

\DeclareMathOperator{\rchar}{char}

\newtheorem*{claim}{Claim}

\title{IMO 2015 A4 (P5)}
\author{}
\date{}

\begin{document}

\maketitle



\subsection*{Problem}

Fix a domain $R$.
Determine all functions $f : R \to R$ such that, for any $x, y \in R$,
\[ f(x + f(x + y)) + f(xy) = x + f(x + y) + f(x) y. \tag{*}\label{2015a4-eq0} \]



\subsection*{Answer}

The identity map and $x \mapsto 2 - x$.



\subsection*{Solution}

Official solution: \url{http://www.imo-official.org/problems/IMO2015SL.pdf}

We modify the official solution for the case where $\rchar(R) \neq 2$.
We will proceed with our own solution for the case $\rchar(R) = 2$.
First, it is easy to see that the identity map and $x \mapsto 2 - x$ satisfies~\eqref{2015a4-eq0}.


\subsubsection*{Case 1: $\rchar(R) \neq 2$.}

Substituting $(x, y) = (t, 1)$ into~\eqref{2015a4-eq0} yields
\[ f(t + f(t + 1)) = t + f(t + 1) \quad \forall t \in R. \tag{1}\label{2015a4-eq1} \]
Substituting $(x, y) = (0, t + f(t + 1))$ into~\eqref{2015a4-eq0} and using~\eqref{2015a4-eq1} yields $f(0) = f(0) (t + f(t + 1))$ for all $t \in R$.
In particular, if $f(0) \neq 0$, since $R$ is a domain, we get $t + f(t + 1) = 1$ for all $t \in R$.
Replacing $t$ with $t - 1$ gives us $f(t) = 2 - t$ for all $t \in R$.
Thus, from now on, we assume that $f(0) = 0$ and prove that $f$ is the identity map.

First, plugging $t = -1$ into~\eqref{2015a4-eq1} gives us $f(-1) = -1$.
Plugging $(x, y) = (1, -1)$ into~\eqref{2015a4-eq0} now gives us $f(1) = 1$.
We now claim that we are done if $f$ is odd.

Consider the equation~\eqref{2015a4-eq0} with $(x, y) = (1, t)$ and $(x, y) = (-1, -t)$.
If $f$ is odd, adding the two equations yield $2 f(t) = 2t$.
But $R$ is a domain and $\rchar(R) \neq 2$.
This implies $f(t) = t$ for any $t \in R$.
It now remains to show that $f$ is odd.

If we set $y = -1$ into~\eqref{2015a4-eq0}, we get $f(x + f(x - 1)) + f(-x) = x + f(x - 1) - f(x)$ for any $x \in R$.
Thus, to prove that $f$ is odd, it suffices to prove that $x + f(x - 1)$ is a fixed point of $f$.
Replacing $x$ with $x + 2$, it suffices to prove that $x + f(x + 1) + 2$ is a fixed point of $f$ for any $x \in R$.

Set $t = x + f(x + 1)$.
Then~\eqref{2015a4-eq1} tells us that $f(t) = t$.
Meanwhile,~\eqref{2015a4-eq0} with $(x + 1, 0)$ replacing $(x, y)$ gives us $f(t + 1) = t + 1$.
Finally,~\eqref{2015a4-eq0} with $(x, y) = (1, t)$ gives us $f(t + 2) = t + 2$.


\subsubsection*{Case 2: $\rchar(R) = 2$.}

Substituting $f(x) = g(x) + x$ for some $g : R \to R$, it suffices to show that $g = 0$.
Equation~\eqref{2015a4-eq0} now reads as
\[ g(y + g(x + y)) + g(xy) = g(x) y \; \forall x, y \in R. \tag{2}\label{2015a4-eq2} \]
Our aim is to obtain the two following equalities: for any $y \in R$,
\[ g(t + 1) (t + 1) = g(t) t, \tag{3}\label{2015a4-eq3} \]
\[ g(g(t) t) = g(t). \tag{4}\label{2015a4-eq4} \]
The two equalities suffice to show that $g = 0$.
Indeed, for any $t \in R$,~\eqref{2015a4-eq3} and~\eqref{2015a4-eq4} yields $g(t + 1) = g(g(t + 1) (t + 1)) = g(g(t) t) = g(t)$.
Plugging into~\eqref{2015a4-eq3} gives us $g(t) (t + 1) = g(t) t$, and thus $g(t) = 0$.

First, substituting $(x, y) = (1, 1)$ into~\eqref{2015a4-eq2} gives us $g(1 + g(0)) = 0$.
Then plugging $(x, y) = (0, 1 + g(0))$ into~\eqref{2015a4-eq2} yields $g(0)^2 = 0$, so $g(0) = 0$.
In particular, we also get $g(1) = 0$.

Now, to prove~\eqref{2015a4-eq3}, for an arbitrary $t \in R$, plug $(x, y) = (t, t + 1)$ into~\eqref{2015a4-eq2}.
We get $g(t + 1) + g(t(t + 1)) = g(t) (t + 1)$, or $g(t + 1) + g(t) (t + 1) = g(t(t + 1))$.
Similarly, with $t + 1$ in place of $t$, we get $g(t) + g(t + 1) t = g(t(t + 1))$.
Then we have $g(t + 1) + g(t) (t + 1) = g(t) + g(t + 1) t$, implying~\eqref{2015a4-eq3}.

To prove~\eqref{2015a4-eq4}, first plug $(x, y) = (t, 0)$ into~\eqref{2015a4-eq2} to get $g(g(t)) = 0$.
Next, plugging $(x, y) = (0, t)$ into~\eqref{2015a4-eq2} yields $g(g(t) + t) = 0$.
Then plugging $(x, y) = (g(t), t)$ into~\eqref{2015a4-eq2} yields~\eqref{2015a4-eq4}.



\end{document}
