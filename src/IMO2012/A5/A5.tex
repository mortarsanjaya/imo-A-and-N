\documentclass{article}

\usepackage{amsmath, amsfonts, amssymb, amsthm}
\usepackage{hyperref}
\usepackage{float}

\setlength{\parindent}{0pt}
\setlength{\parskip}{5pt}

\newcommand{\F}{\mathbb{F}}
\newcommand{\R}{\mathbb{R}}
\newcommand{\Z}{\mathbb{Z}}

\DeclareMathOperator{\rchar}{char}

\newtheorem{lemma}{Lemma}
\newtheorem*{claim}{Claim}

\title{IMO 2012 A5}
\author{}
\date{}

\begin{document}

\maketitle





\section*{Problem}

Let $R$ be a commutative ring and $S$ be a division ring.
Find all functions $f : R \to S$ such that, for any $x, y \in R$,
\[ f(xy + 1) - f(x + y) = f(x) f(y). \tag{*}\label{2012a5-eq0} \]

\textit{
    Note: In the solution below, we will only completely solve the problem for the case $R = \R$.
    We will present some cases in generality; the rest will be done with the assumption $R = \R$.}





\section*{Answer}

The solutions can be described via the following nine classes of functions.
For the $8^{\text{th}}$ class, $\phi, \psi \in S$ satisfies $\phi + \psi = 1$ and $\phi \psi = -1$.
The map into the intermediate ring is necessarily a ring homomorphism in this context.

\begin{table}[H]
\centering
\begin{tabular}{|c|c|c|}
    \hline
    No. & Intermediate ring & Map to $S$ \\ \hline
    1 & $0$ & $0 \mapsto 0$ \\ \hline
    2 & $S$ & $x \mapsto x - 1$ \\ \hline
    3 & $\F_3$ & $[0, 1, 2] \mapsto [-1, 0, 1]$ \\ \hline
    4 & $S$ & $x \mapsto x^2 - 1$ \\ \hline
    5 & $\F_3$ & $[0, 1, 2] \mapsto [-1, 0, 0]$ \\ \hline
    6 & $\Z/4\Z$ & $[0, 1, 2, 3] \mapsto [-1, 0, 1, 0]$ \\ \hline
    7 & $\F_2$ & $[0, 1] \mapsto [-1, 0]$ \\ \hline
    8 & $\F_4$ & $[0, 1, X, X + 1] \mapsto [-1, 0, \phi, \psi]$ \\ \hline
    9 & $\F_2[X]/\langle X^2 \rangle$ & $[0, 1, X, X + 1] \mapsto [-1, 0, 1, 0]$ \\ \hline
\end{tabular}
\end{table}





\section*{Solution}

Through this solution, we assume that $R$ is an arbitrary ring, except when specified.
We also assume that $S$ is a domain, though not necessarily commutative in most cases.

We start with some small observations.
Plugging $x = y = 1$ into~\eqref{2012a5-eq0} yields $f(1) = 0$.
Now plugging $y = 0$ into~\eqref{2012a5-eq0} yields $-f(x) = f(x) f(0)$ for all $x \in R$.
This means that either $f \equiv 0$ or $f(0) = -1$.
From now on, we assume that $f(0) = -1$.

Plugging $y = -1$ into~\eqref{2012a5-eq0} yields $f(1 - x) - f(x - 1) = f(x) f(-1)$ for all $x \in R$.
This also implies $f(-x) - f(x) = f(x + 1) f(-1)$ for all $x \in R$.
In particular, the case $f(-1) = 0$ yields that $f$ is even.

Before we split into cases, we prove some lemmas.
As the solution classes involve homomorphisms into explicit rings, we want to take quotients.
The idea of the lemmas is that we can pass on to the quotients of $R$.
For convenience, from now on we assume that $R$ is commutative and that $S$ is a domain.

\begin{lemma}\label{2012a5-1}
For any $c \in R$, we have $f(c + x) = -f(c) f(x)$ for all $x \in R$ if and only if $f(cx + 1) = 0$ for all $x \in R$.
The set $J$ of such $c$ forms an ideal of $R$.
\end{lemma}
\begin{proof}
The first part is clear since $f(cx + 1) - f(c + x) = f(c) f(x)$.
For the second part, closure under addition is easy to prove from the equation $f(c + x) = -f(c) f(x)$.
Meanwhile, closure under multiplication in $R$ follows from the equation $f(cx + 1) = 0$.
Finally, $f(1) = 0$ yields that $0 \in J$.
\end{proof}

\begin{lemma}\label{2012a5-2}
Let $I = \{c \in R : \forall x \in R, f(c + x) = f(x)\}$ and $J = \{c \in R : \forall x \in R, f(cx + 1) = 0\}$.
Then $c \in I$ if and only if $c \in J$ and $f(c) = -1$.
The set $I$ is an ideal of $R$.
\end{lemma}
\begin{proof}
If $c \in I$, then $f(c) = f(0) = -1$, and $c \in J$ holds by the previous lemma.
The converse is also easy to check.
Now we prove that $I$ is an ideal of $R$.
It is easily a subgroup of $R$, so we just need to prove closure under multiplication over $R$.

Let $c \in I$ and $x, y \in R$.
Since $f(c + z) = f(z)$ for all $z \in R$,~\eqref{2012a5-eq0} yields $f(x(c + y) + 1) = f(xy + 1)$ for all $y \in R$.
Since $J$ is an ideal containing $I$, we have $cy \in J$.
That means we have $-f(cx) f(xy + 1) = f(xy + 1)$ for all $x, y \in R$.
Fixing $x$, this implies either $f(cx) = -1$ or $f(xy + 1) = 0$ for all $y \in R$.
Note that $cx \in J$ since $c \in I \subseteq J$.
Thus, we get either $cx \in I$ or $x \in J$.
That is, if $x \notin J$, then $cx \in I$.

Finally, take an arbitrary $c \in I$ and $x \in R$.
If $x \notin J$, then $cx \in I$ as we've seen above.
If $x \in J$, then $x - 1, 1 \notin J$, so $c(x - 1), c \in I$ and thus $cx \in I$.
\end{proof}

For any $a, b \in R$ such that $a - b \in I$, it is clear that $f(a) = f(b)$.
Thus, we get an induced map $g : R/I \to S$ such that $f = g \circ \phi$, where $\phi : R \to R/I$ is the quotient map.
Since $\phi$ is surjective, one can check that $g$ satisfies~\eqref{2012a5-eq0} as well.
Thus, from now on, we can WLOG assume that $I = \{0\}$.

We now split into two cases: $f(-1) \neq 0$ and $f(-1) = 0$.
The above equation is only saying that $f$ is even if $f(-1) = 0$.
We start with the former case.



\subsection*{Case 1: $f(-1) \neq 0$}

The equation $f(-x) - f(x) = f(x + 1) f(-1)$ now easily yields
\[ f(1 - x) = -f(1 + x). \tag{1.1}\label{2012a5-eq1-1} \]
Now plug $x = 2$ into~\eqref{2012a5-eq0}.
Using~\eqref{2012a5-eq1-1}, we have $f(2) = -f(0) = 1$ and $f(y + 2) = -f(-y)$, so
\[ f(2y + 1) = f(y) - f(-y) = -f(y + 1) f(-1). \tag{1.2}\label{2012a5-eq1-2} \]
By replacing $(x, y)$ with $(-x, -y)$ in~\eqref{2012a5-eq0}, we get $f(xy + 1) - f(-x - y) = f(-x) f(-y)$.
Subtracting by~\eqref{2012a5-eq0} and applying~\eqref{2012a5-eq1-2} yields
\[ -f(x + y + 1) f(-1) = f(x + y) - f(-x - y) = f(-x) f(-y) - f(x) f(y). \tag{1.3}\label{2012a5-eq1-3} \]

For $x = y$,~\eqref{2012a5-eq1-2} yields $(f(-x) - f(x)) f(-1) = f(-x)^2 - f(x)^2$.
Since $f(-x)$ and $f(x)$ commute, $f(-x)^2 - f(x)^2 = (f(-x) - f(x))(f(-x) + f(x))$.
Thus, for any $x \in R$, we have either $f(-x) = f(x)$ or $f(-x) + f(x) = f(-1)$.
Since $f(-x) - f(x) = f(x + 1) f(-1)$, the former is equivalent to $f(x + 1) = 0$.
Replacing $x$ with $x - 1$, we get
\[ f(x) \neq 0 \implies f(x - 1) - f(x + 1) = f(-1). \tag{1.4}\label{2012a5-eq1-4} \]

On the other hand, suppose that $f(x) = 0$.
Replace $x$ with $x - 1$ in~\eqref{2012a5-eq1-3}, then plug $y = x$.
We get $f(2x - 1) - f(1 - 2x) = f(1 - x) f(-x)$.
By~\eqref{2012a5-eq1-2}, $f(2x - 1) = f(x - 1) - f(1 - x) = -f(x) f(-1) = 0$, and $f(1 - 2x) = -f(1 - x) f(-1)$.
Thus $f(x - 1) = f(1 - x)$ and $f(1 - x) f(-x) = f(1 - x) f(-1)$.
In particular this also means $f(-x) f(1 - x) = f(-1) f(1 - x)$.
Since $f(1 - x) = -f(x + 1)$, we get $f(-x) f(x + 1) = f(-1) f(x + 1)$.
By plugging $y = -x - 1$, we get $f(-1) = f(-x) f(x + 1) = f(-1) f(x + 1)$.
Since $f(-1) \neq 0$, this implies $f(x + 1) = 1$.
In summary,
\[ f(x) = 0 \implies f(x + 1) = 1 \text{ and } f(x - 1) = -1. \tag{1.5}\label{2012a5-eq1-5} \]

\begin{claim}
$f(-1) \in \{-2, 1\}$.
\end{claim}
\begin{proof}
Due to~\eqref{2012a5-eq1-1}, the claim is equivalent to $f(3) \in \{-1, 2\}$.
Now denote $C = f(3) = -f(-1)$ for convenience.
Since $f(0) = -1$, we have $f(2) = 1$.
By the equation $f(x + 2) + f(x) = C f(x + 1)$, we have $f(4) = C^2 - 1$ and $f(5) = C^3 - 2C$.
Plugging $x = y = 2$ into~\eqref{2012a5-eq0} yields $C^3 - 2C - (C^2 - 1) = 1$, which is equivalent to $C(C + 1)(C - 2) = 0$.
This proves the claim, since $C = -f(-1) \neq 0$.
\end{proof}


\subsubsection*{Subcase 1.1: $f(-1) = -2 \neq 0$}

In particular $\rchar(S) \neq 2$.

The corresponding class of solutions is class 2.
We claim that $f + 1 : R \to S$ is a ring homomorphism.
    
Combining~\eqref{2012a5-eq1-4} and~\eqref{2012a5-eq1-5} yields $f(x - 1) - f(x + 1) = -2$ for all $x \in R$.
Using~\eqref{2012a5-eq1-1} and replacing $x$ with $x + 1$, we can rewrite this as $f(x) + f(-x) = -2$ for all $x \in R$.
On the other hand, we have $f(x) - f(-x) = 2 f(x + 1)$ for all $x \in R$.
Adding the two equations yields $2 f(x) = 2 (f(x + 1) - 1)$ for all $x \in R$.
Since $\rchar(S) \neq 2$, this yields $f(x + 1) = f(x) + 1$ for all $x \in R$.

Assuming that $f + 1$ is additive, it is easy to show that $f + 1$ is multiplicative.
Indeed, the original equality yields
\[ f(xy) + 1 = f(x) f(y) + f(x) + f(y) + 1 = (f(x) + 1)(f(y) + 1). \]
It remains to show that $f + 1$ is additive, i.e., $f(x + y) = f(x) + f(y) + 1$ for all $x, y \in R$.
We go back to~\eqref{2012a5-eq1-3} and use the equality $f(x) + f(-x) = -2$, giving us:
\[ 2 f(x + y + 1) = (f(x) + 2) (f(y) + 2) - f(x) f(y) = 2 (f(x) + f(y) + 2). \]
Again, $\rchar(S) \neq 2$, so $f(x + y) + 1 = f(x + y + 1) = f(x) + f(y) + 2$.
We are done.


\subsubsection*{Subcase 1.2: $f(-1) = 1 \neq -2$}

In particular, $\rchar(S) \neq 3$.

The corresponding class of solutions is class 3.
Using Lemma~\ref{2012a5-2}, we can assume that $I = 0$.
Then it suffices to show that $R \cong \F_3$, and $f : \F_3 \to S$ is defined by the corresponding map.
We already know that $f(-1) = 1$, $f(0) = -1$, and $f(1) = 0$.
So, it now suffices to show that $R \cong \F_3$.

First, we prove that $f(c + 1) = 0$ implies $c = 0$.
We have $f(1 - c) = 0$ by~\eqref{2012a5-eq1-1}.
Then~\eqref{2012a5-eq1-5} yields $f(c) = f(-c) = -1$.
Plugging $(x, y) \to (c, x - 1)$ yields $f(c + x) = f(x)$ for all $x \in R$.
Since $I = 0$, this implies $c = 0$, as desired.

The equation~\eqref{2012a5-eq1-2} and~\eqref{2012a5-eq1-1} yields $f(x - 1) + f(x) + f(x + 1) = 0$ for all $x \in R$.
In particular, this implies that $3 \in I$, thus $R$ either has characteristic $3$ or is trivial.
The latter is impossible since $f(0) = -1 \neq 0 = f(1)$.
To show that $R \cong \F_3$, it remains to show that $R = \{-1, 0, 1\}$.

Fix some $x \in R$, and suppose that $x \notin \{-1, 0, 1\}$.
Then $f(x - 1), f(x), f(x + 1)$ are all non-zero.
By~\eqref{2012a5-eq1-5}, $f(x - 1) - f(x + 1) = f(x + 1) - f(x) = f(x) - f(x - 1) = 1$.
Adding all three yields $3 = 0$ in $S$, which is impossible by the subcase assumption, $1 \neq -2 \in S$.
We are done with this subcase.



\subsection*{Case 2: $f(-1) = 0$ (Not generalized)}

\textit{Note: For now, we just work with the case $R = \R$.}

Let us assume that $R = \R$ for this case.
Then the only corresponding solution is $x \mapsto \phi(x)^2 - 1$.

By the equation $f(x) - f(-x) = -f(x + 1) f(-1)$, this time, we get that $f$ is even.
In particular we have $f(xy + 1) - f(x + y) = f(1 - xy) - f(x - y)$ for all $x, y \in \R$.
Rearranging gives us $f(xy + 1) - f(xy - 1) = f(x + y) - f(x - y)$.
In particular, for any $a, b \in \R$, solving for $x + y = a$ and $x - y = b$ gives us
\[ f\left(\frac{a^2 - b^2}{4} + 1\right) - f\left(\frac{a^2 - b^2}{4} - 1\right) = f(a) - f(b). \]
It follows that $f(a) - f(b) = f(c) - f(d)$ for any $a, b, c, d \in \R$ such that $a^2 - b^2 = c^2 - d^2$.
In particular, since $f(0) = -1$, we have $f(\sqrt{u + v}) = f(\sqrt{u}) + f(\sqrt{v}) + 1$ for all $u, v \geq 0$.

Now define $g : \R_{\geq 0} \to R$ by $g(x) = f(\sqrt{x})$ for all $x \geq 0$.
Since $f$ is even, we have $f(x) = g(x^2)$ for all $x \in \R$.
The previous paragraph implies that $g + 1$ is additive.
It remains to show that it is multiplicative.
Going back to~\eqref{2012a5-eq0}, for all $x, y \in \R$, we have
\[ g(x^2) g(y^2) = g(1 + 2xy + x^2 y^2) - g(x^2 + 2xy + y^2) = g(1) + g(x^2 y^2) - g(x^2) - g(y^2). \]
Note that $g(1) = f(1) = 0$.
Thus rearranging gives $g(x^2 y^2) + 1 = (g(x^2) + 1)(g(y^2) + 1)$ for all $x, y \in \R$.
Replacing with square roots prove that $g + 1$ is multiplicative.





\section*{Implementation details}

At the time of writing, \texttt{mathlib} does not have the notion of two-sided ideal.
Thus, we only solve for the case where $R$ is commutative, with the ideal quotient API available.

The file \texttt{A5\_basic} contains the definition of~\eqref{2012a5-eq0} and basic results.
The file \texttt{A5\_period\_quot} contains the construction of the ideal $I$ of periods and the corresponding lift of $f$.
The file \texttt{A5\_main\_real} is just the final solution file for the case $R = \R$.
The rest are split up according to the subcases.



\end{document}
