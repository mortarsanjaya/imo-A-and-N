\documentclass{article}

\usepackage{fullpage}
\usepackage{amsmath, amsfonts, amssymb, amsthm}
\usepackage{hyperref}
\usepackage{float}

\setlength{\parskip}{5pt}

\newcommand{\F}{\mathbb{F}}
\newcommand{\R}{\mathbb{R}}
\newcommand{\Z}{\mathbb{Z}}

\DeclareMathOperator{\rchar}{char}

\newtheorem{lemma}{Lemma}
\newtheorem{corollary}{Corollary}
\newtheorem*{claim}{Claim}

\title{IMO 2012 A5}
\author{}
\date{}

\begin{document}

\maketitle





\section*{Problem}

Let $R$ be a commutative ring and $S$ be a field.
Find all functions $f : R \to S$ such that, for any $x, y \in R$,
\[ f(xy + 1) - f(x + y) = f(x) f(y). \tag{*}\label{2012a5-eq0} \]

\textbf{Warning.}
This problem is an extremely buffed version of the original problem with $R = S = \R$ and without the condition $f(-1) \neq 0$.
The original problem seems to be of difficulty level A5.
If $f(-1) = 0$ is allowed, then the difficulty level would only rise to A6 in the modern era.
Note that Cauchy's functional equation ($f(x + y) = f(x) + f(y)$) becomes well-known nowadays.

However, I believe that the difficulty level of this problem is way beyond the IMO.
Perhaps, the difficulty, mainly coming from its technicality, could be as high as A20.









\section*{Answer}

The solutions can be described via the following nine classes of functions.
For the $8^{\text{th}}$ class, $\phi, \psi \in S$ satisfies $\phi + \psi = 1$ and $\phi \psi = -1$.
The map into the intermediate ring is necessarily a ring homomorphism in this context.

\begin{table}[H]
\centering
\begin{tabular}{|c|c|c|}
    \hline
    Class & Intermediate ring & Map to $S$ \\ \hline
    1 & $0$ & $0 \mapsto 0$ \\ \hline
    2 & $S$ & $x \mapsto x - 1$ \\ \hline
    3 & $S$ & $x \mapsto x^2 - 1$ \\ \hline
    4 & $\F_2$ & $[0, 1] \mapsto [-1, 0]$ \\ \hline
    5 & $\F_3$ & $[0, 1, 2] \mapsto [-1, 0, 1]$ \\ \hline
    6 & $\F_3$ & $[0, 1, 2] \mapsto [-1, 0, 0]$ \\ \hline
    7 & $\Z/4\Z$ & $[0, 1, 2, 3] \mapsto [-1, 0, 1, 0]$ \\ \hline
    8 & $\F_2[X]/\langle X^2 \rangle$ & $[0, 1, X, X + 1] \mapsto [-1, 0, 1, 0]$ \\ \hline
    9 & $\F_4$ & $[0, 1, X, X + 1] \mapsto [-1, 0, \phi, \psi]$ \\ \hline
\end{tabular}
\end{table}









\section*{Solution}

I will present a very long and technical solution.
The solution seems to be generalizable to the case where $R$ is not necessarily commutative.
Most parts of the solution only requires $S$ to be a domain, not even commutative.
Also, most parts can be done without assuming that $R$ is commutative.
However, one part seems to require that $S$ is a field.

Regardless, \texttt{mathlib}'s API for ideal quotient of a ring (\texttt{ideal.quotient}) only supports commutative rings.
Another possibility is using \texttt{ring\_con} to construct the quotients, but I found it hard to use.
Thus, to avoid complications, we will assume that $R$ is commutative.
We also assume that $S$ is an integral domain for the same reason and for the benefit of access to the \texttt{ring} tactic.









\subsection*{Step 1: Small observations}

One can check manually that all functions described above works.
We now focus on showing that they are all the solutions.

Plugging $x = y = 1$ into~\eqref{2012a5-eq0} yields $f(1) = 0$.
Now plugging $y = 0$ into~\eqref{2012a5-eq0} yields $-f(x) = f(x) f(0)$ for all $x \in R$.
This means that we have
\[ f \equiv 0 \text{ or } f(0) = -1. \tag{1.1}\label{2012a5-eq1-1} \]
From now on, we assume that $f(0) = -1$.

Plugging $y = -1$ into~\eqref{2012a5-eq0} yields
\[ f(1 - x) - f(x - 1) = f(x) f(-1) \quad \text{for all } x \in R. \tag{1.2}\label{2012a5-eq1-2} \]
This also implies
\[ f(-x) - f(x) = f(x + 1) f(-1) \quad \text{for all } x \in R. \tag{1.3}\label{2012a5-eq1-3} \]
In particular, the case $f(-1) = 0$ yields that $f$ is even.









\subsection*{Step 2: Ring quotients}

We construct an ideal $I$ of $R$ such that $f$ is constant on the cosets of $I$.
We define the two ideals
\[ J = \{c \in R : \forall x \in R, f(cx + 1) = 0\}, \tag{2.1}\label{2012a5-eq2-1} \]
\[ I = \{c \in R : \forall x \in R, f(c + x) = f(x)\}. \tag{2.2}\label{2012a5-eq2-2} \]

\begin{lemma}\label{2012a5-2-1}
For any $c \in R$, we have $c \in J$ if and only if $f(c + x) = -f(c) f(x)$ for all $x \in R$.
The set $J$ is an ideal of $R$.
\end{lemma}
\begin{proof}
The first part is clear since $f(cx + 1) - f(c + x) = f(c) f(x)$.
For the second part, closure under addition is easy to prove from the equation $f(c + x) = -f(c) f(x)$.
Meanwhile, closure under multiplication in $R$ follows from the equation $f(cx + 1) = 0$.
Finally, $f(1) = 0$ yields $0 \in J$.
\end{proof}

\begin{lemma}\label{2012a5-2-2}
For any $c \in R$, we have $c \in I$ if and only if $c \in J$ and $f(c) = -1$.
The set $I$ is an ideal of $R$.
\end{lemma}
\begin{proof}
If $c \in I$, then $f(c) = f(0) = -1$, and $c \in J$ holds by the previous lemma.
The converse is also easy to check.
Now we prove that $I$ is an ideal of $R$.
It is easily a subgroup of $R$, so we just need to prove closure under multiplication over $R$.

Let $c \in I$ and $d \in R$.
Since $f(c + y) = f(y)$ for all $y \in R$,~\eqref{2012a5-eq0} yields $f(d(c + x) + 1) = f(dx + 1)$ for all $x \in R$.
Since $J$ is an ideal containing $I$, we have $cx \in J$.
That means we have $-f(dc) f(dx + 1) = f(dx + 1)$ for all $x \in R$.
Fixing $x$, this implies either $f(dc) = -1$ or $f(dx + 1) = 0$ for all $y \in R$.
Note that $dc \in J$ since $c \in I \subseteq J$.
Thus, we get either $dc \in I$ or $d \in J$.
That is, if $d \notin J$, then $dc \in I$.

Finally, take an arbitrary $c \in I$ and $d \in R$.
If $d \notin J$, then $dc \in I$ as we've seen above.
If $d \in J$, then $d - 1, 1 \notin J$, so $(d - 1)c, c \in I$ and thus $dc \in I$.
\end{proof}

For any $a, b \in R$ such that $a - b \in I$, it is clear that $f(a) = f(b)$.
Thus, we get an induced map $g : R/I \to S$ such that $f = g \circ \phi$, where $\phi : R \to R/I$ is the quotient map.
Since $\phi$ is surjective, one can check that $g$ satisfies~\eqref{2012a5-eq0} as well.
Thus, from now on, we can WLOG assume that $I = \{0\}$ whenever necessary.

\begin{lemma}\label{2012a5-2-3}
Suppose that $I \subsetneq J$, and fix some $c \in J \setminus I$.
Then $R/I = \{[0], [1], [c], [c + 1]\}$, where the bracket denotes quotient mod $I$.
\end{lemma}
\begin{proof}
First note that $|J/I| = 2$.
Indeed, consider any $c, d \in J \setminus I$.
By Lemma~\ref{2012a5-2-1}, it is easy to check that $f(c - d) = f(0) = -1$.
Thus, by Lemma~\ref{2012a5-2-2}, $c - d \in I$, so $[c] = [d]$.

Now fix the unique element $c \in J \setminus I$.
We claim that if $d \notin J$, then $dc \notin I$.
This suffices to prove the lemma as follows.
If $d \notin J$, we have $dc, c \in J \setminus I$.
The previous paragraph yields $(d - 1) c = dc - c \in I$.
Now the claim implies $d - 1 \in J$.
As a result, we have either $d \in J$ or $d - 1 \in J$.
Since $J/I = \{[0], [c]\}$, this yields $[d] \in \{[0], [1], [c], [c + 1]\}$.
This would prove the lemma.

We now prove the claim.
Fix $d \notin J$; our goal is to show that $dc \notin I$.
Since $d \notin J$, we can choose $x \in R$ such that $f(dx + 1) \neq 0$.
Since $c \in J \setminus I$, we have $f(c) = 1$ and thus
\[ f(d(c + x) + 1) = f(d + c + x) + f(d) f(c + x) = -(f(d + x) + f(d) f(x)) = -f(dx + 1). \]
On the other hand, since $J$ is an ideal, we have $dc \in J$.
This means $f(d(c + x) + 1) = -f(dc) f(dx + 1)$, and thus either $f(dc) = 1$ or $f(dx + 1) = 0$.
By the choice of $x$, the former must hold, so $f(dc) = 1$.

To show that $dc \notin I$, we still need to show that $-1 \neq 1$ in $S$.
However, $-1 = 1$ in $S$ would imply $f(c) = 1 = -1$, which implies $c \in I$ by Lemma~\ref{2012a5-2-2} since $c \in J$.
A contradiction!
\end{proof}









\subsection*{Step 3: Case 1: $f(-1) \neq 0$}

The results displayed here are relevant up to Step 5.

Here,~\eqref{2012a5-eq1-3} yields
\[ f(-x + 1) = -f(x + 1) \quad \text{for all } x \in R. \tag{3.1}\label{2012a5-eq3-1} \]
In particular, since $f(0) = -1$, we have
\[ f(2) = 1. \tag{3.2}\label{2012a5-eq3-2} \]
Plugging~\eqref{2012a5-eq3-1} back into~\eqref{2012a5-eq1-2} yields
\[ f(x + 1) + f(x - 1) = -f(x) f(-1) \quad \text{for all } x \in R. \tag{3.3}\label{2012a5-eq3-3} \]
Plugging $x = 2$ into~\eqref{2012a5-eq0} and using~\eqref{2012a5-eq3-1} yields
\[ f(2y + 1) = f(y) - f(-y). \tag{3.4}\label{2012a5-eq3-4} \]
Then~\eqref{2012a5-eq1-3} again yields
\[ f(2y + 1) = -f(y + 1) f(-1). \tag{3.5}\label{2012a5-eq3-5} \]
This is sufficient to compute $f(-1)$.

\begin{claim}
We have \[ f(-1) \in \{-2, 1\}. \]
\end{claim}
\begin{proof}
Let $C = -f(-1)$.
Note that $f(3) = C$ by~\eqref{2012a5-eq3-1}.
The claim is equivalent to $C \in \{-1, 2\}$.

By~\eqref{2012a5-eq3-5}, $f(5) = C^2$.
On the other hand, recall that $f(2) = 1$ from~\eqref{2012a5-eq3-2}.
Thus, by~\eqref{2012a5-eq3-3}, we have $f(4) = C^2 - 1$ and then $f(5) = C (C^2 - 2)$.
That is, we get $C^2 = C(C^2 - 2)$.
Since $C \neq 0$ by the case assumption, we get
\[ C^2 - C - 2 = 0 \iff C \in \{-1, 2\}, \]
    as desired.
\end{proof}

Before continuing with the subcases, we have some more useful observations.
Compare the~\eqref{2012a5-eq0} with the same equality but with $(x, y)$ replaced by $(-x, -y)$.
Subtracting yields
\[ f(x + y) - f(-x - y) = f(-x) f(-y) - f(x) f(y). \tag{3.6}\label{2012a5-eq3-6} \]
Applying~\eqref{2012a5-eq1-3} yields
\[ -f(x + y + 1) f(-1) = f(-x) f(-y) - f(x) f(y). \tag{3.7}\label{2012a5-eq3-7} \]
Plugging $x = y$ and applying~\eqref{2012a5-eq3-5} again into the above equation yields
\[ f(x + 1) f(-1)^2 = f(-x)^2 - f(x)^2 = (f(-x) - f(x)) (f(-x) + f(x)). \]
By~\eqref{2012a5-eq1-3}, we get
\[ f(x + 1) f(-1)^2 = f(x + 1) f(-1) (f(-x) + f(x)). \]
Since $f(-1) \neq 0$, we get that for any $x \in R$,
\[ f(x + 1) = 0 \text{ implies } f(-x) + f(x) = f(-1). \tag{3.8}\label{2012a5-eq3-8} \]

Finally, before we move on to the two subcases, we prove that for any $x \in R$,
\[ f(x + 1) = 0 \text{ implies } f(-x) = f(x) = -1. \tag{3.9}\label{2012a5-eq3-9} \]
Suppose that $f(x + 1) = 0$.
Plugging $y = x + 1$ into~\eqref{2012a5-eq3-6} gives us
\[ f(2x + 1) - f(-2x - 1) = f(-x) f(-x - 1). \]
By~\eqref{2012a5-eq3-5}, we get
\[ f(2x + 1) - f(-2x - 1) = -f(x + 1) f(-1) + f(-x) f(-1) = f(-x) f(-1). \]
On the other hand, plugging $y = -x - 1$ into~\eqref{2012a5-eq3-6} (or~\eqref{2012a5-eq3-7}) gives us
\[ f(-1) = -f(x) f(-x - 1). \]
Since $f(x + 1) = 0$,~\eqref{2012a5-eq1-3} yields $f(-x) = f(x)$, so
\[ f(-1) = f(-x) f(-x - 1) = f(-x) f(-1). \]
Finally, since $f(-1) \neq 0$, this yields $f(-x) = 1$.
Since we already have $f(-x) = f(x)$, we have proved~\eqref{2012a5-eq3-9}.

We now split into two subcases.









\subsection*{Step 4: Subcase 1.1: $f(-1) = -2 \neq 0$}

Here, we consider the subcase where $f(-1) = -2 \neq 0$.
Note that this also means that $\rchar(S) \neq 2$.
We claim that $f + 1$ is a ring homomorphism from $R$ to $S$.
That is, $f + 1$ is both additive and multiplicative.
This corresponds to the class 2 solutions.

Combining~\eqref{2012a5-eq3-8} and~\eqref{2012a5-eq3-9} yields
\[ f(-x) + f(x) = -2 \quad \text{for all } x \in R. \tag{4.1}\label{2012a5-eq4-1} \]
On the otehr hand~\eqref{2012a5-eq1-3} gives us $f(-x) - f(x) = -2 f(x + 1)$ for all $x \in R$.
Subtracting the second equation from the first equation yields $2 f(x) = 2 f(x + 1) - 2$.
Since $2 \neq 0$ in $S$, this gives us
\[ f(x + 1) = f(x) + 1. \tag{4.2}\label{2012a5-eq4-2} \]

We first show that $f + 1$ is additive.
That is, $f(x + y) = f(x) + f(y) + 1$ for all $x, y \in R$.
Indeed, recall~\eqref{2012a5-eq3-7}:
\[ 2 f(x + y + 1) = -f(x + y + 1) f(-1) = f(-x) f(-y) - f(x) f(y). \]
Applying~\eqref{2012a5-eq4-2} on the left hand side and~\eqref{2012a5-eq4-1} on the right hand side gives
\[ 2 (f(x + y) + 1) = (-f(x) - 2)(f(y) - 2) - f(x) f(y) = 2 (f(x) + f(y) + 2). \]
Again, $2 \neq 0$ in $S$, so we get $f(x + y) + 1 = f(x) + f(y) + 2$ for any $x, y \in R$.

Finally, we give a quick proof that $f + 1$ is multiplicative.
We just go back to~\eqref{2012a5-eq0}.
For any $x, y \in R$,
\begin{align*}
    f(xy) + 1
    &= f(xy + 1) \\
    &= f(x) f(y) + f(x + y) \\
    &= f(x) f(y) + f(x) + f(y) + 1 \\
    &= (f(x) + 1)(f(y) + 1).
\end{align*}









\subsection*{Step 5: Subcase 1.2: $f(-1) = 1 \neq -2$}

Here, we consider the subcase where $f(-1) = 1 \neq -2$.
This also means that $\rchar(S) \neq 3$.
The solution in this subcase corresponds to class 5 solutions.
That is, we claim that $f$ factors through a ring homomorphism $R \to \F_3$.

Recall the definition of the set $I$ as in~\eqref{2012a5-eq2-2}:
\[ I = \{c \in R : \forall x \in R, f(c + x) = f(x)\}. \]
By our work in Step 2 (mainly Lemma~\ref{2012a5-2-2}), we can assume WLOG that $I = 0$.
With this assumption, it suffices to show that $R \cong \F_3$.
Note that the values $f(-1) = 1$, $f(0) = -1$, and $f(1) = 0$ are already known in this subcase.

First we prove that for any $c \in R$,
\[ f(c + 1) = 0 \text{ implies } c \in I \text{ implies } c = 0. \tag{5.1}\label{2012a5-eq5-1} \]
Indeed,~\eqref{2012a5-eq3-9} yields $f(-c) = f(c) = -1$, and thus plugging $x = c$ into~\eqref{2012a5-eq3-7} yields
\[ -f(c + y + 1) = -f(c + y + 1) f(-1) = -(f(-y) - f(y)) \quad \text{for all } y \in R. \]
But~\eqref{2012a5-eq1-3} yields $f(-y) - f(y) = f(y + 1) f(-1) = f(y + 1)$.
Replacing $y$ with $x - 1$ yields $f(c + x) = f(x)$ for all $x \in R$, i.e., $c \in I$.

Now we go back to the main goal.
In this step,~\eqref{2012a5-eq3-3} reads as
\[ f(x + 1) + f(x - 1) = -f(x) \iff f(x - 1) + f(x) + f(x + 1) = 0. \tag{5.2}\label{2012a5-eq5-2} \]
In particular, it is easy to see that $f(x + 2) = f(x - 1)$ for any $x \in R$.
Thus $3 \in I$, and so $\rchar(R) = 3$; note that $R$ cannot be trivial since $f(0) = -1 \neq 0 = f(1)$.

Finally, we show that $x \in \{-1, 0, 1\}$ for any $x \in R$.
Note that by~\eqref{2012a5-eq5-1},~\eqref{2012a5-eq3-8} now reads as
\[ x \neq 0 \text{ implies } f(-x) + f(x) = 1. \tag{5.3}\label{2012a5-eq5-3} \]
Suppose for the sake of contradiction that $x \notin \{-1, 0, 1\}$.
Then $x, x + 1, x - 1 \neq 0$, so we get
\[ f(-x) + f(x) = f(-(x + 1)) + f(x + 1) = f(-(x - 1)) + f(x - 1) = 1. \]
Summing all three equations and using~\eqref{2012a5-eq5-2} yields $3 = 0$.
A contradiction, since $\rchar(S) \neq 3$.









\subsection*{Step 6: Case 2: $f(-1) = 0$}

This covers the rest of the solution, which consists of four subcases.
The Subcase 2.3 needs to steal some results from Subcase 2.1.
The details will come later.

As before, we collect important equations here and do an important computation.
We computed $f(-1)$ in the previous case; now we will compute $f(2)$ instead.

Note that~\eqref{2012a5-eq1-3} only yields that $f$ is even, i.e.,
\[ f(-x) = f(x) \quad \text{for all } x \in R. \tag{6.1}\label{2012a5-eq6-1} \]
Replacing $y$ with $-y$ in~\eqref{2012a5-eq0} now gives
\[ f(xy - 1) - f(x - y) = f(x) f(y) \quad \text{for all } x, y \in R. \tag{6.2}\label{2012a5-eq6-2} \]
Plugging $y = x$ into the above equation yields
\[ f(x^2 - 1) = f(x)^2 - 1 \quad \text{for all } x \in R. \tag{6.3}\label{2012a5-eq6-3} \]
Since $f(-1) = 0$, plugging $y = x + 1$ into~\eqref{2012a5-eq6-2} gives
\[ f(x(x + 1) - 1) = f(x) f(x + 1) \quad \text{for all } y \in R. \tag{6.4}\label{2012a5-eq6-4} \]

Next, consider~\eqref{2012a5-eq0} with $y = 2$.
We apply it twice on the equation $f(2x + 1) = f(-2x - 1)$.
Together with the fact that $f$ is even, we get
\[ f(x + 2) + f(2) f(x) = f(x - 1) + f(2) f(x + 1). \]
This can be rewritten as
\[ f(x + 2) = f(2) (f(x + 1) - f(x)) + f(x - 1) \quad \text{for all } x \in R. \tag{6.5}\label{2012a5-eq6-5} \]

\begin{claim}
We have \[ f(2) \in \{-1, 0, 1, 3\}. \]
\end{claim}
\begin{proof}
For convenience, let $C = f(2)$.
We start with $f(3) = C^2 - 1$ from~\eqref{2012a5-eq6-3}.
Next we have $f(4) = f(2) (f(3) - f(2)) = C (C^2 - 1 - C)$ from~\eqref{2012a5-eq6-5}.
Finally, we compute $f(5)$ in two ways, one using~\eqref{2012a5-eq6-4} and one using~\eqref{2012a5-eq6-5}:
\[ f(5) = f(2) f(3) = C(C^2 - 1), \]
\[ f(5) = f(2) (f(4) - f(3)) + f(2) = C(C(C^2 - 1 - C) - (C^2 - 1)) + C. \]
That is, we have
\[ C^2 + C(C^2 - 1 - C) = C(C(C^2 - 1 - C) - (C^2 - 1)) + C, \]
    which rearranges to
\[ C(C - 1)(C - 3)(C + 1) = 0. \]
This proves the claim.
\end{proof}

Before bashing the four subcases, we prove an equality based on the polynomial identity
\[ x(x + 1) - 1 = (x - 1)(x + 2) + 1 = x^2 + x - 1. \]
Plugging $(x, y) \mapsto (x - 1, x + 2)$ yields
\[ f(x) f(x + 1) = f(x^2 + x - 1) = f(x - 1) f(x + 2) + f(2x + 1) \quad \text{for all } x \in R. \]
Since $f(1) = 0$, applying~\eqref{2012a5-eq0} with $y = 2$ and applying~\eqref{2012a5-eq6-4} yields
\[ f(x) f(x + 1) - f(x - 1) f(x + 2) = f(2) f(x) + f(x + 2) \quad \text{for all } x \in R. \tag{6.6}\label{2012a5-eq6-6} \]









\subsection*{Step 7: Subcase 2.1: $f(2) = 0 \neq 3$}

Here, we consider the subcase where $f(2) = 0 \neq 3$.
This also means that $\rchar(S) \neq 3$.
The solution in this subcase corresponds to class 6 solutions.
That is, we claim that $f$ factors through a ring homomorphism $R \to \F_3$.
Note we already know that $f(0) = -1$ and $f(-1) = f(1) = 0$.

Recall the definition of the set $I$ as in~\eqref{2012a5-eq2-2}:
\[ I = \{c \in R : \forall x \in R, f(c + x) = f(x)\}. \]
As in Step 5, we can assume WLOG that $I = 0$.
From~\eqref{2012a5-eq6-5}, we get $3 \in I \iff 3 = 0$ in $R$.
Plugging the value of $f(2)$ into~\eqref{2012a5-eq6-6} gives
\[ f(x) f(x + 1) - f(x - 1)^2 = f(x - 1) \quad \text{for all } x \in R. \tag{7.1}\label{2012a5-eq7-1} \]
Replacing $x$ with $x + 1$ and using the fact that $3 = 0$ in $R$, we get
\[ f(x + 1) f(x - 1) - f(x)^2 = f(x). \]
Subtracting the two equalities and factoring gives
\[ f(x) = f(x - 1) \text{ or } f(x - 1) + f(x) + f(x + 1) = -1 \quad \text{for all } x \in R. \tag{7.2}\label{2012a5-eq7-2} \]

We show that
\[ f(c + 1) = f(c - 1) = 0 \text{ implies } c = 0 \quad \text{for all } c \in R. \tag{7.3}\label{2012a5-eq7-3} \]
It suffices to show that $f(c + x) = f(x)$ for all $y \in R$.
Indeed, assume that $f(c + 1) = f(c - 1) = 0$.
By~\eqref{2012a5-eq0}, we get the equations
\[ f((c + 1) x + (c - 1)) = f((c^2 - 1) x + 1) = f((c - 1) x + (c + 1)), \]
    which can be rewritten as
\[ f((c + 1)(x + 1) + 1) = f((c - 1)(x + 1) - 1) \]
    since $2 = -1$ in $R$.
Now apply~\eqref{2012a5-eq6-2} on the left hand side and~\eqref{2012a5-eq0} again on the right hand side.
We get $f(x + c + 2) = f(x - c + 2)$ for any $x \in R$.
Now replace $x$ with $x - 2 - c$ and use the fact that $2 = -1 \in R$ again to finish.
This proves~\eqref{2012a5-eq7-3}.

Now we can show that the second equality in~\eqref{2012a5-eq7-2} must always hold.
That is, we have
\[ f(x - 1) + f(x) + f(x + 1) = -1 \quad \text{for all } x \in R. \tag{7.4}\label{2012a5-eq7-4} \]
Suppose for the sake of contradiction that this does not hold.
Then~\eqref{2012a5-eq7-2} tells us that $f(x) = f(x + 1) = f(x - 1)$.
Plugging into~\eqref{2012a5-eq7-1} gives us $f(x) = f(x + 1) = f(x - 1) = 0$.
Then~\eqref{2012a5-eq7-3} yields $x = 0$; but $f(0) = -1 \neq 0$, a contradiction.
This proves~\eqref{2012a5-eq7-4}.

Next, we show that
\[ f(x) \in \{0, -1\} \quad \text{for all } x \in R. \tag{7.5}\label{2012a5-eq7-5} \]
By~\eqref{2012a5-eq0},~\eqref{2012a5-eq7-1}, and the fact that $2 = -1$ in $R$, we get
\[ f(x^2) = f(x + 1) f(x - 1) + f(2x) = f(x)^2 + f(x) + f(-x) = f(x)^2 + 2 f(x) \]
    and
\[ f(x^2 + 1) = f(x)^2 + f(2x) = f(x)^2 + f(x). \]
Now we add the two equalities and also add with~\eqref{2012a5-eq6-3} to get
\[ f(x^2) + f(x^2 + 1) + f(x^2 - 1) = 3 f(x)^2 + 3 f(x) - 1. \]
Using~\eqref{2012a5-eq7-4} gives us $3 f(x)^2 + 3 f(x) = 0$.
Together with $\rchar(S) \neq 3$, this proves~\eqref{2012a5-eq7-5}.

Finally, we show that $R \cong \F_3$, i.e., $x \in \{-1, 0, 1\}$ for any $x \in R$.
By~\eqref{2012a5-eq7-5}, each of $f(x)$, $f(x + 1)$, and $f(x - 1)$ takes the value $0$ or $-1$.
By~\eqref{2012a5-eq7-4}, at least one of the three expressions must be equal to $-1$.
Thus it now suffices to show that
\[ f(x) = -1 \text{ implies } x = 0 \quad \text{for all } x \in R. \tag{7.6}\label{2012a5-eq7-6} \]

\begin{proof}
Fix $x \in R$ and suppose that $f(x) = -1$.
By~\eqref{2012a5-eq7-3}, it suffices to show that $f(x + 1) = f(x - 1) = 0$.
Since $f(x) = -1$,~\eqref{2012a5-eq7-4} yields $f(x + 1) + f(x - 1) = 0$.
On the other hand,~\eqref{2012a5-eq7-1} with $x$ replaced by $x + 1$ yields $f(x + 1) f(x - 1) = 0$.
The two equations yields $f(x + 1) = f(x - 1) = 0$, as desired.
\end{proof}









\subsection*{Step 8: Subcase 2.2: $f(2) = 1 \neq -1$}

Here, we consider the subcase where $f(2) = 1 \neq -1$.
This also means that $\rchar(S) \neq 2$.
The solution in this subcase corresponds to class 7 solutions.
That is, we claim that $f$ factors through a ring homomorphism $R \to \Z/4\Z$.
Note we already know that $f(0) = -1$, $f(-1) = f(1) = 0$, and $f(2) = 1$.

Recall from~\eqref{2012a5-eq2-1} that
\[ J = \{c \in R : \forall x \in R, f(cx + 1) = 0\}. \]
Clearly, $2 \neq 0$ in $R$ since $f(2) = 1 \neq -1 = f(0)$.
By Lemma~\ref{2012a5-2-3}, it suffices to show that $2 \in J$.
That is, $f(2x + 1) = 0$ for any $x \in R$.

Since $f(2) = 1$,~\eqref{2012a5-eq6-5} yields
\[ f(x + 2) + f(x) = f(x + 1) + f(x - 1) \quad \text{for all } x \in R. \tag{8.1}\label{2012a5-eq8-1} \]
It is easy to see from the above equation that $4 \in I$.
Now fix any $x \in R$.
By~\eqref{2012a5-eq0},~\eqref{2012a5-eq8-1} implies $f(2x + 1) = f(2x - 1)$.
Now~\eqref{2012a5-eq6-2} with $(x, y)$ replaced by $(2, 2x + 1)$ implies
\[ f(4x + 1) = f(2x + 1) + f(2x - 1) = 2 f(2x + 1). \]
Note that $f(4x + 1) = 0$ since $4 \in I$.
Thus, we also get $f(2x + 1) = 0$ since $\rchar(S) \neq 2$.
This shows that $2 \in J$.









\subsection*{Step 9: Subcase 2.3: $f(2) = 3 \neq 1$}

Here, we consider the subcase where $f(2) = 3 \neq 1$.
This also means that $\rchar(S) \neq 2$.
The solution in this subcase corresponds to class 3 solutions.
That is, we claim that there exists a ring homomorphism $\phi : R \to S$ such that $f = \phi^2 - 1$.

The solution in this subsection requires at least that $2$ is invertible in $S$.
For convenience, we just assume that $S$ is a field.
We will still begin with only assuming that $S$ is an integral domain.
Note that we will borrow~\eqref{2012a5-eq7-4} to cover the case $\rchar(S) = 3$ in the first important statement.

In this subcase,~\eqref{2012a5-eq6-5} reads as
\[ f(x + 2) = 3 (f(x + 1) - f(x)) + f(x - 1) \quad \text{for all } x \in R. \tag{9.1}\label{2012a5-eq9-1} \]
Applying the above equality to~\eqref{2012a5-eq6-6} yields
\[ f(x) (3 f(x - 1) + f(x + 1)) - (f(x - 1) + 3 f(x + 1)) (1 + f (x - 1)) = 0 \quad \text{for all } x \in R. \tag{9.2}\label{2012a5-eq9-2} \]
Replacing $x$ with $-x$ and using~\eqref{2012a5-eq6-1} yields
\[ f(x) (3 f(x + 1) + f(x - 1)) - (f(x + 1) + 3 f(x - 1)) (1 + f (x + 1)) = 0 \quad \text{for all } x \in R. \]
Subtracting~\eqref{2012a5-eq9-2} from this equation yields
\[ f(x + 1) + f(x - 1) = 2 f(x) + 2 \text{ or } f(x + 1) = f(x - 1) \quad \text{for all } x \in R. \tag{9.3}\label{2012a5-eq9-3} \]

Next, we prove that
\[ f(x + 1) + f(x - 1) = 2 f(x) + 2 \text{ or } f(x + 1) = f(x - 1) = 0 \quad \text{for all } x \in R. \tag{9.4}\label{2012a5-eq9-4} \]
Suppose that the former does not hold for some $x \in R$.
By~\eqref{2012a5-eq9-3}, we have $f(x + 1) = f(x - 1)$.
Plugging into~\eqref{2012a5-eq9-2} yields
\[ f(x + 1) = f(x - 1) = 0 \text{ or } f(x + 1) = f(x - 1) = f(x) - 1. \]
By~\eqref{2012a5-eq9-1}, the latter yields $f(x + 2) = f(x) - 4$.
By manual computation and $\rchar(S) \neq 2$, this contradicts~\eqref{2012a5-eq9-3}.
So, we have $f(x + 1) = f(x - 1) = 0$, as desired.

Now note that~\eqref{2012a5-eq9-1} yields $f(x + 1) + f(x - 1) - 2 f(x) = f(x + 2) + f(x) - 2 f(x + 1)$.
Thus,~\eqref{2012a5-eq9-4} yields a stronger result:
\[ f(x + 1) + f(x - 1) = 2 f(x) + 2 \text{ or } f(x + 1) = f(x) = f(x - 1) = 0 \quad \text{for all } x \in R. \tag{9.5}\label{2012a5-eq9-5} \]
Finally, we show that the latter yields a contradiction.
For the case $\rchar(S) = 3$, we have $f(2) = 3 = 0$.
We can then borrow~\eqref{2012a5-eq7-4} from Step 7, which does not require $f(2) \neq 3$.
It directly yields a contradiction.
It remains to consider the case $\rchar(S) \neq 3$.

By~\eqref{2012a5-eq6-2} and~\eqref{2012a5-eq0}, we have $f(2x + 1) = f(2x - 1) = 0$.
So~\eqref{2012a5-eq9-5} with $\rchar(S) \neq 2$ tells us that $f(2x) \in \{-1, 0\}$.
On the other hand, we will show that $f(2x) = -3$, which is a contradiction since $\rchar(S) \neq 2, 3$.
Indeed, by~\eqref{2012a5-eq6-2} and~\eqref{2012a5-eq0},
\[ f(x^2 - 2) = f(x + 1) f(x - 1) + f(2) = 3, \]
\[ f(x^2 - 1) = f(x)^2 - 1 = -1, \]
\[ f(x^2) = f(x + 1) f(x - 1) + f(2x) = f(2x), \]
\[ f(x^2 + 1) = f(x)^2 + f(2x) = f(2x), \]
    so~\eqref{2012a5-eq9-1} implies
\[ f(2x) = 3 (f(2x) + 1) + 3 \iff 2 (f(2x) + 3) = 0 \iff f(2x) = -3, \]
    since $\rchar(S) \neq 2$.
This proves
\[ f(x + 1) + f(x - 1) = 2 f(x) + 2 \quad \text{for all } x \in R. \tag{9.6}\label{2012a5-eq9-6} \]
For the rest of the subsection, we assume that $S$ is a field.

Consider the function $g : R \to S$ defined by
\[ g(x) = \frac{f(x) - f(x - 1) + 1}{2}. \]
Since $f(1) = 0$ and $f(0) = -1$, we have
\[ g(0) = 0. \tag{9.g1}\label{2012a5-eq9-g1} \]
By~\eqref{2012a5-eq9-6}, we have
\[ g(x + 1) = g(x) + 1 \quad \text{for all } x \in R, \tag{9.g2}\label{2012a5-eq9-g2} \]
By~\eqref{2012a5-eq9-6} and~\eqref{2012a5-eq6-1}, we have
\[ g(-x) = -g(x) \quad \text{for all } x \in R, \tag{9.g3}\label{2012a5-eq9-g3} \]
We claim that $g$ is the desired homomorphism.

We first prove that
\[ f(x) = g(x)^2 - 1 \quad \text{for all } x \in R. \tag{9.g4}\label{2012a5-eq9-g4} \]
We plug~\eqref{2012a5-eq9-6} to~\eqref{2012a5-eq9-2}.
After some rearranging, we get
\[ (f(x) - f(x - 1) - 1)^2 + 4 = 4 f(x), \]
    which implies~\eqref{2012a5-eq9-g4} after division by $4 = 2^2 \neq 0 \in S$.

Now, by~\eqref{2012a5-eq9-g2},~\eqref{2012a5-eq0} becomes
\[ (g(xy) + 1)^2 - g(x + y)^2 = (g(x)^2 - 1)(g(y)^2 - 1) \quad \text{for all } x, y \in R. \tag{9.g5}\label{2012a5-eq9-g5} \]
On the other hand,~\eqref{2012a5-eq6-2} becomes
\[ (g(xy) - 1)^2 - g(x - y)^2 = (g(x)^2 - 1)(g(y)^2 - 1) \quad \text{for all } x, y \in R. \tag{9.g6}\label{2012a5-eq9-g6} \]
Subtracting the two equalities and using~\eqref{2012a5-eq9-g2} gives
\[ 4 g(xy) = g(x + y)^2 - g(x - y)^2 \quad \text{for all } x, y \in R. \tag{9.g7}\label{2012a5-eq9-g7} \]
Now adding~\eqref{2012a5-eq9-g5} and~\eqref{2012a5-eq9-g6} gives
\[ 2 g(xy)^2 + 2 - (g(x + y)^2 + g(x - y)^2) = 2 (g(x)^2 - 1)(g(y)^2 - 1). \]
Multiplying by $2^3 = 8$ and using~\eqref{2012a5-eq9-g7} gives
\[ (g(x + y)^2 - g(x - y)^2)^2 + 16 = 8(g(x + y)^2 + g(x - y)^2) + 16 (g(x)^2 - 1)(g(y)^2 - 1). \tag{9.g8}\label{2012a5-eq9-g8} \]

Consider the following identity, which holds due to~\eqref{2012a5-eq9-g2}:
\[ g(x + 1)^2 + g(x - 1)^2 = 2 g(x)^2 + 2 \quad \text{for all } x \in R. \]
We use this identity as follows.
Consider~\eqref{2012a5-eq9-g8} with $y$ replaced by $y + 1$ and also by $y - 1$.
Add the two equations, and then subtract with twice of~\eqref{2012a5-eq9-g8}.
Due to the above identity, after some heavy rearrangement and simplification, we get
\[ 8 (g(x + y) + g(x - y))^2 = 32 g(x)^2. \]
That is, we get
\[ g(x + y) + g(x - y) = \pm 2 g(x) \quad \text{for all } x, y \in R. \]

We now remove the sign constrain and prove
\[ g(x + y) + g(x - y) = 2 g(x) \quad \text{for all } x, y \in R. \tag{9.g9}\label{2012a5-eq9-g9} \]
Suppose for the sake of contradiction that the above equality does not hold for some $x, y \in R$.
Then we have
\[ g(x + y) + g(x - y) = -2 g(x) \text{ and } g(x + y + 1) + g(x - y + 1) = -2 g(x + 1), \]
    since $g(x + y + 1) + g(x - y + 1) = 2 g(x + 1)$ implies~\eqref{2012a5-eq9-g9}.
The two equalities yield $4 = 0$ in $S$; a contradiction.

Now~\eqref{2012a5-eq9-g3} also yields $g(x + y) - g(x - y) = 2 g(y)$.
Thus,~\eqref{2012a5-eq9-g7} becomes $4 g(xy) = 4 g(x) g(y)$ for all $x, y \in R$.
Since $\rchar(S) \neq 2$, we get that
\[ g(xy) = g(x) g(y) \quad \text{for all } x, y \in R. \tag{9.g10}\label{2012a5-eq9-g10} \]

Finally, we prove that $g$ is additive.
Plugging $x = 2$ yields $g(2y) = 2g(y)$ for all $y \in R$ since $g(2) = 2$.
Now replace $x$ and $y$ in~\eqref{2012a5-eq9-g9} with $x + y$ and $x - y$, respectively.
We get $2 g(x) + 2 g(y) = 2 g(x + y)$, and thus
\[ g(x) + g(y) = g(x + y) \quad \text{for all } x, y \in R \]
    since $\rchar(S) \neq 2$.
Thus $g$ is indeed a ring homomorphism.
The formula~\eqref{2012a5-eq9-4} now proves that $f$ can be written as $g^2 - 1$, where $g : R \to S$ is a ring homomorphism.









\subsection*{Step 10: Subcase 2.4: $f(2) = -1$}

Finally, we consider the subcase where $f(2) = -1$.
The solution in this subcase corresponds to four class of solutions: 2, 4, 8, 9.
More precisely, class 2 solutions occur when $\rchar(S) = 2$, so that $1 = -1$ in $S$.
The other three classes occur in the case $\rchar(S) \neq 2$.

Again, recall the definition of the set $I$ as in~\eqref{2012a5-eq2-2}:
\[ I = \{c \in R : \forall x \in R, f(c + x) = f(x)\}. \]
As in Step 5, we can assume WLOG that $I = 0$.
Our first step here is proving that
\[ 2 \in I \iff f(x + 1) = f(x - 1) \quad \text{for all } x \in R. \]

The equation~\eqref{2012a5-eq6-5} reads as
\[ f(x + 2) = f(x) - f(x + 1) + f(x - 1). \]
Using the above equation,~\eqref{2012a5-eq6-6} reads as
\[ f(x) f(x + 1) - f(x - 1) (f(x) - f(x + 1) + f(x - 1)) = f(x - 1) - f(x + 1). \]
Rearranging yields
\[ (f(x) + f(x - 1))(f(x + 1) - f(x - 1)) = -(f(x + 1) - f(x - 1)). \]
Thus, we have
\[ f(x + 1) = f(x - 1) \text{ or } f(x) + f(x - 1) = -1 \quad \text{for all } x \in R. \]

Now we are ready to prove that $2 \in I$.
Fix $x \in R$.
Replace $x$ with $-x$ in the above equation.
Using the fact that $f$ is even (that is,~\eqref{2012a5-eq6-1}), we get
\[ f(x + 1) = f(x - 1) \text{ or } f(x) + f(x + 1) = -1. \]
If $f(x + 1) \neq f(x - 1)$, then we get $f(x) + f(x - 1) = f(x) + f(x + 1) = -1$.
This still leads to $f(x + 1) = f(x - 1)$.
This proves that $2 \in I$.
From now on, we may assume that $2 = 0$ in $R$.

The two equations~\eqref{2012a5-eq6-4} and~\eqref{2012a5-eq6-3}, respectively, becomes
\[ f(x(x + 1) + 1) = f(x) f(x + 1) \quad \text{for all } x \in R. \tag{10.1}\label{2012a5-eq10-1} \]
\[ f(x^2 + 1) = f(x)^2 - 1 \quad \text{for all } x \in R, \tag{10.2}\label{2012a5-eq10-2} \]
The latter yields
\[ f(x^2) = f(x + 1)^2 - 1  \quad \text{for all } x \in R, \tag{10.3}\label{2012a5-eq10-3} \]
Next, we prove the following lemma.

\begin{lemma}\label{2012a5-10-1}
For any $x \in R$, either $f(x)^2 + f(x + 1)^2 = 1$ or $f(x) + f(x + 1) = 1$ holds.
\end{lemma}
\begin{proof}
We start by proving the following equality statement:
\[ (f(x + 1)^2 - 1)(f(x + 1) - 1) + f(x) f(x + 1) = f(x) f(x^2 + x) \quad \text{for all } x \in R. \tag{10.L1.1}\label{2012a5-eq10-lem1-1} \]
The equality is based on factoring $x^2 (x + 1)$ in two ways, as $x^2 \cdot (x + 1)$ and $x \cdot x(x + 1)$.
For any $x \in R$,~\eqref{2012a5-eq0} yields
\[ f(x^2) f(x + 1) + f(x^2 + x + 1) = f(x^2 + x) f(x) + f(x^2), \]
    as both sides are equal to $f(x^2 (x + 1) + 1)$.
By~\eqref{2012a5-eq10-1},~\eqref{2012a5-eq10-3}, and some algebraic manipulation,~\eqref{2012a5-eq10-lem1-1} follows.

Now fix $x$ and let $a = f(x)$ and $b = f(x + 1)$.
Then~\eqref{2012a5-eq10-lem1-1} gives us
\[ (b^2 - 1)(b - 1) + ab = a f(x^2 + x), \]
\[ (a^2 - 1)(a - 1) + ab = b f(x^2 + x). \]
Eliminating terms containing $f(x^2 + x)$ gives us
\[ b((b^2 - 1)(b - 1) + ab) = a((a^2 - 1)(a - 1) + ab). \]
After some factoring, this is equivalent to
\[ (b - a)(a^2 + b^2 - 1)(a + b - 1) = 0. \]
That is, we have either $f(x)^2 + f(x + 1)^2 = 1$, $f(x) + f(x + 1) = 1$.
It remains to show that the $f(x) = f(x + 1)$ has to yield $f(x) + f(x + 1) = 1$.

Now suppose for the sake of contradiction that Lemma~\ref{2012a5-10-1} does not hold.
Then we have $a = b$, and~\eqref{2012a5-eq10-lem1-1} yields
\[ a f(x^2 + x) = (a^2 - 1)(a - 1) + a^2 = a^3 - a + 1. \]
By~\eqref{2012a5-eq10-1},~\eqref{2012a5-eq10-3}, and~\eqref{2012a5-eq10-2},
\[ f(x^4 + x^2 + 1) = f(x^2) f(x^2 + 1) = (a^2 - 1)^2. \]
On the other hand, by~\eqref{2012a5-eq10-2} directly,
\[ f(x^4 + x^2 + 1) = f(x^2 + x)^2 - 1. \]
As a result, we have $(a^2 - 1)^2 = f(x^2 + x)^2 - 1$, and thus
\[ a^2 (a^2 - 1)^2 = (a^3 - a + 1)^2 - a^2, \]
    which is equivalent to $(1 - 2a)(a^2 - 1) = 0$.
So we have either $1 = 2a = a + b$, which still implies Lemma~\ref{2012a5-10-1}, or $a = \pm 1$.

It remains to show that $a = b = \pm 1$ is impossible.
Indeed, it yields $f(x^2) = f(x^2 + 1) = a^2 - 1 = 0$.
This contradicts~\eqref{2012a5-eq10-lem1-1} with $x$ replaced by $x^2$.
\end{proof}

We are now ready to finish for the case $\rchar(S) = 2$.
We prove that $f + 1$ is a ring homomorphism.

\begin{claim}
If $\rchar(S) = 2$, then $f + 1 : R \to S$ is a ring homomorphism.
\end{claim}
\begin{proof}
For the case $\rchar(S) = 2$, Lemma~\ref{2012a5-10-1} simplifies to
\[ f(x + 1) = f(x) + 1 \quad \text{for all } x \in R. \tag{10.L2.1}\label{2012a5-eq10-lem2-1} \]
This means that~\eqref{2012a5-eq0} can be written as
\[ f(xy) = f(x) f(y) + f(x + y) + 1 \quad \text{for all } x, y \in R. \tag{10.L2.2}\label{2012a5-eq10-lem2-2} \]
First suppose that $f + 1$ is additive, i.e.,
\[ f(x + y) = f(x) + f(y) + 1 \quad \text{for all } x, y \in R. \]
Then~\eqref{2012a5-eq10-lem2-2} yields
\[ f(xy) + 1 = (f(x) + 1)(f(y) + 1) \quad \text{for all } x, y \in R, \]
    and thus $f + 1$ is both additive and multiplicative.
Thus, we only need to prove that $f + 1$ is additive.

We write $f(xy(x + 1)(y + 1) + 1)$ in two ways using~\eqref{2012a5-eq10-lem2-2}.
We get
\[ f(xy) f((x + 1)(y + 1)) + f(x + y + 1) = f(x(y + 1)) f(y(x + 1)) + f(x + y). \]
Using~\eqref{2012a5-eq10-lem2-1}, we get
\[ f(xy) f((x + 1) (y + 1)) + 1 = f(x (y + 1)) f(y(x + 1)). \]
For convenience, set $a = f(x)$, $b = f(y)$, and $c = f(x + y)$.
Then, using~\eqref{2012a5-eq10-lem2-2} repeatedly gives
\[ (ab + c)((a + 1)(b + 1) + c) + 1 = (a(b + 1) + c + 1)(b(a + 1) + (c + 1)), \]
    which is equivalent to $c = a + b + 1$ after some heavy algebraic manipulation.
This is the equation we wanted to prove, so we are done.
\end{proof}



The rest of the solution is going to focus on the case $\rchar(S) \neq 2$.
The main goal is to prove that $f(c) = -1$ implies $c = 0$.
We have useful statements along that way that also helps solving the subcase after proving the main goal.

\begin{lemma}\label{2012a5-10-2}
For any $x \in R$, we have either $f(x) f(x + 1) = 0$, or $f(x) + f(x + 1) = 1$ and $f(x) f(x + 1) = -1$.
\end{lemma}
\begin{proof}
We start by rewriting $f(x^4)$ using~\eqref{2012a5-eq10-2} and~\eqref{2012a5-eq10-3}:
\[ f(x^4) = f(x)^2 (f(x)^2 - 2) = f(x)^4 - 2 f(x)^2. \]
Now we write $f(x^8 + x^4 + 1)$ in two ways, using the above equality and~\eqref{2012a5-eq10-1} in different order.
For convenience, denote $a = f(x)$ and $b = f(x + 1)$.
Then we have
\[ f(x^8 + x^4 + 1) = f(x^4) f(x^4 + 1) = (a^4 - 2a^2)(b^4 - 2b^2), \]
\[ f(x^8 + x^4 + 1) = f(x^2 + x + 1)^4 - f(x^2 + x + 1)^2 = a^4 b^4 - 2 a^2 b^2. \]
Thus, we have
\[ (a^4 - 2a^2)(b^4 - 2b^2) = a^4 b^4 - 2 a^2 b^2 \iff 2 (ab)^2 (a^2 + b^2 - 3) = 0. \]
Since $2 \neq 0$ in $S$, this reduces to either $ab = 0$ or $a^2 + b^2 = 3$.

Now we assume the latter.
Lemma~\ref{2012a5-10-1} tells us that either $a^2 + b^2 = 1$ or $a + b = 1$ holds.
Since $a^2 + b^2 = 3 \neq 1$, we get $a + b = 1$, and
\[ 2ab = (a + b)^2 - (a^2 + b^2) = -2. \]
Since $\rchar(S) \neq 2$, we get $ab = -1$, as desired.
\end{proof}

\begin{lemma}\label{2012a5-10-3}
For any $c \in R$, $f(c) = -1$ implies $c = 0$.
\end{lemma}
\begin{proof}
First, by Lemma~\ref{2012a5-10-2}, we have
\[ f(x) = -1 \text{ implies } f(x + 1) = 0 \quad \text{for all } x \in R. \tag{10.L3.1}\label{2012a5-eq10-lem3-1} \]
Another important observation is the following statement:
\[ \text{if } f(c) = -1, \text{ then } f(c^2 + cx + 1) = -f(cx + 1) \quad \text{for all } x \in R, \tag{10.L3.2}\label{2012a5-eq10-lem3-2} \]
    which follows from applying~\eqref{2012a5-eq0} twice on both sides.

Now fix some $c \in R$ with $f(c) = -1$.
By~\eqref{2012a5-eq10-lem3-1} and~\eqref{2012a5-eq10-3}, we get
\[ f(c + 1) = 0 \implies f(c^2) = 0^2 - 1 = -1. \]
Now, apply~\eqref{2012a5-eq10-lem3-2} twice to get
\[ f(c^4 + c^2 + c^2 x + 1) = f(c^2 x + 1) \quad \text{for all } x \in R. \]
By~\eqref{2012a5-eq10-lem3-1} and~\eqref{2012a5-eq10-1}, we have
\[ f(c^2 + c + 1) = f(c) f(c + 1) = 0. \]
Then by~\eqref{2012a5-eq10-3}, we get $f(c^4 + c^2) = -1$.
Thus, the above equation yields
\[ f((c^6 + c^4) x + (c^4 + c^2) + 1) = 0 \quad \text{for all } x \in R. \]
Using the above equation again yields $c^6 + c^4 \in J$.

Notice that $c^6 + c^4 = (c^2 + c^3 + 1)^2 - 1$.
But we also have
\[ f(c^3 + c^2 + 1) = f(c^2) f(c + 1) + f(c^2 + c + 1) = 0. \]
Thus $f(c^6 + c^4) = -1$.
By Lemma~\ref{2012a5-2-2}, we have $c^6 + c^4 \in I$, so $c^6 + c^4 = 0$.

Now we reduce from $c^6 + c^4 = 0$ to $c = 0$.
By~\eqref{2012a5-eq10-lem3-2} and the fact that $\rchar(S) \neq 2$, it suffices to show that $c^2 = 0$.
Repeating the same argument again, it suffices to show that $c^4 = 0$.

Consider that for any $x \in R$,
\[ f((c^2 + 1)(c^4 x + 1) + 1) = f(c^2 + 1 + 1) = f(c^2) = -1. \]
On the other hand,~\eqref{2012a5-eq10-2} yields $f(c^2 + 1) = 0$, and~\eqref{2012a5-eq0} then yields
\[ f((c^2 + 1)(c^4 x + 1) + 1) = f((c^2 + 1) + (c^4 x + 1)) = f(c^4 x + c^2). \]
Thus $f(c^4 x + c^2) = -1$.
Applying~\eqref{2012a5-eq10-lem3-1}, we get
\[ f(c^4 x + c^2 + 1) = 0 \quad \text{for all } x \in R. \]
Applying~\eqref{2012a5-eq10-lem3-2}, we get $f(c^4 x + 1) = 0$ for any $x \in R$, so $c^4 \in J$.
By Lemma~\ref{2012a5-2-2}, it remains to show that $f(c^4) = -1$.
Since $f(c^2) = -1$, this follows by~\eqref{2012a5-eq10-lem3-1} and~\eqref{2012a5-eq10-3}.
\end{proof}

\begin{lemma}\label{2012a5-10-4}
For any $x \in R$, we have either $x^2 = 0$, $(x + 1)^2 = 0$, or $f(x) + f(x + 1) = 1$ and $x^2 + x + 1 = 0$.
\end{lemma}
\begin{proof}

If $ab = 0$, then either $f(x) = 0$ or $f(x + 1) = 0$.
By~\eqref{2012a5-eq10-2} and Lemma~\ref{2012a5-10-3}, they yield $(x + 1)^2 = 1$ and $x^2 = 0$, respectively.

Now suppose that $a^2 + b^2 = 3$.
Lemma~\ref{2012a5-10-1} yields $a + b = 1$.
Thus $2ab = 1^2 - 3 = -2$.
Since $\rchar(S) \neq 2$, this yields $f(x) f(x + 1) = ab = -1$.
By~\eqref{2012a5-eq10-1} and Lemma~\ref{2012a5-10-3}, we get $x^2 + x + 1 = 0$.
\end{proof}

Now we just consider the above lemma and split into several subsubcases.
The three subsubcases yield three distinct solution classes.

First suppose that there exists $c \in R$ such that $c \neq 0$ and $c^2 = 0$.
By~\eqref{2012a5-eq10-3}, we have
\[ -1 = f(0) = f(c^2 x^2) = f(cx + 1)^2 - 1 \quad \text{for all } x \in R. \]
In particular, $f(cx + 1) = 0$ for all $x \in R$, i.e., $c \in J$.
Since $f(c^2 + 1) = f(1) = 0$,~\eqref{2012a5-eq10-2} yield $f(c) = -1$.
But $f(c) = -1$ implies $c = 0$, so this necessarily implies $f(c) = 1$.
Lemma~\ref{2012a5-2-3} yields $R \cong \F_2[X]/\langle X^2 \rangle$ with $c \mapsto X$.
Also, we have computed $f(0) = -1$, $f(1) = 0$, $f(c) = 1$, and $f(c + 1) = 0$.

Now suppose that in $R$, $x^2 = 0$ implies $x = 0$.
Then, $x^2 = 1$ implies $(x - 1)^2 = 0$ and thus $x = 1$.
That is, Lemma~\ref{2012a5-10-4} simplifies to
\[ x = 0 \text{ or } x = 1 \text{ or } x^2 + x + 1 = 0 \quad \text{for all } x \in R. \]
If the case $x^2 + x + 1 = 0$ never occurs, then clearly $R \cong \F_2$ and we are done.
It remains to consider the case where $f(c) + f(c + 1) = 1$ and $c^2 + c + 1 = 0$ for some $c \in R$.

Fix such $c$, and consider an arbitrary $d \in R$ such that $d^2 + d + 1 = 0$.
Then $c + d \in \{0, 1\}$, since
\[ (c + d)^2 + (c + d) + 1 = (c^2 + c + 1) + (d^2 + d + 1) + 1 = 1 \neq 0. \]
That is, we have $d \in \{c, c + 1\}$.
In summary, any $x \in R$ satisfies $x \in \{0, 1, c, c + 1\}$.
This yields $R \cong \F_4$, and it remains to check that $f(c) + f(c + 1) = 1$ and $f(c) f(c + 1) = -1$.
The former is given, and the latter follows by~\eqref{2012a5-eq10-1}.









\end{document}
