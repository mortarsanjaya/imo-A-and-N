\documentclass{article}

\usepackage{amsmath, amsfonts, amssymb, amsthm}
\usepackage{hyperref}

\setlength{\parindent}{0pt}
\setlength{\parskip}{5pt}

\newcommand{\Z}{\mathbb{Z}}

\title{IMO 2012 N3}
\author{}
\date{}

\begin{document}

\maketitle



\subsection*{Problem}

Determine all integers $m > 1$ such that, $n \mid \binom{n}{m - 2n}$ for every $n \leq m/2$ non-negative.



\subsection*{Answer}

Prime numbers.



\subsection*{Solution}

Official solution: \url{https://www.imo-official.org/problems/IMO2012SL.pdf}

The following solution we present is the official solution.

First, suppose that $m$ is prime.
Clearly, $0$ divides $\binom{0}{m} = 0$ since $m > 0$.
So, it remains to consider the case where $n$ is positive.
For $m = 2$, the only remaining case to consider is $n = 1$, for which the divisibility is trivial.
So now, suppose that $m$ is odd; in particular, $m - 2n > 0$ for any $n \leq m/2$.
Since $m$ is prime and $0 < n \leq m/2 < m$, we have $\gcd(n, m - 2n) = 1$.
On the other hand, we have the identity $(m - 2n) \binom{n}{m - 2n} = n \binom{n - 1}{p - 2n - 1}$.
Then $\gcd(n, m - 2n) = 1$ implies $n \mid \binom{n}{m - 2n}$, as desired.

Conversely, suppose that $n$ divides $\binom{n}{m - 2n}$ whenever $n \leq m/2$.
Write $m = pr$, where $p$ is the smallest prime divisor of $m$ and $r \geq 1$.
If $p = 2$, then choosing $n = r$ yields $r \mid \binom{r}{0} = 1$, so $r = 1$ and thus $m = 2$.
So now, suppose that $p > 2$.
Then $m$ is odd, and so is $r$, so we can write $r = 2k + 1$ for some $k \geq 0$.

Now, picking $n = pk < m/2$, we have $pk \mid \binom{pk}{p}$.
Suppose for the sake of contradiction that $k > 0$.
Consider the equation
\[ p! \binom{pk}{p} = pk \prod_{j = 1}^{p - 1} (p(k - 1) + j) \implies (p - 1)! \binom{pk}{p} = k \prod_{j = p(k - 1) + 1}^{p(k - 1) + (p - 1)} j. \]
The right hand side is not divisible by $pk$, so $pk \nmid \binom{pk}{p}$.
A contradiction, so $k = 0$, and thus $m = pr = p(2k + 1) = p$; $m$ is a prime.



\subsection*{Extra notes}

The original problem has a more restricted inequality, $m/3 \leq n \leq m/2$.
We remove the inequality $m/3 \leq n$ since the only thing that this inequality does is making sure that $\binom{n}{m - 2n}$ is a positive integer.
Note that, by convention, we define $\binom{a}{b} = 0$ if $0 \leq a < b$, and also we define $0 \mid 0$.



\end{document}
