\documentclass{article}

\usepackage{amsmath, amsfonts, amssymb, amsthm}
\usepackage{hyperref}

\setlength{\parindent}{0pt}
\setlength{\parskip}{5pt}

\title{IMO 2014 N2}
\author{}
\date{}

\begin{document}

\maketitle



\subsection*{Problem}

Prove that, for any positive integer $n > 1$, there exists infinitely many positive integer $m$ such that $\left\lfloor \dfrac{n^m}{m} \right\rfloor$ is odd.



\subsection*{Solution}

Official solution: \url{https://www.imo-official.org/problems/IMO2014SL.pdf}

The presented solution below is Solution 2 of the official solution, with a difference in one case.
When $n > 2$ and $n$ is even, instead of taking $k$ to be the powers of a fixed prime factor of $n - 1$, we just use powers of $n - 1$.

Since $n > 1$, we have $k < n^k$ for any $k \geq 0$.
So, if $n$ is odd, $\dfrac{n^{n^k}}{n^k} = n^{n^k - k}$ is odd for any $k \geq 0$.
It remains to consider the case where $n$ is even.

First, suppose that $n > 2$ is even.
We take $m = (n - 1)^k$, where $k > 0$.
One can prove by induction on $k$ that, for any positive integer $m$ and $k \geq 0$, $m^k \mid (m + 1)^{m^k} - 1$.
Using $m = n - 1$ in this case gives us $n^{(n - 1)^k} \equiv 1 \pmod{(n - 1)^k}$.
Thus, if we denote $q = \left\lfloor \dfrac{n^{(n - 1)^k}}{(n - 1)^k} \right\rfloor$, we have $n^{(n - 1)^k} = q (n - 1)^k + 1$ since $q > 1$.
But $n^{(n - 1)^k}$ is even, while $(n - 1)^k$ is odd, so $q$ must be odd.

Finally, suppose that $n = 2$.
This time, take $m = 3 \cdot 4^k$ where $k > 0$, and let $q = \left\lfloor \dfrac{2^{3 \cdot 4^k}}{3 \cdot 4^k} \right\rfloor = \left\lfloor \dfrac{2^{3 \cdot 4^k - 2^k}}{3} \right\rfloor$.
Following the same argument as in the previous paragraph, it suffices to show that $2^{3 \cdot 4^k - 2k} \equiv 1 \pmod{3}$.
This is easy since $3 \cdot 4^k - 2k$ is even and positive for $k > 0$.



\subsubsection*{Implementation details}

The following fact is used in two separate cases in the above solution:
    for any positive integers $m$ and $n$, if $m$ is even and the remainder of $m$ divided by $n$ is exactly $1$, then $n$ is odd.
Thus, we implement this result as a separate lemma.



\end{document}
