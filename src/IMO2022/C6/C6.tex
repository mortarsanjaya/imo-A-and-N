\documentclass{article}

\usepackage{amsmath, amsfonts, amssymb, amsthm}
\usepackage{hyperref}

\setlength{\parindent}{0pt}
\setlength{\parskip}{5pt}

\newcommand{\N}{\mathbb{N}}

\newtheorem*{claim}{Claim}

\title{IMO 2022 C6}
\author{}
\date{}

\begin{document}

\maketitle



\subsection*{Problem}

In a single-player game, initially a multiset $S_0$ of natural numbers is given on a board.
At any time, given a multiset $S$ of natural numbers on the board, we may perform one of the following moves:

\begin{itemize}

    \item
    Take two elements $a, b \in S$ so that $\{a, b\} \leq S$ as a multiset.
    Pick some $c \in \N$ with $c \leq \min\{a, b\}$.
    Then remove $\{a, b\}$ from $S$ and add $\{2c, a - c, b - c\}$ into $S$.

    \item
    Remove any occurrence of $0$ in $S$.

\end{itemize}

Given $S_0$, find the smallest possible size of the multiset on the board at any point.



\subsection*{Answer}

For any multiset $S$ of natural numbers, denote by $\sigma_S$ the sum of all elements of $S$, including multiplicity.
The answer is:

\begin{itemize}

    \item 
    $0$, if $S_0$ only consists of $0$.

    \item
    $1$, if $\sigma_{S_0} = 2^k m$ with $m$ odd and $m$ divides all elements of $S_0$,

    \item
    $2$, otherwise.

\end{itemize}



\subsection*{Solution}

Official solution: \url{http://www.imo-official.org/problems/IMO2022SL.pdf}

Solution 1 only describes the case where $S_0$ consists of just ones.
However, the remark on starting configurations that follows after the solution solves this more general problem.
We will adapt Solution 1 but on this more general setting.
Given multisets $A$ and $B$ of natural numbers, we denote $A \to B$ if $B$ is obtained from one move on $A$.
We denote $A \implies B$ is $B$ can be obtained from $A$ through any (possibly empty) sequence of moves.
In this case, we also say that $A$ reaches $B$.

Clearly the answer is $0$ if $S_0$ only consists of $0$.
For example, induction on $|S_0|$ works.
From now on, we assume that $S_0$ contains a positive integer.

Using the same notation $\sigma_S$, one can observe that $A \to B$ implies $\sigma_A = \sigma_B$.
By induction, $A \implies B$ implies $\sigma_A = \sigma_B$.
If $S_0$ contains a positive integer, then at any time, the multiset $S$ on the board satisfies $\sigma_S = \sigma_{S_0} > 0$.
In particular, $|S| = 0$ is impossible, so the answer is at least $1$.

Now consider the second case.
Write $\sigma_{S_0} = 2^k m$ where $m$ is odd, and suppose that $m$ divides every element of $S_0$.
One can prove that (with the pointwise multiplication notation) for any multisets $A$ and $B$, $A \to B$ implies $mA \to mB$.
In particular, $A \implies B$ implies $mA \implies mB$.
Thus, dividing by $m$, it suffices to consider the case where $\sigma_{S_0} = 2^k$, i.e., $m = 1$.

For convenience, denote by $R_a(b)$ the multiset consisting of $a$ copies of $b$.
Consider the following general observations.

\begin{itemize}

    \item
    For any multiset $A, B, X$ of natural numbers, $A \to B$ implies $A + X \to B + X$.
    By induction, $A \implies B$ implies $A + X \implies B + X$.
    In particular, given multisets $A, B, C, D$, $A \implies B$ and $C \implies D$ implies $A + C \implies B + D$.

    \item
    For any $n \in \N$, we have $\{2, n + 1\} \to \{2, 1, n\}$.
    Indeed, this is a legal move of the first type, with $a = c = 1$ and $b = n$.
    By induction, one can show that $A \implies \{2\} + R_{\sigma_A - 2}(1)$ for any multiset $A$ with $\sigma_A \geq 2$.
    Here the number of ones in the second multiset is $\sigma_A - 2$.

\end{itemize}

Now, for $k = 0$, $\sigma_{S_0} = 2^k = 1$ implies $S_0 = \{1\} + R_a(0)$ for some $a \in \N$.
Removing all occurrences of zero leaves us with a singleton, $\{1\}$.
Now suppose that $k \geq 1$.
By the above observation, $S_0 \implies \{2\} + R_{2^k - 2}(1)$.
Since $\{1, 1\} \to \{2, 0\} \to \{2\}$, one can induct to show that $R_{2n}(k) \implies R_n(2k)$ for any $n, k \in \N$.
In particular, we get $S_0 \implies R_{2^{k - 1}}(2)$.
Inducting again allows us to prove $R_{2^{k - 1}}(2) \implies R_1(2^k) = \{2^k\}$, and we are done.

For the remaining case, an odd integer $m$ divides $\sigma_{S_0}$ but it does not divide at least one element of $S_0$.
The following claim implies that $S_0$ cannot reach a singleton.

\begin{claim}
Fix an odd integer $m$ and multisets $A$ and $B$ such that $A \to B$.
If some element of $A$ is not divisible by $m$, then some element of $B$ is not divisible by $m$.
\end{claim}

Indeed, by induction, the claim follows if we replace $A \to B$ with $A \implies B$.
If $S_0$ reaches a singleton $\{N\}$, then $m \mid N = \sigma_{S_0}$.
Yet the claim implies $m \nmid N$; a contradiction.

\begin{proof}
Arguing by contrapositive, suppose that $m$ divides every element of $B$.
If $A \to B$ via the second type move, then $A = \{0\} + B$ and the goal is proved.
If $A \to B$ via the first type move, then $A = \{a + c, b + c\} + X$ and $B = \{a, b, 2c\} + X$ for some $a, b, c \in \N$.
Every element of $X$ is divisible by $m$, and $a, b, 2c$ are all divisible by $m$.
Since $m$ is odd, we also have $m \mid c$, and thus $m$ divides both $a + c$ and $b + c$.
\end{proof}

It now remains to show that any multiset can reach a multiset of size at most $2$.
Using one of the above observation, it suffices to prove this result when the multiset is of form $\{2\} + R_n(1)$ for some $n \in \N$.

Let $k$ be the largest non-negative integer such that $2^k \leq n + 2$; note that $k \geq 1$.
By the case $\sigma_{S_0} = 2^k$, if $n + 2 = 2^k$, then $\{2\} + R_n(1) \implies \{2^k\}$.
If $n + 2 = 2^k + 1$, then $\{2\} + R_n(1) \implies \{2^k, 1\}$; both has size at most $2$.
Finally, consider the case $n \geq 2^k$, and write $n = 2^k + m$ for some $0 \leq m < 2^k - 2$.
Then $\{2\} + R_n(1) \implies \{2^k, 2\} + R_m(1)$.
By one of the above observations and induction, we have $\{a, 2\} \implies \{b, 2\} + R_{a - b}(1)$ for any $a \geq b \geq 0$.
In particular, $\{2^k, 2\} + R_m(1) \implies \{m + 2, 2\} + R_{2^k - 2}(1) \implies \{m + 2, 2^k\}$.
We are done.



\subsection*{Extra notes}

In correspondence to the original formulation, each number $k$ in $S$ represents a pile of $k$ pebbles.
If one literally translates the original formulation, the only legal move is the first option listed above.
The problem changes to finding the smallest possible number of non-zero elements of $S$, up to multiplicity.
This seems to be more cumbersome than just allowing to remove any zeroes from $S$.



\subsection*{Implementation details}

We split into two files.
The file \texttt{C6_basic.lean} focuses on the general properties of a multiset obtainable from another multiset via series of moves.
The file \texttt{C6_main.lean} focuses on the problem itself; the minimal possible size of $A$ such that $S_0 \implies A$, where $S_0$ is fixed.



\end{document}
