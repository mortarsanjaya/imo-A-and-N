\documentclass{article}

\usepackage{amsmath, amsfonts, amssymb, amsthm}
\usepackage{hyperref}

\setlength{\parindent}{0pt}
\setlength{\parskip}{5pt}

\title{IMO 2022 A1}
\author{}
\date{}

\begin{document}

\maketitle



\subsection*{Problem}

Let $R$ be a totally ordered ring.
Let $(a_n)_{n \geq 0}$ be a sequence of non-negative elements of $R$ such that $a_{n + 1}^2 + a_n a_{n + 2} \leq a_n + a_{n + 2}$ for all $n \geq 0$.
Show that $a_{N + 2} \leq 1$ for all $N \geq 0$.



\subsection*{Solution}

Official solution: \url{http://www.imo-official.org/problems/IMO2022SL.pdf}

We use an alternative way to prove that two consecutive terms without $a_0$ cannot be greater than $1$.

First consider any $b, c > 1$ and $a \geq 0$ in $R$ such that $b^2 + ac \leq a + c$.
Rearranging the equations yield $b^2 - 1 \leq (1 - a)(c - 1)$.
Since $b, c > 1$, this gives us $2(b - 1) < b^2 - 1 \leq (1 - a)(c - 1) \leq c - 1$.

Now consider $a, b, c, d \geq 0$ in $R$ such that $b^2 + ac \leq a + c$ and $c^2 + bd \leq b + d$.
If $b, c > 1$, the previous paragraph yields $2(b - 1) < c - 1$ and $2(c - 1) < b - 1$.
Then $4(b - 1) < b - 1$; a contradiction.
In particular, this means that in the sequence $(a_n)_{n \geq 0}$, $a_{n + 1} > 1$ implies $a_{n + 2} \leq 1$.

Finally, suppose for the sake of contradiction that $a_{N + 2} > 1$ for some $n > 0$.
Again, we get $0 < a_{N + 2}^2 - 1 \leq (1 - a_{N + 1})(a_{N + 3} - 1)$.
This forces either $a_{N + 1} > 1$ or $a_{N + 3} > 1$.
But both yields contradiction from the previous paragraph.



\end{document}
