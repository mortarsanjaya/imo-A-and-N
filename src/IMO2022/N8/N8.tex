\documentclass{article}

\usepackage{amsmath, amsfonts, amssymb, amsthm}
\usepackage{hyperref}

\setlength{\parindent}{0pt}
\setlength{\parskip}{5pt}

\newcommand{\N}{\mathbb{N}}
\newcommand{\Z}{\mathbb{Z}}

\title{IMO 2022 N8}
\author{}
\date{}

\begin{document}

\maketitle



\subsection*{Problem}

Given $n \in \N$ such that $2^n + 65 \mid 5^n - 3^n$, prove that $n = 0$.



\subsection*{Solution}

Official solution: \url{http://www.imo-official.org/problems/IMO2022SL.pdf}

We present a rather quick Solution 2 of the official solution.
We can even afford to generalize $65$ to any number congruent to $5$ modulo $60$.
Here, fix any positive integer $N \equiv 5 \pmod{60}$.

Clearly, $n = 1$ does not work as $N + 2 > 2 = 5 - 3$.
If $n > 0$ is even, then $3$ divides $2^n + N$ but not $5^n - 3^n$.
Finally, if $n > 1$ is odd, then
\[ \left(\frac{5}{2^n + N}\right) = \left(\frac{5^n}{2^n + N}\right) = \left(\frac{3^n}{2^n + N}\right) = \left(\frac{3}{2^n + N}\right). \]
By quadratic reciprocity, Jacobi symbol version,
\[ \left(\frac{2^n + N}{5}\right) = \left(\frac{2^n + N}{3}\right). \]
Using the fact that $n$ is odd again and $5 \mid N$, the left hand side equals $\left(\frac{2}{5}\right) = -1$.
Meanwhile, $2^n \equiv N \equiv 2 \pmod{3}$, so the right hand side equals $\left(\frac{1}{3}\right) = 1$.
Contradiction.



\end{document}
