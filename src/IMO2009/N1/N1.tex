\documentclass{article}

\usepackage{amsmath, amsfonts, amssymb, amsthm}
\usepackage{hyperref}

\setlength{\parindent}{0pt}
\setlength{\parskip}{5pt}

\newcommand{\Z}{\mathbb{Z}}

\newtheorem*{claim}{Claim}

\title{IMO 2009 N1 (P1)}
\author{}
\date{}

\begin{document}

\maketitle



\subsection*{Problem}

Let $n$ be a positive integer and $k \geq 1$.
Let $a_1, a_2, \ldots, a_{k + 1}$ be distinct integers in the set $\{1, 2, \ldots, n\}$ such that $n \mid a_i (a_{i + 1} - 1)$ for every $i \leq k$.
Prove that $n \nmid a_{k + 1} (a_1 - 1)$.



\subsection*{Solution}

Official solution: \url{https://www.imo-official.org/problems/IMO2009SL.pdf}

We over-generalize Solution 3 of the official solution by the following claim.

\begin{claim}
Let $M$ be a commutative monoid, written multiplicatively.
Let $k \geq 1$ and $a_1, a_2, \ldots, a_k$ be elements of $M$.
Suppose that $a_i a_{i + 1} = a_i$ for each $i \leq k$, where we denote $a_{k + 1} = a_1$.
Then $a_i = a_j$ for each $i, j \leq k$.
\end{claim}
\begin{proof}
We extend $a_1, a_2, \ldots, a_k$ by defining $a_{k + i} = a_i$ for all $i \geq 1$.
It is easy to see by induction on $m$ that $a_i a_{i + 1} a_{i + 2} \ldots a_{i + m} = a_i$ for all $m \geq 0$.
Taking $m = k - 1$ gives us $a_1 a_2 \ldots a_k = a_i$ for all $i \leq k$.
This shows that $a_i = a_j$ for each $i, j \leq k$.
\end{proof}

Now, suppose for the sake of contradiction that $n \mid a_{k + 1} (a_1 - 1)$.
Using the claim with $M = \Z/n\Z$, we get that the $a_i$s are equal modulo $n$.
But they belong in $\{1, 2, \ldots, n\}$, so they must be all equal.
In particular, since $k + 1 \geq 2$, we have $a_1 = a_2$; a contradiction.



\subsection*{Implementation details}

Clearly, we will implement the above claim.
We will also implement a version of the statement in general commutative rings and then in $\Z$ with divisibility.
In commutative rings, we replace $a_i a_{i + 1} = a_i$ with $a_i (a_{i + 1} - 1) = 0$.
In $\Z$, we replace $a_i (a_{i + 1} - 1) = 0$ with $n \mid a_i (a_{i + 1} - 1)$.



\end{document}
