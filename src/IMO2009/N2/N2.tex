\documentclass{article}

\usepackage{amsmath, amsfonts, amssymb, amsthm}
\usepackage{hyperref}

\setlength{\parindent}{0pt}
\setlength{\parskip}{5pt}

\newcommand{\N}{\mathbb{N}}
\newcommand{\Z}{\mathbb{Z}}

\title{IMO 2009 N2}
\author{}
\date{}

\begin{document}

\maketitle



\subsection*{Problem}

For each positive integer $n$, let $\Omega(n)$ denote the number of prime factors of $n$, counting multiplicity.
For convenience, we denote $\Omega(0) = 0$.
\begin{enumerate}
    \item Prove that for any $N \geq 0$, there exists $a, b \geq 0$ distinct such that $\Omega((a + k)(b + k))$ is even for all $0 \leq k < N$.
    \item Prove that for any $a, b \geq 0$, if $\Omega((a + k)(b + k))$ is even for all $k \geq 0$, then $a = b$.
\end{enumerate}



\subsection*{Solution}

Official solution: \url{https://www.imo-official.org/problems/IMO2009SL.pdf}

The following solution is a rewrite of the official solution.


\subsubsection*{Part 1}

For convenience, denote $[N] = \{0, 1, \ldots, N - 1\}$ and $S = (\Z/2\Z)^[N]$.
Define the map $f : \N \to S$ by $f(a, k) = \overline{\Omega(a + k)}$ for all $a \in \N$ and $k \in S$.
The overbar denotes reduction modulo $2$.
Since $S$ is finite, there must exist $a \neq b$ such that $f(a) = f(b)$.
This implies that $\Omega((a + k)(b + k))$ is even for all $k \in [N]$, as desired.


\subsubsection*{Part 2}

Without loss of generality, let $a \leq b$.
Write $b = a + c$, $c \in \N$, and suppose for the sake of contradiction that $c > 0$.
For any $m \geq 0$, taking $k = a(c - 1) + mc$, we get that $\Omega((a + m)(a + m + 1) c^2)$ is even.
Since $c > 0$, this implies that $\Omega(k(k + 1))$ is even for all $k \geq a$.
Using reduction modulo $2$, by small induction on $k$, we get $\overline{\Omega(k)} = \overline{\Omega(a)}$ for all $k \geq a$ if $a > 0$.
Note that this also holds if $a = 0$.

Finally, let $p$ be a prime greater than $a$.
The previous result gives us $0 = \overline{\Omega(p^2)} = \overline{\Omega(p)} = 1$; a contradiction.




\end{document}
