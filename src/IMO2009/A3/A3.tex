\documentclass{article}

\usepackage{amsmath, amsfonts, amssymb, amsthm}
\usepackage{hyperref}

\setlength{\parindent}{0pt}
\setlength{\parskip}{5pt}

\newcommand{\N}{\mathbb{N}}

\title{IMO 2009 A3 (P5)}
\author{}
\date{}

\begin{document}

\maketitle



\subsection*{Problem}

Find all functions $f : \N \to \N$ such that for any $x, y \in \N$, the non-negative integers $x$, $f(y)$, and $f(y + f(x))$ form the sides of a possibly degenerate triangle.

\emph{Note:}
Three non-negative real numbers $a$, $b$, and $c$ forms a possibly degenerate triangle if and only if the sum of two of them is greater than the third real number.
Equivalently, $a + b + c \geq 2 \max\{a, b, c\}$.



\subsection*{Answer}

$f(x) = x$.



\subsection*{Solution}

Official solution: \url{https://www.imo-official.org/problems/IMO2009SL.pdf}

We present the official solution, modified for our version of the problem.
That is, we show that $f(0) = 0$, $f(f(x)) = x$ for all $x \in \N$, and then $f(x) = x$ for all $x \in \N$.
However, we do things differently for the final step.
We prove $f(1) = 1$ and then use induction to prove that $f(x) = x$ for all $x \in \N$.

First, using $x = 0$, we get that $f(y) = f(y + f(0))$ for all $y \in \N$.
If $f(0) > 0$, this means that $f$ is periodic, and thus bounded, say by $N$.
Plugging $x = 2N + 1$ gives a contradiction, as then $x > 2N \geq f(y) + f(y + f(x))$.
This shows that $f(0) = 0$.
Now $f(f(x)) = x$ is immediate by setting $y = 0$.
In particular, $f$ is bijective.
Plugging $x = 1$ yields that for all $y \in \N$,
\[ f(y + f(1)) \leq f(y) + 1. \tag{1}\label{2009a3-eq1} \]

We prove by induction that $f(n f(1)) = n$ for all $n \in \N$.
Indeed, we have $f(0) = 0$ and $f(f(1)) = 1$.
In particular, $f(1) \neq 0$.
Now suppose that for some $k \geq 1$, we have $f(n f(1)) = n$ for all $n \leq k$.
By~\eqref{2009a3-eq1}, $f((k + 1) f(1)) \leq f(k f(1)) + 1 = k + 1$.
If $f((k + 1) f(1)) = c < k + 1$, then $c \leq k$, so $c = f(c f(1))$.
By injectivity, then $(k + 1) f(1) = c f(1)$ and then $c = k + 1$, since $f(1) \neq 0$.
This shows that $f((k + 1) f(1)) = k + 1$.

As a result, $f(f(1)^2) = f(1) \implies f(1)^2 = 1 \implies f(1) = 1$.
We can now prove by induction on $n$ that $f(n) = n$ for all $n \geq 0$.
Again, we only need to care about the induction step.
suppose that for some $k \geq 1$, we have $f(n) = n$ for all $n \leq k$.
By~\eqref{2009a3-eq1}, $f(k + 1) \leq f(k) + 1 = k + 1$.
If $f(k + 1) < k + 1$, then $f(k + 1) \leq k$.
So, by induction hypothesis, $k + 1 = f(f(k + 1)) = f(k + 1)$.
We are done.



\subsection*{Extra notes}

The main formulation considers a function $f$ from $\N^+$ to itself such that for any $x, y \in \N^+$, $x$, $f(y)$, and $f(y + f(x) - 1)$ form the sides of a non-degenerate triangle.
It is not hard to show that our formulation is equivalent.



\subsection*{Implementation details}

We use $\N$/\texttt{nat} as opposed to $\N^+$/\texttt{pnat} for the main formulation.
The main reason is a better Lean API for \texttt{nat} to implement the above solution.
However, we will still implement the \texttt{pnat} version of the problem.
This is done through the bijection $f \iff (n \mapsto f(n + 1) - 1)$ between $\N \to \N$ and $\N^+ \to \N^+$.



\end{document}
