\documentclass{article}

\usepackage{amsmath, amsfonts, amssymb, amsthm}
\usepackage{hyperref}

\setlength{\parindent}{0pt}
\setlength{\parskip}{5pt}

\newcommand{\N}{\mathbb{N}}

\title{IMO 2019 N4}
\author{}
\date{}

\begin{document}

\maketitle



\subsection*{Problem}

Fix some $C \geq 0$.
Find all functions $f : \N \to \N$ such that $a + f(b) \mid a^2 + f(a) b$ for any $a, b \in \N$ satisfying $a + b > C$.



\subsection*{Answer}

$n \mapsto kn$ for some $k \in \N$, regardless of $C$.



\subsection*{Solution}

Official solution: \url{https://www.imo-official.org/problems/IMO2019SL.pdf}

We follow Solution 1 of the official solution.
We have a slightly different problem, but after obtaining $f(0) = 0$, this problem is essentially the same as the original problem with the exception that $f$ may attain zero at non-zero input.
The official solution implicitly uses pigeonhole principle after obtaining $p \mid f(p)$ for all $p$ prime.
We will instead explicitly show that $f(p) = p f(1)$ for every $p$ prime large enough.

Plugging $a = 1$ gives $1 + f(b) \mid 1 + f(1) b$, which implies $f(b) \leq f(1) b$ for $b > C$ since $1 + b f(1) > 0$.
We now show that $b \mid f(b)^2$ for all $b > 0$.
Since $b > 0$, there exists $n \in \N$ such that $nb > C + f(b)$.
Then plugging $a = nb - f(b)$ yields $b \mid a^2$ and thus $b \mid f(b)^2$ since $b \mid a + f(b)$.
In particular, for any $p$ prime, $p \mid f(p)$.

We claim that $f(p) = f(1) p$ for any $p$ prime with $p > \max\{C, f(1)\}$.
Write $f(p) = kp$ for some $k \in \N$; note that $k \leq f(1)$ since $p > C$ and thus $f(p) \leq f(1) p$.
Then plugging $a = 1$ again and $b = p$ yields $1 + kp \mid 1 + f(1) p \iff 1 + kp \mid (f(1) - k) p$.
But $1 + kp$ and $p$ are coprime, so $1 + kp \mid f(1) - k$.
Since $p > f(1)$, this forces $k \in \{0, f(1)\}$.

Now suppose that $k = 0$, i.e., $f(p) = 0$.
Plugging $a = p$ and $b = 1$ yields $p + f(1) \mid p^2$, which means $f(1) \in \{0, p(p - 1)\}$.
But $f(1) < p$, so this means $f(1) = 0$, which also means $f(p) = 0 = f(1) p$ regardless.

Finally, we prove that $f(n) = f(1) n$ for all $n \in \N$.
We split into three cases.

\begin{itemize}

    \item
    $f(1) = 0$.

    Pick $p$ large enough with $p > \max\{C, f(n)\}$.
    The previous claim yields $f(p) = 0$.
    Plugging $a = p$ and $b = n$ yields $p + f(n) \mid p^2$.
    Since $p > f(n)$ by our choice of $p$, this yields $f(n) = 0 = f(1) n$.

    \item 
    $n = 0$.
    
    Let $p > \max\{C, f(0)\}$ be a prime, and plug $a = p$ and $b = 0$.
    We get $p + f(0) \mid p^2$, and $p > f(0)$ yields $f(0) = 0$.

    \item
    $f(1), n > 0$.

    Let $p$ be a large enough prime, and plug $a = n$ and $b = p$.
    We get $n + f(1) p \mid n^2 + f(n) p$; in particular $n (n + f(1) p) \equiv n^2 + f(n) p \pmod{n + f(1) p}$.
    For $p$ large, $\gcd(n + f(1) p, p) = 1$, so this yields $n f(1) \equiv f(n) \pmod{n + f(1) p}$.
    Finally, for $p$ large enough with respect to $n$, this forces $f(n) = f(1) n$.

\end{itemize}



\subsection*{Extra notes}

The original problem can actually be deduced from this alternative problem.
Indeed, fix some $g : \N^+ \to \N^+$ satisfying the original condition.
Extend $g$ to $f : \N \to \N$ by defining $f(0) = 0$.
Then one could verify that $f$ satisfies the above condition.
Thus, by our solution above, $f = n \mapsto kn$ for some $k \in \N$.
The same holds for $g$, and $k > 0$ since $f \neq 0$.



\end{document}
