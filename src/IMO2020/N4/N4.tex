\documentclass{article}

\usepackage{amsmath, amsfonts, amssymb, amsthm}
\usepackage{hyperref}

\setlength{\parindent}{0pt}
\setlength{\parskip}{5pt}

\newtheorem{theorem}{Theorem}
\newtheorem{lemma}[theorem]{Lemma}

\DeclareMathOperator*{\rad}{rad}

\newcommand{\N}{\mathbb{N}}

\title{IMO 2020 N4 \\ Generalized Version}
\date{}
\author{}


\begin{document}

\maketitle



\subsection*{Problem}

Fix an odd positive integer $p > 1$.
For any $n \in \N$, let $d_p(n) \in \{0, 1, \ldots, p - 1\}$ be the remainder when $n$ is divided by $p$, and denote $F_p(n) = n + d_p(n)$.

\begin{enumerate}

\item
A pair $(a_1, a_2)$ of non-negative integers is called \emph{$p$-alternating} if $F_p^n(a_1) > F_p^n(a_2)$ for infinitely many $n$ and $F_p^n(a_1) < F_p^n(a_2)$ for infinitely many $n$.
We call $p$ \emph{good} if there exists a $p$-alternating pair $(a_1, a_2)$ with $a_1$ and $a_2$ coprime with $p$.
Prove that $p$ is good if and only if $p \notin \{1, 3, 7, 15\}$.

\item
We say that $p$ is \emph{balanced} if there exists no positive integers $a_1 > a_2$ coprime with $p$ such that $F_p^n(a_1) < F_p^n(a_2)$ for all $n \geq 1$.
Given that $p$ is a prime, prove that $p$ is balanced if and only if the order of $2$ modulo $p$ is even.

\end{enumerate}



\subsection*{Solution}

We need to set up some common notation for both parts.
Let $T$ denote the order of $2$ modulo $p$.
Next, for any $n \in \N$, let $S_p(n) = \displaystyle \sum_{i = 0}^{T - 1} d_p(2^i n)$.
The following properties of $F_p$ and $S_p$ are easy to prove.
\begin{itemize}

\item
For any $n \in \N$, $d_p(F_p(n)) = d_p(2n)$.
In particular, for all $n, k \in \N$, $d_p(F_p^k(n)) = d_p(2^k n)$.

\item
$F_p$ is injective.
Indeed, taking modulo $p$, $F_p(m) = F_p(n)$ yields $d_p(2m) = d_p(2n)$ and thus $d_p(m) = d_p(n)$.
So we can write $m = q_1 p + r$ and $n = q_2 p + r$, where $q_1, q_2, r \in \N$ with $r < p$.
Then $F_p(m) = F_p(n)$ yields $q_1 = q_2$ and thus $m = n$.

\item
For any $k, m, n \in \N$ such that $k \leq m$, $F_p^k(n) \leq F_p^m(n)$.
This follows from $F_p(n) \geq n$ for any $n \in \N$.

\item
For any $n \in \N$, $S_p(2n) = S_p(n) = S_p(d_p(n))$.
In particular, we also have $S_p(F_p(n)) = S_p(n)$.

\item
For any $n \in \N$, $F_p^T(n) = n + S_p(n)$.
More generally, $F_p^{kT + r}(n) = F_p^r(n) + k S_p(n)$ for any $k, r, n \in \N$.

\item
For any $m, n \in \N$ not divisible by $p$ such that $p \mid m + n$, $S_p(m) + S_p(n) = pT$.
Indeed, since $p$ is odd, this follows from the fact that $d_p(m) + d_p(n) = p$ if $p \mid m + n$ and $p \nmid m$.
This will only be used in Part 2.

\end{itemize}

The following lemma is the main ingredient in finding balanced odd numbers.
However, it turns out to be useful in formalizing the solution for Part 1 as well.

\begin{lemma}\label{2020n4-1}
For any positive integers $a$ and $b$ such that $S_p(a) < S_p(b)$, there exists $N \geq 0$ such that $F_p^n(a) < F_p^n(b)$ for all $n \geq N$.
\end{lemma}
\begin{proof}
Since $S_p(a) < S_p(b)$, there exists $K$ large enough such that $a + (K + 1) S_p(a) < b + K S_p(b)$.
This is just saying that $F_p^{(K + 1)T}(a) < F_p^{KT}(b)$.
We claim that $N = KT$ works.

Fix some $n \geq KT$.
We can write $n$ as $kT + r$, where $k \geq K$ and $0 \leq r < T$.
One can easily compute and get $F_p^{(k + 1)T}(a) < F_p^{kT}(b)$ since $k \geq K$.
Then, since $F_p(x) \geq x$ for all $x$ and $kT \leq n < (k + 1)T$, we get $F_p^n(a) < F_p^{(k + 1)T}(a) < F_p^{kT}(b) \leq F_p^n(b)$.
The claim is proved.
\end{proof}


\subsubsection*{Part 1}

We first reduce goodness to an alternative criterion which is easier to compute.

\begin{theorem}\label{2020n4-2}
For any $p$ odd, $p$ is good if and only if there exists positive integers $x$ and $y$ coprime with $p$ such that $p > y > p/2 + x$ and $S_p(x) = S_p(y)$.
\end{theorem}
\begin{proof}
First, suppose that such $x$ and $y$ exists.
We claim that $(p + x, y)$ is a $p$-alternating pair with $p + x$ and $y$ coprime with $p$.
The latter is clear, so we just prove that $(p + x, y)$ is $p$-alternating.
Indeed, for any $k \geq 0$,
\[ F_p^{kT}(p + x) = p + x + k S_p(x) > y + k S_p(y) = F_p^{kT}(y), \]
\[ F_p^{kT + 1}(p + x) = F_p(p + x) + k S_p(x) = p + 2x + k S_p(y) < 2y + k S_p(y) = F_p^{kT + 1}(y). \]

For the converse, suppose that there exists a $p$-alternating pair $(a_1, a_2)$, with $a_1$ and $a_2$ coprime with $p$.
Without loss of generality, suppose that $a_1 > a_2$.
Lemma~\ref{2020n4-1} then yields that $S_p(a_1) = S_p(a_2)$.
Going back to the property of $p$-alternating pairs, there must exist $N \in \N$ such that $F_p^N(a_1) < F_p^N(a_2)$.
Take $N$ to be minimal with respect to this property, so $F_p^n(a_1) \geq F_p^n(a_2)$ for any $n < N$.
Due to $F_p$ being injective, this inequality is actually strict.
Since $a_1 > a_2$, we have $N > 0$.

Now, write $F_p^{N - 1}(a_i) = q_i p + r_i$ for $i = 1, 2$, where $0 \leq r_1, r_2 < p$.
Then we have $q_1 p + r_1 > q_2 p + r_2$ and $q_1 p + 2 r_1 = F_p^N(a_1) < F_p^N(a_2) = q_2 p + 2 r_2$.
Clearly, this means $r_1 < r_2$, and the both inequalities yield $q_1 > q_2$ and $q_1 \leq q_2 + 2$, respectively.
This forces $q_1 = q_2 + 1$, and the second inequality reduces to $p + 2 r_1 < 2 r_2$, or $r_2 > p/2 + r_1$.
Thus, we can take $x = r_1$ and $y = r_2$.
\end{proof}

This means that the rest of the solution is about finding $x$ and $y$ as in Theorem~\ref{2020n4-2}, or proving that they do not exist.
There are several cases; we present all of them below.

\begin{itemize}

    \item
    \textit{\underline{Case 1.}}
    $p + 1$ is not a power of $2$.
    That is, $p \neq 2^k - 1$ for any $k \in \N$.

    Pick $x = 1$ and $y = 2^d$, where $d = \lfloor \log_2 p \rfloor$ is the non-negative integer such that $2^d \leq p < 2^{d + 1}$.
    Clearly, $S_p(1) = S_p(2^d)$, so it remains to check that $p > 2^d > p/2 + 1$.
    Note that, since $p$ is odd and $p \neq 2^k - 1$ for any $k \in \N$, we have $2^d < p < 2^{d + 1} - 1$.
    This implies the desired inequality.

    \item
    \textit{\underline{Case 2.}}
    $p = 2^k - 1$ for some $k \geq 5$.

    First, suppose that $k = 6$.
    Then take $x = 5$, $y = 40 = 2^3 \cdot 5$, and one can verify all the desired properties.

    Now, suppose that $k \neq 6$.
    We will use the fact that $\phi(k) > 2$ for $k \geq 5$, $k \neq 6$, where $\phi$ is the Euler's totient function.
    This means that there exists a positive integer $c$ such that $1 < c < k - 1$ and $\gcd(c, k) = 1$.
    We take $x = 2^c - 1$ and $y = 2^{k - c} (2^c - 1) = 2^k - 2^{k - c}$.
    It is evident that $p > y$, $S_p(x) = S_p(y)$, and $\gcd(x, p) = \gcd(y, p)$.
    Furthermore, $\gcd(x, p) = \gcd(2^c - 1, 2^k - 1) = 2^{\gcd(c, k)} - 1$.
    But $c$ and $k$ are coprime, so $x$ and $p$ must also be coprime.
    Finally, since $k \geq 5$, we have
    \[ y - x = 2^k - 2^{k - c} - 2^c + 1 \geq 2^k - 2^{k - 2} - 3 > \frac{2^k - 1}{2}. \]

    \item
    \textit{\underline{Case 3.}}
    $p = 2^k - 1$ for some $k \leq 4$.

    The case $k \leq 2$ is easy.
    There are no positive integers coprime with $p$ such that $p > y > p/2 + x$ in these cases.
    One can check by hand that this holds.

    The case $k = 3$ and $k = 4$ can also be done by hand.
    However, the cleanest proof is as follows.
    In both cases, first check that every $z < p$ coprime with $p$ can be written as either $2^m$ or $p - 2^m$ for some $0 \leq m < k$.
    In particular, $S_p(z)$ must be equal to either $S_p(1) = p$ or $S_p(p - 1) = p(k - 1) \neq p$.
    Thus, $S_p(x) = S_p(y)$ forces either $(x, y) = (2^m, 2^n)$ for some $0 \leq m, n < k$ or $(x, y) = (p - 2^m, p - 2^n)$ for some $0 \leq m, n < k$.
    In both cases, one can check that $y > p/2 + x \iff y \geq 2^{k - 1} + x$ cannot hold.

\end{itemize}


\subsubsection*{Part 2}

The balanced criterion can also be reduced to an equivalent, easier to compute criterion.
The following theorem describes this criterion.

\begin{theorem}\label{2020n4-3}
Let $p$ be an odd positive integer and $T$ be the order of $2$ modulo $p$.
Then $p$ is balanced if and only if $S_p(x) = S_p(y)$ for all $x, y \in \N$ coprime with $p$.
\end{theorem}
\begin{proof}
We prove the contrapositive of the statement.
First suppose that there exists $a, b \in \N$ coprime with $p$ such that $a > b$ and $F_p^n(a) < F_p^n(b)$ for all $n \geq 1$.
Then $a + S_p(a) = F_p^T(a) < F_p^T(b) = b + S_p(b)$ implies $S_p(a) < S_p(b)$.
This means we can take $x = a$ and $y = b$.

Conversely, suppose that $S_p(x) \neq S_p(y)$ for some $x, y \in \N$ coprime with $p$.
Without loss of generality, suppose that $S_p(x) < S_p(y)$.
First, we can take some positive integers $c > d$ such that $c \equiv x \pmod{p}$ and $d \equiv y \pmod{p}$.
For example, we can take $c = x + py > y$ and $d = y$.
By Lemma~\ref{2020n4-1}, there exists $N \geq 0$ such that $F_p^n(c) < F_p^n(d)$ for all $n \geq N$.
Take a minimal such $N$, and note that $N > 0$ since $c > d$.
Then we can take $a = F_p^{N - 1}(c)$ and $b = F_p^{N - 1}(d)$.
They are coprime with $p$ since $c$ and $d$ must be coprime with $p$.
By minimality of $N$, $a \geq b$, but $F_p^n(a) < F_p^n(b)$ for all $n \geq 1$.
Finally, the first inequality is in fact strict by injectivity of $F_p$. 
\end{proof}

\begin{theorem}\label{2020n4-4}
With the same notation as in Theorem~\ref{2020n4-3}, $p$ is balanced if and only if $2 S_p(x) = pT$ for all $x \in \N$ coprime with $p$.
\end{theorem}
\begin{proof}
We use Theorem~\ref{2020n4-4} as the criterion of $p$ being balanced.
Clearly, if $2 S_p(x) = pT$ for any $x$ coprime with $p$, then $S_p(x) = S_p(y)$ for all $x, y \in \N$ coprime with $p$.
For the converse, we get $2 S_p(x) = S_p(x) + S_p(p - d_p(x)) = pT$ for any $x \in \N$ coprime with $p$.
\end{proof}

We continue with more lemmmas.
We will still not assume that $p$ is prime; this condition is used only at the very end of the solution.

\begin{lemma}\label{2020n4-5}
Suppose that $-1$ is a power of $2$ modulo $p$.
Then $p$ is balanced.
\end{lemma}
\begin{proof}
Indeed, by Theorem~\ref{2020n4-4}, it suffices to show that $2 S_p(x) = pT$ for any $x \in \N$ coprime with $p$.
Take any $y \in \N$ such that $p \mid x + y$, e.g. $y = p - d_p(x)$.
Take some $m \in \N$ such that $2^m \equiv -1 \pmod{p}$.
Then $y \equiv 2^m x \pmod{p}$, which implies $S_p(y) = S_p(2^m x) = S_p(x)$.
This means $2 S_p(x) = S_p(x) + S_p(y) = pT$ since $p \mid x + y$.
\end{proof}

\begin{lemma}\label{2020n4-6}
If $p$ is balanced, then $T$, the order of $2$ modulo $p$, is even.
\end{lemma}
\begin{proof}
Well, by Theorem~\ref{2020n4-4}, $2 S_p(1) = pT$.
So, $pT$ is even, but $p$ is odd.
Thus, $T$ must be even.
\end{proof}

Finally, if $p$ is prime (or a prime power), then $T$ being even does imply that $-1$ is a power of $2$ modulo $p$.
Indeed, $p$ divides $(2^{T/2} - 1)(2^{T/2} + 1)$, but $p$ does not divide $2^{T/2} - 1$, so $p$ must divide $2^{T/2} + 1$.



\subsection*{Implementation details}

We have quite a big result here, so we will use multiple \texttt{.lean} files.
The definition and basic properties of $F_p$ and $S_p$ are provided in \texttt{N4\_basic.lean}.
This includes some properties of $S_p$ when $p$ is a Mersenne number.
Actually, $S_p$ is implemented as \texttt{S0} instead of \texttt{S}.
We use \texttt{S} for the general sum of form
\[ \sum_{i = 0}^{k - 1} d_p(2^i n). \]

We solve Part 1 and Part 2 separately.
Part 1 goes to \texttt{N4\_part1.lean}.
As Theorem~\ref{2020n4-2} is big enough, we will use smaller lemmas.
Similarly, the case divisions are big enough that we will need lemmas for each case.
One general result that is used in here is that $\phi(n) \leq 2$ iff $n \leq 6$ and $n \neq 5$.
We will prove bounds on the totient function in an extra file:
\[ \texttt{extra/number\_theory/totient\_bound.lean}. \]

Part 2 goes to \texttt{N4\_part2.lean}.
We solve Part 2 for not only primes, but also odd prime powers.
We will implement Lemma~\ref{2020n4-5} and Lemma~\ref{2020n4-6} as well; they have meanings when $p$ is not a prime power.
Furthermore, we also prove that primes congruent to $7 \pmod{8}$ are not balanced, while primes congruent to $3, 5 \pmod{8}$ are balanced.
In particular, using Dirichlet's theorem on arithmetic progressions, there exists infinitely many odd primes that are not balanced.
There also exists infinitely many odd primes that are balanced.



\subsection*{Notes}

In the original version, only the case where $p$ is prime is considered in both parts.
Furthermore, it only asks whether there are infinitely many good primes for Part 1, and infinitely many primes that are not balanced for Part 2.

The general case seems to be wild for Part 2.
When $p$ is not prime, $p$ could be balanced even if $-1$ is not a power of $2$ modulo $p$.
Some examples are $p = 69$ and $p = 141$.
In particular, being balanced is not divisor-closed.
Apparently, it is not even true that $p$ is balanced iff $\rad(p)$ is balanced, with one example case being $p = 1827$, $\rad(p) = 609$.



\end{document}
