\documentclass{article}

\usepackage{amsmath, amsfonts, amssymb, amsthm}
\usepackage{hyperref}

\setlength{\parindent}{0pt}
\setlength{\parskip}{5pt}

\newcommand{\N}{\mathbb{N}}
\newcommand{\Z}{\mathbb{Z}}

\title{IMO 2020 A6}
\author{}
\date{}

\begin{document}

\maketitle



\subsection*{Problem}

Find all functions $f : \Z \to \Z$ such that, for any $a, b \in \Z$,
\[ f^{a^2 + b^2}(a + b) = a f(a) + b f(b). \tag{*}\label{2020a6-eq0} \]



\subsection*{Answer}

$x \mapsto 0$ and $x \mapsto x + 1$.



\subsection*{Solution}

Official solution: \url{https://www.imo-official.org/problems/IMO2020SL.pdf}

We follow the official solution, but we avoid the explicit use of finiteness of orbits.
Instead, if $f$ is injective, we are done after $f^{a^2 + 1}(a - 1) = f^{a^2}(a)$.
If $f$ is not injective, then we show that the orbit $\{f^n(0) : n \in \N\}$ is bounded, and proceed similar to case 1.

Plugging $b = 0$ yields $f^{a^2}(a) = a f(a)$ for all $a \in \Z$.
Then plugging $a = -1$, $b = 0$ yields $f(-1) = 0$.
Next, plugging $b = -1$ yields $f^{a^2 + 1}(a - 1) = a f(a) = f^{a^2}(a)$ for all $a \in \Z$.
In particular, if $f$ is injective, we immediately get $f = x \mapsto x + 1$.
For the rest of the solution, we assume that $f$ is not injective.

Consider the equation $f^{(a + 1)^2 + 1}(a) = f^{(a + 1)^2}(a + 1)$.
This means that $f^{n + 1}(a) = f^n(a + 1)$ whenever $n \geq (a + 1)^2$.
From this, we can induct on $k \in \N$ to show that, for any integer $a$, there exists $N \in \N$ such that $f^{n + k}(a) = f^n(a + k)$ for all $n \geq N$.
We now show, from $f$ being non-injective, that $f^m(0) = f^{m + k}(0)$ for some $m, k \in \N$ with $k > 0$.

First, since $f$ is not injective, we can find $a, k \in \Z$ with $k > 0$ such that $f(a + k) = f(a)$.
Then for some positive integer $N$, we have $f^{n + k}(a) = f^n(a + k) = f^n(a)$ for all $n \geq N$.
If $a \geq 0$, then for $n$ big enough, we will get $f^n(a) = f^{n + a}(0)$.
If $a < 0$, replace $n$ with $n + |a|$ and for $n$ big enough, we will get $f^{n + |a|}(a) = f^n(0)$.
Either way, we have $f^{n + k}(0) = f^k(f^n(0)) = f^n(0)$ for $n$ big enough.

From this, we can show that the set $\{f^n(0) : n \in \N\}$ is finite, thus bounded.
Say that for some $N \in \N$ we have $|f^n(0)| < N$ for any $n \in \N$.
For any integer $a$, plugging $b = -a$ into~\eqref{2020a6-eq0} yields $a (f(a) - f(-a)) = f^{2a^2}(0)$.
For $|a| \geq N$, this forces $f(a) = f(-a)$ and $f^{2a^2}(0) = 0$.
Plugging this equation for $a = N$ and $a = N^2 + 1$ gives us $f^{2N^2}(0) = f^{2 N^2 (N^2 + 2) + 2}(0) = 0$.
Then we get $f(f(0)) = f^2(0) = 0$, which also means that $f(a) = f(-a)$ for all integers $a$.
So now, we have that $f$ is even and $f(f(0)) = 0$.

Next, for $a \neq 0$, we have $f^{a^2}(a) = f^{a^2}(-a)$ and thus $-a f(a) = -a f(-a) = a f(a)$.
This forces $f(a) = 0$ for all $a \neq 0$.
Finally, plugging $a = b = 1$ into~\eqref{2020a6-eq0} yields $f^2(2) = 2 f(1) = 0$.
Since $f(2) = 0$, this gives us $f(0) = 0$.



\end{document}
